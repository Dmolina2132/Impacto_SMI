\documentclass[11pt]{article}

    \usepackage[breakable]{tcolorbox}
    \usepackage{parskip} % Stop auto-indenting (to mimic markdown behaviour)
    

    % Basic figure setup, for now with no caption control since it's done
    % automatically by Pandoc (which extracts ![](path) syntax from Markdown).
    \usepackage{graphicx}
    % Keep aspect ratio if custom image width or height is specified
    \setkeys{Gin}{keepaspectratio}
    % Maintain compatibility with old templates. Remove in nbconvert 6.0
    \let\Oldincludegraphics\includegraphics
    % Ensure that by default, figures have no caption (until we provide a
    % proper Figure object with a Caption API and a way to capture that
    % in the conversion process - todo).
    \usepackage{caption}
    \DeclareCaptionFormat{nocaption}{}
    \captionsetup{format=nocaption,aboveskip=0pt,belowskip=0pt}

    \usepackage{float}
    \floatplacement{figure}{H} % forces figures to be placed at the correct location
    \usepackage{xcolor} % Allow colors to be defined
    \usepackage{enumerate} % Needed for markdown enumerations to work
    \usepackage{amsmath} % Equations
    \usepackage{amssymb} % Equations
    \usepackage{textcomp} % defines textquotesingle
    % Hack from http://tex.stackexchange.com/a/47451/13684:
    \AtBeginDocument{%
        \def\PYZsq{\textquotesingle}% Upright quotes in Pygmentized code
    }
    \usepackage{upquote} % Upright quotes for verbatim code
    \usepackage{eurosym} % defines \euro


    \usepackage[left=3cm, right=3cm, top=2.5cm, bottom=2.5cm]{geometry} % Márgenes
    \usepackage{fontspec} % Para definir la tipografía
    \usepackage{setspace} % Para el espaciado entre líneas
    \usepackage{indentfirst} % Sangrado en la primera línea de cada párrafo
    \usepackage{titlesec} % Para personalizar títulos de capítulos
    \usepackage{fancyhdr} % Para configurar los números de página
    \usepackage{parskip} % Para evitar espacio entre párrafos
    \usepackage{biblatex} % Para gestión de bibliografía estilo autor-fecha
    % Configuración de la fuente Arial
    \setmainfont{Arial}
    \renewcommand{\baselinestretch}{1.0}
    \setlength{\parindent}{0.5cm}
    \setlength{\parskip}{0pt} % Sin espacio entre párrafos
    \usepackage{titlesec} % Paquete para personalizar títulos

    % Configurar el tamaño de los títulos de secciones (14 puntos)
    \titleformat{\section}
    {\normalfont\bfseries\fontsize{14pt}{16pt}}{\thesection}{1em}{}

    % Configurar el tamaño de los títulos de subsecciones (12 puntos)
    \titleformat{\subsection}
    {\normalfont\bfseries\fontsize{12pt}{14pt}}{\thesubsection}{1em}{}

    % Configurar el tamaño de los títulos de subsubsecciones (11 puntos)
    \titleformat{\subsubsection}
    {\normalfont\bfseries\fontsize{11pt}{13pt}}{\thesubsubsection}{1em}{}

    % Configuración de paginación
    \pagestyle{fancy}
    \fancyhf{}
    \fancyfoot[C]{\thepage}
    \renewcommand{\headrulewidth}{0pt}

    \usepackage{iftex}
    \ifPDFTeX
        \usepackage[T1]{fontenc}
        \IfFileExists{alphabeta.sty}{
              \usepackage{alphabeta}
          }{
              \usepackage[mathletters]{ucs}
              \usepackage[utf8x]{inputenc}
          }
    \else
        \usepackage{fontspec}
        \usepackage{unicode-math}
    \fi

    \usepackage{fancyvrb} % verbatim replacement that allows latex
    \usepackage{grffile} % extends the file name processing of package graphics
                         % to support a larger range
    \makeatletter % fix for old versions of grffile with XeLaTeX
    \@ifpackagelater{grffile}{2019/11/01}
    {
      % Do nothing on new versions
    }
    {
      \def\Gread@@xetex#1{%
        \IfFileExists{"\Gin@base".bb}%
        {\Gread@eps{\Gin@base.bb}}%
        {\Gread@@xetex@aux#1}%
      }
    }
    \makeatother
    \usepackage[Export]{adjustbox} % Used to constrain images to a maximum size
    \adjustboxset{max size={0.9\linewidth}{0.9\paperheight}}

    % The hyperref package gives us a pdf with properly built
    % internal navigation ('pdf bookmarks' for the table of contents,
    % internal cross-reference links, web links for URLs, etc.)
    \usepackage{hyperref}
    % The default LaTeX title has an obnoxious amount of whitespace. By default,
    % titling removes some of it. It also provides customization options.
    \usepackage{titling}
    \usepackage{longtable} % longtable support required by pandoc >1.10
    \usepackage{booktabs}  % table support for pandoc > 1.12.2
    \usepackage{array}     % table support for pandoc >= 2.11.3
    \usepackage{calc}      % table minipage width calculation for pandoc >= 2.11.1
    \usepackage[inline]{enumitem} % IRkernel/repr support (it uses the enumerate* environment)
    \usepackage[normalem]{ulem} % ulem is needed to support strikethroughs (\sout)
                                % normalem makes italics be italics, not underlines
    \usepackage{soul}      % strikethrough (\st) support for pandoc >= 3.0.0
    \usepackage{mathrsfs}
    

    
    % Colors for the hyperref package
    \definecolor{urlcolor}{rgb}{0,.145,.698}
    \definecolor{linkcolor}{rgb}{.71,0.21,0.01}
    \definecolor{citecolor}{rgb}{.12,.54,.11}

    % ANSI colors
    \definecolor{ansi-black}{HTML}{3E424D}
    \definecolor{ansi-black-intense}{HTML}{282C36}
    \definecolor{ansi-red}{HTML}{E75C58}
    \definecolor{ansi-red-intense}{HTML}{B22B31}
    \definecolor{ansi-green}{HTML}{00A250}
    \definecolor{ansi-green-intense}{HTML}{007427}
    \definecolor{ansi-yellow}{HTML}{DDB62B}
    \definecolor{ansi-yellow-intense}{HTML}{B27D12}
    \definecolor{ansi-blue}{HTML}{208FFB}
    \definecolor{ansi-blue-intense}{HTML}{0065CA}
    \definecolor{ansi-magenta}{HTML}{D160C4}
    \definecolor{ansi-magenta-intense}{HTML}{A03196}
    \definecolor{ansi-cyan}{HTML}{60C6C8}
    \definecolor{ansi-cyan-intense}{HTML}{258F8F}
    \definecolor{ansi-white}{HTML}{C5C1B4}
    \definecolor{ansi-white-intense}{HTML}{A1A6B2}
    \definecolor{ansi-default-inverse-fg}{HTML}{FFFFFF}
    \definecolor{ansi-default-inverse-bg}{HTML}{000000}

    % common color for the border for error outputs.
    \definecolor{outerrorbackground}{HTML}{FFDFDF}

    % commands and environments needed by pandoc snippets
    % extracted from the output of `pandoc -s`
    \providecommand{\tightlist}{%
      \setlength{\itemsep}{0pt}\setlength{\parskip}{0pt}}
    \DefineVerbatimEnvironment{Highlighting}{Verbatim}{commandchars=\\\{\}}
    % Add ',fontsize=\small' for more characters per line
    \newenvironment{Shaded}{}{}
    \newcommand{\KeywordTok}[1]{\textcolor[rgb]{0.00,0.44,0.13}{\textbf{{#1}}}}
    \newcommand{\DataTypeTok}[1]{\textcolor[rgb]{0.56,0.13,0.00}{{#1}}}
    \newcommand{\DecValTok}[1]{\textcolor[rgb]{0.25,0.63,0.44}{{#1}}}
    \newcommand{\BaseNTok}[1]{\textcolor[rgb]{0.25,0.63,0.44}{{#1}}}
    \newcommand{\FloatTok}[1]{\textcolor[rgb]{0.25,0.63,0.44}{{#1}}}
    \newcommand{\CharTok}[1]{\textcolor[rgb]{0.25,0.44,0.63}{{#1}}}
    \newcommand{\StringTok}[1]{\textcolor[rgb]{0.25,0.44,0.63}{{#1}}}
    \newcommand{\CommentTok}[1]{\textcolor[rgb]{0.38,0.63,0.69}{\textit{{#1}}}}
    \newcommand{\OtherTok}[1]{\textcolor[rgb]{0.00,0.44,0.13}{{#1}}}
    \newcommand{\AlertTok}[1]{\textcolor[rgb]{1.00,0.00,0.00}{\textbf{{#1}}}}
    \newcommand{\FunctionTok}[1]{\textcolor[rgb]{0.02,0.16,0.49}{{#1}}}
    \newcommand{\RegionMarkerTok}[1]{{#1}}
    \newcommand{\ErrorTok}[1]{\textcolor[rgb]{1.00,0.00,0.00}{\textbf{{#1}}}}
    \newcommand{\NormalTok}[1]{{#1}}

    % Additional commands for more recent versions of Pandoc
    \newcommand{\ConstantTok}[1]{\textcolor[rgb]{0.53,0.00,0.00}{{#1}}}
    \newcommand{\SpecialCharTok}[1]{\textcolor[rgb]{0.25,0.44,0.63}{{#1}}}
    \newcommand{\VerbatimStringTok}[1]{\textcolor[rgb]{0.25,0.44,0.63}{{#1}}}
    \newcommand{\SpecialStringTok}[1]{\textcolor[rgb]{0.73,0.40,0.53}{{#1}}}
    \newcommand{\ImportTok}[1]{{#1}}
    \newcommand{\DocumentationTok}[1]{\textcolor[rgb]{0.73,0.13,0.13}{\textit{{#1}}}}
    \newcommand{\AnnotationTok}[1]{\textcolor[rgb]{0.38,0.63,0.69}{\textbf{\textit{{#1}}}}}
    \newcommand{\CommentVarTok}[1]{\textcolor[rgb]{0.38,0.63,0.69}{\textbf{\textit{{#1}}}}}
    \newcommand{\VariableTok}[1]{\textcolor[rgb]{0.10,0.09,0.49}{{#1}}}
    \newcommand{\ControlFlowTok}[1]{\textcolor[rgb]{0.00,0.44,0.13}{\textbf{{#1}}}}
    \newcommand{\OperatorTok}[1]{\textcolor[rgb]{0.40,0.40,0.40}{{#1}}}
    \newcommand{\BuiltInTok}[1]{{#1}}
    \newcommand{\ExtensionTok}[1]{{#1}}
    \newcommand{\PreprocessorTok}[1]{\textcolor[rgb]{0.74,0.48,0.00}{{#1}}}
    \newcommand{\AttributeTok}[1]{\textcolor[rgb]{0.49,0.56,0.16}{{#1}}}
    \newcommand{\InformationTok}[1]{\textcolor[rgb]{0.38,0.63,0.69}{\textbf{\textit{{#1}}}}}
    \newcommand{\WarningTok}[1]{\textcolor[rgb]{0.38,0.63,0.69}{\textbf{\textit{{#1}}}}}


    % Define a nice break command that doesn't care if a line doesn't already
    % exist.
    \def\br{\hspace*{\fill} \\* }
    % Math Jax compatibility definitions
    \def\gt{>}
    \def\lt{<}
    \let\Oldtex\TeX
    \let\Oldlatex\LaTeX
    \renewcommand{\TeX}{\textrm{\Oldtex}}
    \renewcommand{\LaTeX}{\textrm{\Oldlatex}}
    % Document parameters
    % Document title

    
    
    
    
    
    
    
% Pygments definitions
\makeatletter
\def\PY@reset{\let\PY@it=\relax \let\PY@bf=\relax%
    \let\PY@ul=\relax \let\PY@tc=\relax%
    \let\PY@bc=\relax \let\PY@ff=\relax}
\def\PY@tok#1{\csname PY@tok@#1\endcsname}
\def\PY@toks#1+{\ifx\relax#1\empty\else%
    \PY@tok{#1}\expandafter\PY@toks\fi}
\def\PY@do#1{\PY@bc{\PY@tc{\PY@ul{%
    \PY@it{\PY@bf{\PY@ff{#1}}}}}}}
\def\PY#1#2{\PY@reset\PY@toks#1+\relax+\PY@do{#2}}

\@namedef{PY@tok@w}{\def\PY@tc##1{\textcolor[rgb]{0.73,0.73,0.73}{##1}}}
\@namedef{PY@tok@c}{\let\PY@it=\textit\def\PY@tc##1{\textcolor[rgb]{0.24,0.48,0.48}{##1}}}
\@namedef{PY@tok@cp}{\def\PY@tc##1{\textcolor[rgb]{0.61,0.40,0.00}{##1}}}
\@namedef{PY@tok@k}{\let\PY@bf=\textbf\def\PY@tc##1{\textcolor[rgb]{0.00,0.50,0.00}{##1}}}
\@namedef{PY@tok@kp}{\def\PY@tc##1{\textcolor[rgb]{0.00,0.50,0.00}{##1}}}
\@namedef{PY@tok@kt}{\def\PY@tc##1{\textcolor[rgb]{0.69,0.00,0.25}{##1}}}
\@namedef{PY@tok@o}{\def\PY@tc##1{\textcolor[rgb]{0.40,0.40,0.40}{##1}}}
\@namedef{PY@tok@ow}{\let\PY@bf=\textbf\def\PY@tc##1{\textcolor[rgb]{0.67,0.13,1.00}{##1}}}
\@namedef{PY@tok@nb}{\def\PY@tc##1{\textcolor[rgb]{0.00,0.50,0.00}{##1}}}
\@namedef{PY@tok@nf}{\def\PY@tc##1{\textcolor[rgb]{0.00,0.00,1.00}{##1}}}
\@namedef{PY@tok@nc}{\let\PY@bf=\textbf\def\PY@tc##1{\textcolor[rgb]{0.00,0.00,1.00}{##1}}}
\@namedef{PY@tok@nn}{\let\PY@bf=\textbf\def\PY@tc##1{\textcolor[rgb]{0.00,0.00,1.00}{##1}}}
\@namedef{PY@tok@ne}{\let\PY@bf=\textbf\def\PY@tc##1{\textcolor[rgb]{0.80,0.25,0.22}{##1}}}
\@namedef{PY@tok@nv}{\def\PY@tc##1{\textcolor[rgb]{0.10,0.09,0.49}{##1}}}
\@namedef{PY@tok@no}{\def\PY@tc##1{\textcolor[rgb]{0.53,0.00,0.00}{##1}}}
\@namedef{PY@tok@nl}{\def\PY@tc##1{\textcolor[rgb]{0.46,0.46,0.00}{##1}}}
\@namedef{PY@tok@ni}{\let\PY@bf=\textbf\def\PY@tc##1{\textcolor[rgb]{0.44,0.44,0.44}{##1}}}
\@namedef{PY@tok@na}{\def\PY@tc##1{\textcolor[rgb]{0.41,0.47,0.13}{##1}}}
\@namedef{PY@tok@nt}{\let\PY@bf=\textbf\def\PY@tc##1{\textcolor[rgb]{0.00,0.50,0.00}{##1}}}
\@namedef{PY@tok@nd}{\def\PY@tc##1{\textcolor[rgb]{0.67,0.13,1.00}{##1}}}
\@namedef{PY@tok@s}{\def\PY@tc##1{\textcolor[rgb]{0.73,0.13,0.13}{##1}}}
\@namedef{PY@tok@sd}{\let\PY@it=\textit\def\PY@tc##1{\textcolor[rgb]{0.73,0.13,0.13}{##1}}}
\@namedef{PY@tok@si}{\let\PY@bf=\textbf\def\PY@tc##1{\textcolor[rgb]{0.64,0.35,0.47}{##1}}}
\@namedef{PY@tok@se}{\let\PY@bf=\textbf\def\PY@tc##1{\textcolor[rgb]{0.67,0.36,0.12}{##1}}}
\@namedef{PY@tok@sr}{\def\PY@tc##1{\textcolor[rgb]{0.64,0.35,0.47}{##1}}}
\@namedef{PY@tok@ss}{\def\PY@tc##1{\textcolor[rgb]{0.10,0.09,0.49}{##1}}}
\@namedef{PY@tok@sx}{\def\PY@tc##1{\textcolor[rgb]{0.00,0.50,0.00}{##1}}}
\@namedef{PY@tok@m}{\def\PY@tc##1{\textcolor[rgb]{0.40,0.40,0.40}{##1}}}
\@namedef{PY@tok@gh}{\let\PY@bf=\textbf\def\PY@tc##1{\textcolor[rgb]{0.00,0.00,0.50}{##1}}}
\@namedef{PY@tok@gu}{\let\PY@bf=\textbf\def\PY@tc##1{\textcolor[rgb]{0.50,0.00,0.50}{##1}}}
\@namedef{PY@tok@gd}{\def\PY@tc##1{\textcolor[rgb]{0.63,0.00,0.00}{##1}}}
\@namedef{PY@tok@gi}{\def\PY@tc##1{\textcolor[rgb]{0.00,0.52,0.00}{##1}}}
\@namedef{PY@tok@gr}{\def\PY@tc##1{\textcolor[rgb]{0.89,0.00,0.00}{##1}}}
\@namedef{PY@tok@ge}{\let\PY@it=\textit}
\@namedef{PY@tok@gs}{\let\PY@bf=\textbf}
\@namedef{PY@tok@ges}{\let\PY@bf=\textbf\let\PY@it=\textit}
\@namedef{PY@tok@gp}{\let\PY@bf=\textbf\def\PY@tc##1{\textcolor[rgb]{0.00,0.00,0.50}{##1}}}
\@namedef{PY@tok@go}{\def\PY@tc##1{\textcolor[rgb]{0.44,0.44,0.44}{##1}}}
\@namedef{PY@tok@gt}{\def\PY@tc##1{\textcolor[rgb]{0.00,0.27,0.87}{##1}}}
\@namedef{PY@tok@err}{\def\PY@bc##1{{\setlength{\fboxsep}{\string -\fboxrule}\fcolorbox[rgb]{1.00,0.00,0.00}{1,1,1}{\strut ##1}}}}
\@namedef{PY@tok@kc}{\let\PY@bf=\textbf\def\PY@tc##1{\textcolor[rgb]{0.00,0.50,0.00}{##1}}}
\@namedef{PY@tok@kd}{\let\PY@bf=\textbf\def\PY@tc##1{\textcolor[rgb]{0.00,0.50,0.00}{##1}}}
\@namedef{PY@tok@kn}{\let\PY@bf=\textbf\def\PY@tc##1{\textcolor[rgb]{0.00,0.50,0.00}{##1}}}
\@namedef{PY@tok@kr}{\let\PY@bf=\textbf\def\PY@tc##1{\textcolor[rgb]{0.00,0.50,0.00}{##1}}}
\@namedef{PY@tok@bp}{\def\PY@tc##1{\textcolor[rgb]{0.00,0.50,0.00}{##1}}}
\@namedef{PY@tok@fm}{\def\PY@tc##1{\textcolor[rgb]{0.00,0.00,1.00}{##1}}}
\@namedef{PY@tok@vc}{\def\PY@tc##1{\textcolor[rgb]{0.10,0.09,0.49}{##1}}}
\@namedef{PY@tok@vg}{\def\PY@tc##1{\textcolor[rgb]{0.10,0.09,0.49}{##1}}}
\@namedef{PY@tok@vi}{\def\PY@tc##1{\textcolor[rgb]{0.10,0.09,0.49}{##1}}}
\@namedef{PY@tok@vm}{\def\PY@tc##1{\textcolor[rgb]{0.10,0.09,0.49}{##1}}}
\@namedef{PY@tok@sa}{\def\PY@tc##1{\textcolor[rgb]{0.73,0.13,0.13}{##1}}}
\@namedef{PY@tok@sb}{\def\PY@tc##1{\textcolor[rgb]{0.73,0.13,0.13}{##1}}}
\@namedef{PY@tok@sc}{\def\PY@tc##1{\textcolor[rgb]{0.73,0.13,0.13}{##1}}}
\@namedef{PY@tok@dl}{\def\PY@tc##1{\textcolor[rgb]{0.73,0.13,0.13}{##1}}}
\@namedef{PY@tok@s2}{\def\PY@tc##1{\textcolor[rgb]{0.73,0.13,0.13}{##1}}}
\@namedef{PY@tok@sh}{\def\PY@tc##1{\textcolor[rgb]{0.73,0.13,0.13}{##1}}}
\@namedef{PY@tok@s1}{\def\PY@tc##1{\textcolor[rgb]{0.73,0.13,0.13}{##1}}}
\@namedef{PY@tok@mb}{\def\PY@tc##1{\textcolor[rgb]{0.40,0.40,0.40}{##1}}}
\@namedef{PY@tok@mf}{\def\PY@tc##1{\textcolor[rgb]{0.40,0.40,0.40}{##1}}}
\@namedef{PY@tok@mh}{\def\PY@tc##1{\textcolor[rgb]{0.40,0.40,0.40}{##1}}}
\@namedef{PY@tok@mi}{\def\PY@tc##1{\textcolor[rgb]{0.40,0.40,0.40}{##1}}}
\@namedef{PY@tok@il}{\def\PY@tc##1{\textcolor[rgb]{0.40,0.40,0.40}{##1}}}
\@namedef{PY@tok@mo}{\def\PY@tc##1{\textcolor[rgb]{0.40,0.40,0.40}{##1}}}
\@namedef{PY@tok@ch}{\let\PY@it=\textit\def\PY@tc##1{\textcolor[rgb]{0.24,0.48,0.48}{##1}}}
\@namedef{PY@tok@cm}{\let\PY@it=\textit\def\PY@tc##1{\textcolor[rgb]{0.24,0.48,0.48}{##1}}}
\@namedef{PY@tok@cpf}{\let\PY@it=\textit\def\PY@tc##1{\textcolor[rgb]{0.24,0.48,0.48}{##1}}}
\@namedef{PY@tok@c1}{\let\PY@it=\textit\def\PY@tc##1{\textcolor[rgb]{0.24,0.48,0.48}{##1}}}
\@namedef{PY@tok@cs}{\let\PY@it=\textit\def\PY@tc##1{\textcolor[rgb]{0.24,0.48,0.48}{##1}}}

\def\PYZbs{\char`\\}
\def\PYZus{\char`\_}
\def\PYZob{\char`\{}
\def\PYZcb{\char`\}}
\def\PYZca{\char`\^}
\def\PYZam{\char`\&}
\def\PYZlt{\char`\<}
\def\PYZgt{\char`\>}
\def\PYZsh{\char`\#}
\def\PYZpc{\char`\%}
\def\PYZdl{\char`\$}
\def\PYZhy{\char`\-}
\def\PYZsq{\char`\'}
\def\PYZdq{\char`\"}
\def\PYZti{\char`\~}
% for compatibility with earlier versions
\def\PYZat{@}
\def\PYZlb{[}
\def\PYZrb{]}
\makeatother


    % For linebreaks inside Verbatim environment from package fancyvrb.
    \makeatletter
        \newbox\Wrappedcontinuationbox
        \newbox\Wrappedvisiblespacebox
        \newcommand*\Wrappedvisiblespace {\textcolor{red}{\textvisiblespace}}
        \newcommand*\Wrappedcontinuationsymbol {\textcolor{red}{\llap{\tiny$\m@th\hookrightarrow$}}}
        \newcommand*\Wrappedcontinuationindent {3ex }
        \newcommand*\Wrappedafterbreak {\kern\Wrappedcontinuationindent\copy\Wrappedcontinuationbox}
        % Take advantage of the already applied Pygments mark-up to insert
        % potential linebreaks for TeX processing.
        %        {, <, #, %, $, ' and ": go to next line.
        %        _, }, ^, &, >, - and ~: stay at end of broken line.
        % Use of \textquotesingle for straight quote.
        \newcommand*\Wrappedbreaksatspecials {%
            \def\PYGZus{\discretionary{\char`\_}{\Wrappedafterbreak}{\char`\_}}%
            \def\PYGZob{\discretionary{}{\Wrappedafterbreak\char`\{}{\char`\{}}%
            \def\PYGZcb{\discretionary{\char`\}}{\Wrappedafterbreak}{\char`\}}}%
            \def\PYGZca{\discretionary{\char`\^}{\Wrappedafterbreak}{\char`\^}}%
            \def\PYGZam{\discretionary{\char`\&}{\Wrappedafterbreak}{\char`\&}}%
            \def\PYGZlt{\discretionary{}{\Wrappedafterbreak\char`\<}{\char`\<}}%
            \def\PYGZgt{\discretionary{\char`\>}{\Wrappedafterbreak}{\char`\>}}%
            \def\PYGZsh{\discretionary{}{\Wrappedafterbreak\char`\#}{\char`\#}}%
            \def\PYGZpc{\discretionary{}{\Wrappedafterbreak\char`\%}{\char`\%}}%
            \def\PYGZdl{\discretionary{}{\Wrappedafterbreak\char`\$}{\char`\$}}%
            \def\PYGZhy{\discretionary{\char`\-}{\Wrappedafterbreak}{\char`\-}}%
            \def\PYGZsq{\discretionary{}{\Wrappedafterbreak\textquotesingle}{\textquotesingle}}%
            \def\PYGZdq{\discretionary{}{\Wrappedafterbreak\char`\"}{\char`\"}}%
            \def\PYGZti{\discretionary{\char`\~}{\Wrappedafterbreak}{\char`\~}}%
        }
        % Some characters . , ; ? ! / are not pygmentized.
        % This macro makes them "active" and they will insert potential linebreaks
        \newcommand*\Wrappedbreaksatpunct {%
            \lccode`\~`\.\lowercase{\def~}{\discretionary{\hbox{\char`\.}}{\Wrappedafterbreak}{\hbox{\char`\.}}}%
            \lccode`\~`\,\lowercase{\def~}{\discretionary{\hbox{\char`\,}}{\Wrappedafterbreak}{\hbox{\char`\,}}}%
            \lccode`\~`\;\lowercase{\def~}{\discretionary{\hbox{\char`\;}}{\Wrappedafterbreak}{\hbox{\char`\;}}}%
            \lccode`\~`\:\lowercase{\def~}{\discretionary{\hbox{\char`\:}}{\Wrappedafterbreak}{\hbox{\char`\:}}}%
            \lccode`\~`\?\lowercase{\def~}{\discretionary{\hbox{\char`\?}}{\Wrappedafterbreak}{\hbox{\char`\?}}}%
            \lccode`\~`\!\lowercase{\def~}{\discretionary{\hbox{\char`\!}}{\Wrappedafterbreak}{\hbox{\char`\!}}}%
            \lccode`\~`\/\lowercase{\def~}{\discretionary{\hbox{\char`\/}}{\Wrappedafterbreak}{\hbox{\char`\/}}}%
            \catcode`\.\active
            \catcode`\,\active
            \catcode`\;\active
            \catcode`\:\active
            \catcode`\?\active
            \catcode`\!\active
            \catcode`\/\active
            \lccode`\~`\~
        }
    \makeatother

    \let\OriginalVerbatim=\Verbatim
    \makeatletter
    \renewcommand{\Verbatim}[1][1]{%
        %\parskip\z@skip
        \sbox\Wrappedcontinuationbox {\Wrappedcontinuationsymbol}%
        \sbox\Wrappedvisiblespacebox {\FV@SetupFont\Wrappedvisiblespace}%
        \def\FancyVerbFormatLine ##1{\hsize\linewidth
            \vtop{\raggedright\hyphenpenalty\z@\exhyphenpenalty\z@
                \doublehyphendemerits\z@\finalhyphendemerits\z@
                \strut ##1\strut}%
        }%
        % If the linebreak is at a space, the latter will be displayed as visible
        % space at end of first line, and a continuation symbol starts next line.
        % Stretch/shrink are however usually zero for typewriter font.
        \def\FV@Space {%
            \nobreak\hskip\z@ plus\fontdimen3\font minus\fontdimen4\font
            \discretionary{\copy\Wrappedvisiblespacebox}{\Wrappedafterbreak}
            {\kern\fontdimen2\font}%
        }%

        % Allow breaks at special characters using \PYG... macros.
        \Wrappedbreaksatspecials
        % Breaks at punctuation characters . , ; ? ! and / need catcode=\active
        \OriginalVerbatim[#1,codes*=\Wrappedbreaksatpunct]%
    }
    \makeatother

    % Exact colors from NB
    \definecolor{incolor}{HTML}{303F9F}
    \definecolor{outcolor}{HTML}{D84315}
    \definecolor{cellborder}{HTML}{CFCFCF}
    \definecolor{cellbackground}{HTML}{F7F7F7}

    % prompt
    \makeatletter
    \newcommand{\boxspacing}{\kern\kvtcb@left@rule\kern\kvtcb@boxsep}
    \makeatother
    \newcommand{\prompt}[4]{
        {\ttfamily\llap{{\color{#2}[#3]:\hspace{3pt}#4}}\vspace{-\baselineskip}}
    }
    

    
    % Prevent overflowing lines due to hard-to-break entities
    \sloppy
    % Setup hyperref package
    \hypersetup{
      breaklinks=true,  % so long urls are correctly broken across lines
      colorlinks=true,
      urlcolor=urlcolor,
      linkcolor=linkcolor,
      citecolor=citecolor,
      }
    % Slightly bigger margins than the latex defaults
    
    \geometry{verbose,tmargin=1in,bmargin=1in,lmargin=1in,rmargin=1in}
    \setcounter{tocdepth}{1}
    

\begin{document}

    \title{\fontsize{14pt}{14pt}\selectfont Predicción de efectos del salario mínimo mediante algoritmos de aprendizaje automático}
    \author{Diego Molina González}
    \date{}
    \maketitle

    \section{Introducción}\label{introduccion}

    A lo largo de las últimas décadas se ha popularizado entre las
instituciones internacionales de diversos países el uso del salario
mínimo como medida paliativa a los bajos salariados percibidos entre las
clases menos pudientes.

Estos actos parten de la premisa de que el establecimiento o incremento
de un salario mínimo impulsará la capacidad adquisitiva de la clase
afectada, produciendo así una mayor demanda de bienes y servicios por
parte de la misma y en consecuencia impulsando el crecimiento económico.
Si bien es cierto que la teoría económica respalda este argumento, los
efectos negativos de estas políticas han sido ampliamente estudiados.
Desde el lado crítico hacia estas políticas se argumenta que en
determinadas circunstancias un incremento del salario mínimo puede
fomentar el empleo informal o en última instancia destruir empleo.

El presente estudio no tendrá como objetivo específico afirmar o negar
la efectividad del salario mínimo ni matizar los grupos más vulnerables
a las bondades o daños del mismo, sino alcanzar un entendimiento
relativo de las condicones adecuadas para aplicar el salario mínimo, así
como entender qué efectos tiene este salario mínimo aplicado a nivel
nacional sobre las diferentes regiones que componen el país considerando
sus particularidades económicas.

    \section{Objetivos}

    Los objetivos del presente estudio será responder a las siguientes
preguentas:

\begin{itemize}
\tightlist
\item
  ¿Cuáles son los efectos económicos generales de un incremento del
  salario mínimo en cada comunidad autónoma?
\item
  ¿Existe un punto y ritmo óptimo de subida del salario mínimo con el
  que se puedan maximizar los beneficios que aporta el mismo?
\end{itemize}

Lo que se espera obtener con el presente trabajo respondiendo a estas
preguntas no es tanto si es positivo o negativo aplicar un salario
mínimo sino cuáles son los factores económicos que pueden permitir
implementarlo con relativo éxito.

    \section{Base de Datos}

    Los datos utilizados para el presente análisis tienen como fuente
principal el Instituto Nacional de Estadística, la Agencia Tributaria
Española y el Ministerio de Seguridad Social. Los sets de datos
empleados y sus respectivas variables son:

\begin{itemize}
\item
  \textbf{Encuestas de estructura salarial:} Proporciona información de
  los salarios por hora tanto a nivel nacional como a nivel autonómico
  hasta por 3 segmentaciones, siendo las utilizadas en este caso las
  segmentaciones por por sexo y servicio. También se ha obtenido de este
  dataset el salario medio mensual por decil, comunidad autónoma y tipo
  de jornada, así como la desigualdad medida por el índice de Gini e
  índice S80/S20.
\item
  \textbf{Índices de precios de consumo:} Proporciona información sobre
  la evolución de los precios a nivel mensual, segmentado por tipo de
  índice y a nivel mensual.
\item
  \textbf{Encuesta de presupuestos familiares:} Proporciona información
  sobre el gasto medio de los hogares a nivel autonómico y de concepto
  de gasto clasificados por grupo de producto.
\item
  \textbf{Estadísticas de movilidad nacional y geografía:} Proporciona
  información acerca del número de parados, que incluye información
  también sobre los parados de larga duración.
\item
  \textbf{Encuesta de condiciones de vida}: Contiene información sobre
  las condiciones de vida de personas y hogares en España, como puede
  ser la tasa de riesgo de pobreza por comunidad autónoma, personas con
  carencia material etc.
\item
  \textbf{Estadística Estructural de Empresas}: Contiene información del
  número de empresas, así como del flujo de altas y bajas de las mismas.
\item
  \textbf{Encuesta de Población Activa}: Contiene información sobre el
  número de parados y ocupados, así como su edad y tipo de trabajo entre
  otros.
\end{itemize}

Las variables de estos test a lo largo de las diferentes segmentaciones
serán sometidas a análisis con el fin de obtener y justificar cuáles son
las más útiles y relevantes para explicar los efectos del salario
mínimo. El limpiado básico (correcciones de decimales, cambio de nombres
de columnas para facilitar su uso y algunas fusiones de tablas) se han
hecho previo al análisis que se va a realizar a continuación.

    \section{Análisis y Modelos}\label{anuxe1lisis-y-modelos}

    \subsection{Importación de paquetes}\label{importaciuxf3n-de-paquetes}

Los análisis pertinentes así como la realización de los modelos se hará
usando el lenguaje de programación python, en concreto usando las
librerías de \emph{pandas} para el análisis y \emph{sklearn} y otros
módulos más específicos para la creación y testeo de modelos. Para la
creación de gráficos ilustrativos para el análisis se utilizará
\emph{matplotlib} y \emph{seaborn}.

Además de estos paquetes, se incluye la importación de paquetes propios
con funciones que tienen como objetivo mejorar la legibilidad del
documento y cuyo contenido se puede revisar entrando en los siguientes
ficheros adjuntos al presente documento:

\begin{itemize}
\item
  \emph{plots.py}: Contiene funciones para hacer representaciones
  gráficas básicas.
\item
  \emph{data\_format.py}: Contiene funciones para el formateado de
  datos, como puede ser la fusión de las tablas que se utilizarán.
\item
  \emph{seleccion\_modelo.py}: Contiene funciones relacionadas con la
  importancia de las variables, útiles para seleccionar el modelo a
  usar.
\item
  \emph{evaluacion\_modelo.py}: Contiene funciones para la selección del
  modelo, principalmente mediante grid search.
\item
  \emph{simulacion.py}: Contiene funciones para realizar simulaciones de
  incrementos del salario mínimo a través de unos modelos dados.
\end{itemize}

    \begin{tcolorbox}[breakable, size=fbox, boxrule=1pt, pad at break*=1mm,colback=cellbackground, colframe=cellborder]
\prompt{In}{incolor}{1}{\boxspacing}
\begin{Verbatim}[commandchars=\\\{\}]
\PY{k+kn}{import} \PY{n+nn}{pandas} \PY{k}{as} \PY{n+nn}{pd}
\PY{k+kn}{import} \PY{n+nn}{numpy} \PY{k}{as} \PY{n+nn}{np}
\PY{k+kn}{import} \PY{n+nn}{warnings}
\PY{k+kn}{import} \PY{n+nn}{data\PYZus{}format} \PY{k}{as} \PY{n+nn}{dformat}
\PY{k+kn}{import} \PY{n+nn}{evaluacion\PYZus{}modelo} \PY{k}{as} \PY{n+nn}{em}
\PY{k+kn}{import} \PY{n+nn}{plots} \PY{k}{as} \PY{n+nn}{p}
\PY{k+kn}{import} \PY{n+nn}{simulacion} \PY{k}{as} \PY{n+nn}{sim}
\PY{k+kn}{import} \PY{n+nn}{seleccion\PYZus{}modelo} \PY{k}{as} \PY{n+nn}{smod}
\PY{k+kn}{from} \PY{n+nn}{pandas}\PY{n+nn}{.}\PY{n+nn}{errors} \PY{k+kn}{import} \PY{n}{SettingWithCopyWarning}

\PY{n}{warnings}\PY{o}{.}\PY{n}{simplefilter}\PY{p}{(}\PY{n}{action}\PY{o}{=}\PY{l+s+s2}{\PYZdq{}}\PY{l+s+s2}{ignore}\PY{l+s+s2}{\PYZdq{}}\PY{p}{,} \PY{n}{category}\PY{o}{=}\PY{n}{SettingWithCopyWarning}\PY{p}{)}
\PY{n}{warnings}\PY{o}{.}\PY{n}{filterwarnings}\PY{p}{(}\PY{l+s+s2}{\PYZdq{}}\PY{l+s+s2}{ignore}\PY{l+s+s2}{\PYZdq{}}\PY{p}{,} \PY{n}{category}\PY{o}{=}\PY{n+ne}{RuntimeWarning}\PY{p}{)}
\PY{k+kn}{import} \PY{n+nn}{seaborn} \PY{k}{as} \PY{n+nn}{sns}
\PY{k+kn}{import} \PY{n+nn}{matplotlib}\PY{n+nn}{.}\PY{n+nn}{pyplot} \PY{k}{as} \PY{n+nn}{plt}
\PY{k+kn}{from} \PY{n+nn}{sklearn}\PY{n+nn}{.}\PY{n+nn}{ensemble} \PY{k+kn}{import} \PY{n}{RandomForestRegressor}\PY{p}{,} \PY{n}{GradientBoostingRegressor}
\PY{k+kn}{from} \PY{n+nn}{sklearn}\PY{n+nn}{.}\PY{n+nn}{linear\PYZus{}model} \PY{k+kn}{import} \PY{n}{LinearRegression}\PY{p}{,} \PY{n}{Lasso}
\PY{k+kn}{from} \PY{n+nn}{sklearn}\PY{n+nn}{.}\PY{n+nn}{tree} \PY{k+kn}{import} \PY{n}{DecisionTreeRegressor}
\PY{k+kn}{from} \PY{n+nn}{sklearn}\PY{n+nn}{.}\PY{n+nn}{svm} \PY{k+kn}{import} \PY{n}{SVR}
\PY{k+kn}{from} \PY{n+nn}{sklearn}\PY{n+nn}{.}\PY{n+nn}{impute} \PY{k+kn}{import} \PY{n}{KNNImputer}
\end{Verbatim}
\end{tcolorbox}

    \subsection{Análisis descriptivo}\label{anuxe1lisis-descriptivo}

    La base de este análisis consistirá en considerar un grupo de variables
de partida que nos permita describir un sistema económico a través de un
determinado número de variables o ratios, y que, a través del salario
mínimo, ver como ese sistema pasa del estado A al estado B. Esto implica
que las variables predictoras también pueden ser utilizadas como
resultados en el entrenamiento, ya que nos interesa ver cómo estas
evolucionan como resultado de los cambios en el salario mínimo.

En última instancia nos interesa analizar qué pasaría si estando en el
estado inicial A con un salario mínimo interprofesional (o una medida
derivada del mismo) X que nos lleva al estado B, qué ocurriría si el SMI
en su lugar fuese Y.

    A continuación se presente el análisis general de variables. Dado que
hay segmentaciones diversas y muchas comunidades autónomas para poder
mostrar todo a la vez, se utilizará un set reducido escogiendo las tres
comunidades autónomas más ricas para mostrar y las tres más pobres.

    \begin{tcolorbox}[breakable, size=fbox, boxrule=1pt, pad at break*=1mm,colback=cellbackground, colframe=cellborder]
\prompt{In}{incolor}{2}{\boxspacing}
\begin{Verbatim}[commandchars=\\\{\}]
\PY{n}{CCAA\PYZus{}mas\PYZus{}gasto} \PY{o}{=} \PY{p}{[}\PY{l+s+s1}{\PYZsq{}}\PY{l+s+s1}{País Vasco}\PY{l+s+s1}{\PYZsq{}}\PY{p}{,} \PY{l+s+s1}{\PYZsq{}}\PY{l+s+s1}{Madrid, Comunidad de}\PY{l+s+s1}{\PYZsq{}}\PY{p}{,} \PY{l+s+s1}{\PYZsq{}}\PY{l+s+s1}{Navarra, Comunidad Foral de}\PY{l+s+s1}{\PYZsq{}}\PY{p}{]}
\PY{n}{CCAA\PYZus{}menos\PYZus{}gasto} \PY{o}{=} \PY{p}{[}\PY{l+s+s1}{\PYZsq{}}\PY{l+s+s1}{Extremadura}\PY{l+s+s1}{\PYZsq{}}\PY{p}{,} \PY{l+s+s1}{\PYZsq{}}\PY{l+s+s1}{Canarias}\PY{l+s+s1}{\PYZsq{}}\PY{p}{,} \PY{l+s+s1}{\PYZsq{}}\PY{l+s+s1}{Murcia, Región de}\PY{l+s+s1}{\PYZsq{}}\PY{p}{]}
\PY{n}{CCAA\PYZus{}mostrar} \PY{o}{=} \PY{n}{CCAA\PYZus{}mas\PYZus{}gasto}\PY{o}{+}\PY{n}{CCAA\PYZus{}menos\PYZus{}gasto}
\end{Verbatim}
\end{tcolorbox}

    \subsubsection{Salario Mínimo
Inteprofesional}\label{salario-muxednimo-inteprofesional}

    La principal variable independiente y que será el principal input será
el salario mínimo interprofesinal (SMI), que es fijado por ley y
generalmente empieza aplicar todos los años en enero o, alternativamente
se aprueba más adelante pero con efecto retroactivo desde enero.

    \begin{tcolorbox}[breakable, size=fbox, boxrule=1pt, pad at break*=1mm,colback=cellbackground, colframe=cellborder]
\prompt{In}{incolor}{3}{\boxspacing}
\begin{Verbatim}[commandchars=\\\{\}]
\PY{n}{smi} \PY{o}{=} \PY{n}{pd}\PY{o}{.}\PY{n}{read\PYZus{}csv}\PY{p}{(}\PY{l+s+s2}{\PYZdq{}}\PY{l+s+s2}{../../processed\PYZus{}data/salarios/salarios\PYZus{}smis\PYZus{}aeat.csv}\PY{l+s+s2}{\PYZdq{}}\PY{p}{)}\PY{p}{[}\PY{p}{[}\PY{l+s+s1}{\PYZsq{}}\PY{l+s+s1}{periodo}\PY{l+s+s1}{\PYZsq{}}\PY{p}{,}\PY{l+s+s1}{\PYZsq{}}\PY{l+s+s1}{smi\PYZus{}14}\PY{l+s+s1}{\PYZsq{}}\PY{p}{]}\PY{p}{]}\PY{o}{.}\PY{n}{drop\PYZus{}duplicates}\PY{p}{(}\PY{p}{)}
\end{Verbatim}
\end{tcolorbox}

    \begin{tcolorbox}[breakable, size=fbox, boxrule=1pt, pad at break*=1mm,colback=cellbackground, colframe=cellborder]
\prompt{In}{incolor}{4}{\boxspacing}
\begin{Verbatim}[commandchars=\\\{\}]
\PY{n}{p}\PY{o}{.}\PY{n}{create\PYZus{}basic\PYZus{}plot}\PY{p}{(}\PY{n}{x}\PY{o}{=}\PY{n}{smi}\PY{p}{[}\PY{l+s+s1}{\PYZsq{}}\PY{l+s+s1}{periodo}\PY{l+s+s1}{\PYZsq{}}\PY{p}{]}\PY{p}{,} \PY{n}{y}\PY{o}{=}\PY{n}{smi}\PY{p}{[}\PY{l+s+s1}{\PYZsq{}}\PY{l+s+s1}{smi\PYZus{}14}\PY{l+s+s1}{\PYZsq{}}\PY{p}{]}\PY{p}{,} \PY{n}{xlabel}\PY{o}{=}\PY{l+s+s2}{\PYZdq{}}\PY{l+s+s2}{Año}\PY{l+s+s2}{\PYZdq{}}\PY{p}{,} \PY{n}{ylabel}\PY{o}{=}\PY{l+s+s2}{\PYZdq{}}\PY{l+s+s2}{SMI}\PY{l+s+s2}{\PYZdq{}}\PY{p}{,} 
                    \PY{n}{title}\PY{o}{=}\PY{l+s+s2}{\PYZdq{}}\PY{l+s+s2}{Evolución del SMI}\PY{l+s+s2}{\PYZdq{}}\PY{p}{,} \PY{n}{xticks\PYZus{}rotation}\PY{o}{=}\PY{l+m+mi}{0}\PY{p}{,} \PY{n}{style}\PY{o}{=}\PY{l+s+s2}{\PYZdq{}}\PY{l+s+s2}{whitegrid}\PY{l+s+s2}{\PYZdq{}}\PY{p}{,} 
                    \PY{n}{color}\PY{o}{=}\PY{l+s+s2}{\PYZdq{}}\PY{l+s+s2}{dodgerblue}\PY{l+s+s2}{\PYZdq{}}\PY{p}{,} \PY{n}{figsize}\PY{o}{=}\PY{p}{(}\PY{l+m+mi}{10}\PY{p}{,} \PY{l+m+mi}{6}\PY{p}{)}\PY{p}{)}
\end{Verbatim}
\end{tcolorbox}

    \begin{center}
    \adjustimage{max size={0.9\linewidth}{0.9\paperheight}}{tfm_project_files/tfm_project_19_0.png}
    \end{center}
    { \hspace*{\fill} \\}
    
    La evolución del SMI nominal es bastante irregular, creciendo un 22.65\%
en 10 años y luego una cantidad muy similar (22.3\%) en solo un año.
Tengamos en cuenta que este SMI es en términos nominales y no tiene en
cuenta la inflación, por ello es importante considerar el ipc dentro de
la ecuación para obtener un resultado en relación de poder adquisitivo.

    \begin{tcolorbox}[breakable, size=fbox, boxrule=1pt, pad at break*=1mm,colback=cellbackground, colframe=cellborder]
\prompt{In}{incolor}{5}{\boxspacing}
\begin{Verbatim}[commandchars=\\\{\}]
\PY{n}{ipc} \PY{o}{=} \PY{n}{pd}\PY{o}{.}\PY{n}{read\PYZus{}csv}\PY{p}{(}\PY{l+s+s2}{\PYZdq{}}\PY{l+s+s2}{../../processed\PYZus{}data/gasto\PYZus{}ipc\PYZus{}ipri/ipc.csv}\PY{l+s+s2}{\PYZdq{}}\PY{p}{)}
\PY{n}{ipc\PYZus{}nacional} \PY{o}{=} \PY{n}{ipc}\PY{p}{[}\PY{p}{(}\PY{n}{ipc}\PY{p}{[}\PY{l+s+s1}{\PYZsq{}}\PY{l+s+s1}{ccaa}\PY{l+s+s1}{\PYZsq{}}\PY{p}{]} \PY{o}{==} \PY{l+s+s2}{\PYZdq{}}\PY{l+s+s2}{Nacional}\PY{l+s+s2}{\PYZdq{}}\PY{p}{)} \PY{o}{\PYZam{}} \PY{p}{(}\PY{n}{ipc}\PY{p}{[}\PY{l+s+s1}{\PYZsq{}}\PY{l+s+s1}{grupo\PYZus{}indice}\PY{l+s+s1}{\PYZsq{}}\PY{p}{]} \PY{o}{==} \PY{l+s+s2}{\PYZdq{}}\PY{l+s+s2}{Índice general}\PY{l+s+s2}{\PYZdq{}}\PY{p}{)}
                    \PY{o}{\PYZam{}} \PY{p}{(}\PY{n}{ipc}\PY{p}{[}\PY{l+s+s1}{\PYZsq{}}\PY{l+s+s1}{tipo\PYZus{}dato}\PY{l+s+s1}{\PYZsq{}}\PY{p}{]} \PY{o}{==} \PY{l+s+s2}{\PYZdq{}}\PY{l+s+s2}{Índice}\PY{l+s+s2}{\PYZdq{}}\PY{p}{)} \PY{o}{\PYZam{}} \PY{p}{(}\PY{n}{ipc}\PY{o}{.}\PY{n}{mes}\PY{o}{==}\PY{l+m+mi}{1}\PY{p}{)} \PY{o}{\PYZam{}} \PY{p}{(}\PY{n}{ipc}\PY{o}{.}\PY{n}{año}\PY{o}{\PYZgt{}}\PY{o}{=}\PY{l+m+mi}{2008}\PY{p}{)}\PY{p}{]}\PY{p}{[}\PY{p}{[}\PY{l+s+s1}{\PYZsq{}}\PY{l+s+s1}{año}\PY{l+s+s1}{\PYZsq{}}\PY{p}{,} \PY{l+s+s1}{\PYZsq{}}\PY{l+s+s1}{Total}\PY{l+s+s1}{\PYZsq{}}\PY{p}{]}\PY{p}{]}
\PY{n}{smi} \PY{o}{=} \PY{n}{smi}\PY{o}{.}\PY{n}{merge}\PY{p}{(}\PY{n}{ipc\PYZus{}nacional}\PY{p}{,} \PY{n}{left\PYZus{}on} \PY{o}{=} \PY{l+s+s2}{\PYZdq{}}\PY{l+s+s2}{periodo}\PY{l+s+s2}{\PYZdq{}}\PY{p}{,} \PY{n}{right\PYZus{}on}\PY{o}{=}\PY{l+s+s2}{\PYZdq{}}\PY{l+s+s2}{año}\PY{l+s+s2}{\PYZdq{}}\PY{p}{)}
\PY{n}{smi}\PY{p}{[}\PY{l+s+s1}{\PYZsq{}}\PY{l+s+s1}{smi\PYZus{}ajustado}\PY{l+s+s1}{\PYZsq{}}\PY{p}{]} \PY{o}{=} \PY{n}{smi}\PY{p}{[}\PY{l+s+s1}{\PYZsq{}}\PY{l+s+s1}{smi\PYZus{}14}\PY{l+s+s1}{\PYZsq{}}\PY{p}{]}\PY{o}{/}\PY{n}{smi}\PY{p}{[}\PY{l+s+s1}{\PYZsq{}}\PY{l+s+s1}{Total}\PY{l+s+s1}{\PYZsq{}}\PY{p}{]}\PY{o}{*}\PY{l+m+mi}{100}
\PY{n}{p}\PY{o}{.}\PY{n}{create\PYZus{}basic\PYZus{}plot}\PY{p}{(}\PY{n}{x}\PY{o}{=}\PY{n}{smi}\PY{p}{[}\PY{l+s+s1}{\PYZsq{}}\PY{l+s+s1}{periodo}\PY{l+s+s1}{\PYZsq{}}\PY{p}{]}\PY{p}{,} \PY{n}{y}\PY{o}{=}\PY{n}{smi}\PY{p}{[}\PY{l+s+s1}{\PYZsq{}}\PY{l+s+s1}{smi\PYZus{}ajustado}\PY{l+s+s1}{\PYZsq{}}\PY{p}{]}\PY{p}{,} \PY{n}{xlabel}\PY{o}{=}\PY{l+s+s2}{\PYZdq{}}\PY{l+s+s2}{Año}\PY{l+s+s2}{\PYZdq{}}\PY{p}{,}
                     \PY{n}{ylabel}\PY{o}{=}\PY{l+s+s2}{\PYZdq{}}\PY{l+s+s2}{SMI (€)}\PY{l+s+s2}{\PYZdq{}}\PY{p}{,} \PY{n}{title}\PY{o}{=}\PY{l+s+s2}{\PYZdq{}}\PY{l+s+s2}{SMI Ajustado por Inflación a Precios de 2021}\PY{l+s+s2}{\PYZdq{}}\PY{p}{,}
                       \PY{n}{xticks\PYZus{}rotation}\PY{o}{=}\PY{l+m+mi}{0}\PY{p}{,} \PY{n}{style}\PY{o}{=}\PY{l+s+s2}{\PYZdq{}}\PY{l+s+s2}{whitegrid}\PY{l+s+s2}{\PYZdq{}}\PY{p}{,} \PY{n}{color}\PY{o}{=}\PY{l+s+s2}{\PYZdq{}}\PY{l+s+s2}{dodgerblue}\PY{l+s+s2}{\PYZdq{}}\PY{p}{,} \PY{n}{figsize}\PY{o}{=}\PY{p}{(}\PY{l+m+mi}{10}\PY{p}{,} \PY{l+m+mi}{6}\PY{p}{)}\PY{p}{)}
\end{Verbatim}
\end{tcolorbox}

    \begin{center}
    \adjustimage{max size={0.9\linewidth}{0.9\paperheight}}{tfm_project_files/tfm_project_21_0.png}
    \end{center}
    { \hspace*{\fill} \\}
    
    Observamos ahora que la evolución del salario mínimo, pese a que
conserva el gran incremento entre 2018 y 2019, tiene variaciones
negativas en determinados años y la gamma de posibles incrementos que
aporta es algo más amplia.

    \subsubsection{IPC}\label{ipc}

    El índice de precios de consumo (IPC) describe la evolución de los
precios a lo largo del tiempo, tomando como base 100 un determinado año
y calculando el resto de valores a partir de los incrementos que se
producen año a año.

    \begin{tcolorbox}[breakable, size=fbox, boxrule=1pt, pad at break*=1mm,colback=cellbackground, colframe=cellborder]
\prompt{In}{incolor}{6}{\boxspacing}
\begin{Verbatim}[commandchars=\\\{\}]
\PY{n}{ipc\PYZus{}nacional\PYZus{}es} \PY{o}{=} \PY{n}{ipc}\PY{p}{[}\PY{p}{(}\PY{n}{ipc}\PY{p}{[}\PY{l+s+s1}{\PYZsq{}}\PY{l+s+s1}{ccaa}\PY{l+s+s1}{\PYZsq{}}\PY{p}{]} \PY{o}{==} \PY{l+s+s2}{\PYZdq{}}\PY{l+s+s2}{Nacional}\PY{l+s+s2}{\PYZdq{}}\PY{p}{)} \PY{o}{\PYZam{}} 
                      \PY{p}{(}\PY{n}{ipc}\PY{p}{[}\PY{l+s+s1}{\PYZsq{}}\PY{l+s+s1}{grupo\PYZus{}indice}\PY{l+s+s1}{\PYZsq{}}\PY{p}{]} \PY{o}{==} \PY{l+s+s2}{\PYZdq{}}\PY{l+s+s2}{Índice general}\PY{l+s+s2}{\PYZdq{}}\PY{p}{)} \PY{o}{\PYZam{}} 
                      \PY{p}{(}\PY{n}{ipc}\PY{p}{[}\PY{l+s+s1}{\PYZsq{}}\PY{l+s+s1}{tipo\PYZus{}dato}\PY{l+s+s1}{\PYZsq{}}\PY{p}{]} \PY{o}{==} \PY{l+s+s2}{\PYZdq{}}\PY{l+s+s2}{Índice}\PY{l+s+s2}{\PYZdq{}}\PY{p}{)}\PY{p}{]}\PY{p}{[}\PY{p}{[}\PY{l+s+s1}{\PYZsq{}}\PY{l+s+s1}{periodo\PYZus{}fecha}\PY{l+s+s1}{\PYZsq{}}\PY{p}{,} \PY{l+s+s1}{\PYZsq{}}\PY{l+s+s1}{Total}\PY{l+s+s1}{\PYZsq{}}\PY{p}{]}\PY{p}{]}\PY{o}{.}\PY{n}{sort\PYZus{}values}\PY{p}{(}\PY{l+s+s1}{\PYZsq{}}\PY{l+s+s1}{periodo\PYZus{}fecha}\PY{l+s+s1}{\PYZsq{}}\PY{p}{,} \PY{n}{ascending} \PY{o}{=} \PY{k+kc}{True}\PY{p}{)}

\PY{n}{p}\PY{o}{.}\PY{n}{create\PYZus{}basic\PYZus{}plot}\PY{p}{(}\PY{n}{ipc\PYZus{}nacional\PYZus{}es}\PY{p}{[}\PY{l+s+s1}{\PYZsq{}}\PY{l+s+s1}{periodo\PYZus{}fecha}\PY{l+s+s1}{\PYZsq{}}\PY{p}{]}\PY{p}{,} \PY{n}{ipc\PYZus{}nacional\PYZus{}es}\PY{p}{[}\PY{l+s+s1}{\PYZsq{}}\PY{l+s+s1}{Total}\PY{l+s+s1}{\PYZsq{}}\PY{p}{]}\PY{p}{,} \PY{n}{xlabel}\PY{o}{=}\PY{l+s+s2}{\PYZdq{}}\PY{l+s+s2}{Periodo}\PY{l+s+s2}{\PYZdq{}}\PY{p}{,} \PY{n}{ylabel}\PY{o}{=}\PY{l+s+s2}{\PYZdq{}}\PY{l+s+s2}{IPC}\PY{l+s+s2}{\PYZdq{}}\PY{p}{,} 
                  \PY{n}{title}\PY{o}{=}\PY{l+s+s2}{\PYZdq{}}\PY{l+s+s2}{Evolución del IPC (Nacional)}\PY{l+s+s2}{\PYZdq{}}\PY{p}{,} \PY{n}{xticks\PYZus{}rotation}\PY{o}{=}\PY{l+m+mi}{45}\PY{p}{,}
                    \PY{n}{style}\PY{o}{=}\PY{l+s+s2}{\PYZdq{}}\PY{l+s+s2}{whitegrid}\PY{l+s+s2}{\PYZdq{}}\PY{p}{,} \PY{n}{color}\PY{o}{=}\PY{l+s+s2}{\PYZdq{}}\PY{l+s+s2}{dodgerblue}\PY{l+s+s2}{\PYZdq{}}\PY{p}{,} \PY{n}{figsize}\PY{o}{=}\PY{p}{(}\PY{l+m+mi}{10}\PY{p}{,} \PY{l+m+mi}{6}\PY{p}{)}\PY{p}{,}
                    \PY{n}{marker} \PY{o}{=} \PY{k+kc}{None} \PY{p}{,} \PY{n}{label} \PY{o}{=} \PY{l+s+s2}{\PYZdq{}}\PY{l+s+s2}{IPC}\PY{l+s+s2}{\PYZdq{}}\PY{p}{)}
\end{Verbatim}
\end{tcolorbox}

    \begin{center}
    \adjustimage{max size={0.9\linewidth}{0.9\paperheight}}{tfm_project_files/tfm_project_25_0.png}
    \end{center}
    { \hspace*{\fill} \\}
    
    \begin{tcolorbox}[breakable, size=fbox, boxrule=1pt, pad at break*=1mm,colback=cellbackground, colframe=cellborder]
\prompt{In}{incolor}{7}{\boxspacing}
\begin{Verbatim}[commandchars=\\\{\}]
\PY{n}{p}\PY{o}{.}\PY{n}{create\PYZus{}multi\PYZus{}category\PYZus{}plot}\PY{p}{(}\PY{n}{data} \PY{o}{=} \PY{n}{ipc}\PY{p}{[}\PY{p}{(}\PY{n}{ipc}\PY{p}{[}\PY{l+s+s1}{\PYZsq{}}\PY{l+s+s1}{ccaa}\PY{l+s+s1}{\PYZsq{}}\PY{p}{]}\PY{o}{.}\PY{n}{isin}\PY{p}{(}\PY{n}{CCAA\PYZus{}mostrar}\PY{p}{)}\PY{p}{)} \PY{o}{\PYZam{}} \PY{p}{(}\PY{n}{ipc}\PY{p}{[}\PY{l+s+s1}{\PYZsq{}}\PY{l+s+s1}{tipo\PYZus{}dato}\PY{l+s+s1}{\PYZsq{}}\PY{p}{]} \PY{o}{==} \PY{l+s+s2}{\PYZdq{}}\PY{l+s+s2}{Índice}\PY{l+s+s2}{\PYZdq{}}\PY{p}{)} \PY{o}{\PYZam{}} 
                                      \PY{p}{(}\PY{n}{ipc}\PY{p}{[}\PY{l+s+s2}{\PYZdq{}}\PY{l+s+s2}{grupo\PYZus{}indice}\PY{l+s+s2}{\PYZdq{}}\PY{p}{]} \PY{o}{==} \PY{l+s+s2}{\PYZdq{}}\PY{l+s+s2}{Índice general}\PY{l+s+s2}{\PYZdq{}}\PY{p}{)}\PY{p}{]}\PY{o}{.}\PY{n}{sort\PYZus{}values}\PY{p}{(}\PY{p}{[}\PY{l+s+s1}{\PYZsq{}}\PY{l+s+s1}{periodo\PYZus{}fecha}\PY{l+s+s1}{\PYZsq{}}\PY{p}{,} \PY{l+s+s1}{\PYZsq{}}\PY{l+s+s1}{grupo\PYZus{}indice}\PY{l+s+s1}{\PYZsq{}}\PY{p}{]}\PY{p}{,}\PY{n}{ascending}\PY{o}{=}\PY{k+kc}{True}\PY{p}{)}\PY{p}{,}
                                        \PY{n}{x\PYZus{}col}\PY{o}{=}\PY{l+s+s2}{\PYZdq{}}\PY{l+s+s2}{periodo\PYZus{}fecha}\PY{l+s+s2}{\PYZdq{}}\PY{p}{,} \PY{n}{y\PYZus{}col}\PY{o}{=}\PY{l+s+s2}{\PYZdq{}}\PY{l+s+s2}{Total}\PY{l+s+s2}{\PYZdq{}}\PY{p}{,} \PY{n}{category\PYZus{}col}\PY{o}{=}\PY{l+s+s2}{\PYZdq{}}\PY{l+s+s2}{ccaa}\PY{l+s+s2}{\PYZdq{}}\PY{p}{,} \PY{n}{xlabel}\PY{o}{=}\PY{l+s+s2}{\PYZdq{}}\PY{l+s+s2}{Año}\PY{l+s+s2}{\PYZdq{}}\PY{p}{,} \PY{n}{ylabel}\PY{o}{=}\PY{l+s+s2}{\PYZdq{}}\PY{l+s+s2}{IPC}\PY{l+s+s2}{\PYZdq{}}\PY{p}{,} 
                                        \PY{n}{title}\PY{o}{=}\PY{l+s+s2}{\PYZdq{}}\PY{l+s+s2}{IPC}\PY{l+s+s2}{\PYZdq{}}\PY{p}{,} \PY{n}{xticks\PYZus{}rotation}\PY{o}{=}\PY{l+m+mi}{0}\PY{p}{,} \PY{n}{style}\PY{o}{=}\PY{l+s+s2}{\PYZdq{}}\PY{l+s+s2}{whitegrid}\PY{l+s+s2}{\PYZdq{}}\PY{p}{,} \PY{n}{figsize}\PY{o}{=}\PY{p}{(}\PY{l+m+mi}{12}\PY{p}{,} \PY{l+m+mi}{7}\PY{p}{)}\PY{p}{)}
\end{Verbatim}
\end{tcolorbox}

    \begin{center}
    \adjustimage{max size={0.9\linewidth}{0.9\paperheight}}{tfm_project_files/tfm_project_26_0.png}
    \end{center}
    { \hspace*{\fill} \\}
    
    Parece claro que la evolución se produce de manera conjunta, tanto en
tendencia como en estacionalidad. Veamos cómo es la variación
interanual.

    \begin{tcolorbox}[breakable, size=fbox, boxrule=1pt, pad at break*=1mm,colback=cellbackground, colframe=cellborder]
\prompt{In}{incolor}{8}{\boxspacing}
\begin{Verbatim}[commandchars=\\\{\}]
\PY{n}{p}\PY{o}{.}\PY{n}{create\PYZus{}multi\PYZus{}category\PYZus{}plot}\PY{p}{(}\PY{n}{data} \PY{o}{=} \PY{n}{ipc}\PY{p}{[}\PY{p}{(}\PY{n}{ipc}\PY{p}{[}\PY{l+s+s1}{\PYZsq{}}\PY{l+s+s1}{ccaa}\PY{l+s+s1}{\PYZsq{}}\PY{p}{]}\PY{o}{.}\PY{n}{isin}\PY{p}{(}\PY{n}{CCAA\PYZus{}mostrar}\PY{p}{)}\PY{p}{)} \PY{o}{\PYZam{}} \PY{p}{(}\PY{n}{ipc}\PY{p}{[}\PY{l+s+s1}{\PYZsq{}}\PY{l+s+s1}{tipo\PYZus{}dato}\PY{l+s+s1}{\PYZsq{}}\PY{p}{]} \PY{o}{==} \PY{l+s+s2}{\PYZdq{}}\PY{l+s+s2}{Variación anual}\PY{l+s+s2}{\PYZdq{}}\PY{p}{)} \PY{o}{\PYZam{}} 
                                    \PY{p}{(}\PY{n}{ipc}\PY{p}{[}\PY{l+s+s2}{\PYZdq{}}\PY{l+s+s2}{grupo\PYZus{}indice}\PY{l+s+s2}{\PYZdq{}}\PY{p}{]} \PY{o}{==} \PY{l+s+s2}{\PYZdq{}}\PY{l+s+s2}{Índice general}\PY{l+s+s2}{\PYZdq{}}\PY{p}{)}\PY{p}{]}\PY{o}{.}\PY{n}{sort\PYZus{}values}\PY{p}{(}\PY{p}{[}\PY{l+s+s1}{\PYZsq{}}\PY{l+s+s1}{periodo\PYZus{}fecha}\PY{l+s+s1}{\PYZsq{}}\PY{p}{,} \PY{l+s+s1}{\PYZsq{}}\PY{l+s+s1}{grupo\PYZus{}indice}\PY{l+s+s1}{\PYZsq{}}\PY{p}{]}\PY{p}{,} \PY{n}{ascending}\PY{o}{=}\PY{k+kc}{True}\PY{p}{)}\PY{p}{,} 
                                    \PY{n}{x\PYZus{}col}\PY{o}{=}\PY{l+s+s2}{\PYZdq{}}\PY{l+s+s2}{periodo\PYZus{}fecha}\PY{l+s+s2}{\PYZdq{}}\PY{p}{,} \PY{n}{y\PYZus{}col}\PY{o}{=}\PY{l+s+s2}{\PYZdq{}}\PY{l+s+s2}{Total}\PY{l+s+s2}{\PYZdq{}}\PY{p}{,} \PY{n}{category\PYZus{}col}\PY{o}{=}\PY{l+s+s2}{\PYZdq{}}\PY{l+s+s2}{ccaa}\PY{l+s+s2}{\PYZdq{}}\PY{p}{,} \PY{n}{xlabel}\PY{o}{=}\PY{l+s+s2}{\PYZdq{}}\PY{l+s+s2}{Año}\PY{l+s+s2}{\PYZdq{}}\PY{p}{,} \PY{n}{ylabel}\PY{o}{=}\PY{l+s+s2}{\PYZdq{}}\PY{l+s+s2}{Variación interanual(}\PY{l+s+s2}{\PYZpc{}}\PY{l+s+s2}{)}\PY{l+s+s2}{\PYZdq{}}\PY{p}{,} 
                                    \PY{n}{title}\PY{o}{=}\PY{l+s+s2}{\PYZdq{}}\PY{l+s+s2}{Variación Anual Porcentual del IPC}\PY{l+s+s2}{\PYZdq{}}\PY{p}{,} \PY{n}{style}\PY{o}{=}\PY{l+s+s2}{\PYZdq{}}\PY{l+s+s2}{whitegrid}\PY{l+s+s2}{\PYZdq{}}\PY{p}{,} \PY{n}{palette}\PY{o}{=}\PY{l+s+s2}{\PYZdq{}}\PY{l+s+s2}{Set2}\PY{l+s+s2}{\PYZdq{}}\PY{p}{,} 
                                    \PY{n}{figsize}\PY{o}{=}\PY{p}{(}\PY{l+m+mi}{12}\PY{p}{,} \PY{l+m+mi}{6}\PY{p}{)}\PY{p}{,} \PY{n}{xticks\PYZus{}rotation}\PY{o}{=}\PY{l+m+mi}{15}\PY{p}{)}
\end{Verbatim}
\end{tcolorbox}

    \begin{center}
    \adjustimage{max size={0.9\linewidth}{0.9\paperheight}}{tfm_project_files/tfm_project_28_0.png}
    \end{center}
    { \hspace*{\fill} \\}
    
    Se observa de manera clara que si bien en magnitud la variación puede
diferir, el patrón es el mismo en todas las comunidades autónomas.

    \subsubsection{Gasto Hogares}\label{gasto-hogares}

    La \emph{Encuesta de Presupuestos Familiares} del INE nos permite
obtener información sobre el gasto medio por hogar y por familiar. Esto
puede ser de gran utilidad, pues a diferencia del IPC por comunidad
autónoma, que tiene un valor fijado en su respectivo territorio en un
periodo dado, el gasto por hogar nos permite hacernos una idea de la
variación de costes de vida entre las diferentes regiones del territorio
Nacional.

    \begin{tcolorbox}[breakable, size=fbox, boxrule=1pt, pad at break*=1mm,colback=cellbackground, colframe=cellborder]
\prompt{In}{incolor}{9}{\boxspacing}
\begin{Verbatim}[commandchars=\\\{\}]
\PY{n}{gasto\PYZus{}hogares} \PY{o}{=} \PY{n}{pd}\PY{o}{.}\PY{n}{read\PYZus{}csv}\PY{p}{(}\PY{l+s+s2}{\PYZdq{}}\PY{l+s+s2}{../../processed\PYZus{}data/gasto\PYZus{}ipc\PYZus{}ipri/gasto\PYZus{}hogar.csv}\PY{l+s+s2}{\PYZdq{}}\PY{p}{)}
\PY{n}{gasto\PYZus{}persona} \PY{o}{=} \PY{n}{gasto\PYZus{}hogares}\PY{p}{[}\PY{p}{(}\PY{n}{gasto\PYZus{}hogares}\PY{p}{[}\PY{l+s+s1}{\PYZsq{}}\PY{l+s+s1}{grupo\PYZus{}gasto}\PY{l+s+s1}{\PYZsq{}}\PY{p}{]} \PY{o}{==} \PY{l+s+s2}{\PYZdq{}}\PY{l+s+s2}{Índice general}\PY{l+s+s2}{\PYZdq{}}\PY{p}{)} \PY{o}{\PYZam{}} 
                              \PY{p}{(}\PY{n}{gasto\PYZus{}hogares}\PY{p}{[}\PY{l+s+s1}{\PYZsq{}}\PY{l+s+s1}{tipo\PYZus{}gasto}\PY{l+s+s1}{\PYZsq{}}\PY{p}{]} \PY{o}{==} \PY{l+s+s2}{\PYZdq{}}\PY{l+s+s2}{Gasto medio por persona}\PY{l+s+s2}{\PYZdq{}}\PY{p}{)} \PY{o}{\PYZam{}} 
                              \PY{p}{(}\PY{n}{gasto\PYZus{}hogares}\PY{p}{[}\PY{l+s+s1}{\PYZsq{}}\PY{l+s+s1}{tipo\PYZus{}dato}\PY{l+s+s1}{\PYZsq{}}\PY{p}{]} \PY{o}{==} \PY{l+s+s2}{\PYZdq{}}\PY{l+s+s2}{Dato base}\PY{l+s+s2}{\PYZdq{}}\PY{p}{)}\PY{p}{]}
\PY{n}{gasto\PYZus{}persona\PYZus{}nacional} \PY{o}{=} \PY{n}{gasto\PYZus{}hogares}\PY{p}{[}\PY{p}{(}\PY{n}{gasto\PYZus{}hogares}\PY{p}{[}\PY{l+s+s1}{\PYZsq{}}\PY{l+s+s1}{ccaa}\PY{l+s+s1}{\PYZsq{}}\PY{p}{]} \PY{o}{==} \PY{l+s+s2}{\PYZdq{}}\PY{l+s+s2}{Total Nacional}\PY{l+s+s2}{\PYZdq{}}\PY{p}{)} \PY{o}{\PYZam{}} 
                                       \PY{p}{(}\PY{n}{gasto\PYZus{}hogares}\PY{p}{[}\PY{l+s+s1}{\PYZsq{}}\PY{l+s+s1}{tipo\PYZus{}dato}\PY{l+s+s1}{\PYZsq{}}\PY{p}{]} \PY{o}{==} \PY{l+s+s2}{\PYZdq{}}\PY{l+s+s2}{Dato base}\PY{l+s+s2}{\PYZdq{}}\PY{p}{)} \PY{o}{\PYZam{}} 
                                       \PY{p}{(}\PY{n}{gasto\PYZus{}hogares}\PY{p}{[}\PY{l+s+s1}{\PYZsq{}}\PY{l+s+s1}{grupo\PYZus{}gasto}\PY{l+s+s1}{\PYZsq{}}\PY{p}{]} \PY{o}{==} \PY{l+s+s2}{\PYZdq{}}\PY{l+s+s2}{Índice general}\PY{l+s+s2}{\PYZdq{}}\PY{p}{)} \PY{o}{\PYZam{}} 
                                       \PY{p}{(}\PY{n}{gasto\PYZus{}hogares}\PY{p}{[}\PY{l+s+s1}{\PYZsq{}}\PY{l+s+s1}{tipo\PYZus{}gasto}\PY{l+s+s1}{\PYZsq{}}\PY{p}{]} \PY{o}{==} \PY{l+s+s2}{\PYZdq{}}\PY{l+s+s2}{Gasto medio por persona}\PY{l+s+s2}{\PYZdq{}}\PY{p}{)}\PY{p}{]}\PY{p}{[}\PY{p}{[}\PY{l+s+s1}{\PYZsq{}}\PY{l+s+s1}{periodo}\PY{l+s+s1}{\PYZsq{}}\PY{p}{,} \PY{l+s+s1}{\PYZsq{}}\PY{l+s+s1}{Total}\PY{l+s+s1}{\PYZsq{}}\PY{p}{]}\PY{p}{]}
\PY{n}{p}\PY{o}{.}\PY{n}{create\PYZus{}basic\PYZus{}plot}\PY{p}{(}\PY{n}{gasto\PYZus{}persona\PYZus{}nacional}\PY{p}{[}\PY{l+s+s1}{\PYZsq{}}\PY{l+s+s1}{periodo}\PY{l+s+s1}{\PYZsq{}}\PY{p}{]}\PY{p}{,} \PY{n}{gasto\PYZus{}persona\PYZus{}nacional}\PY{p}{[}\PY{l+s+s1}{\PYZsq{}}\PY{l+s+s1}{Total}\PY{l+s+s1}{\PYZsq{}}\PY{p}{]}\PY{p}{,} \PY{n}{xlabel}\PY{o}{=}\PY{l+s+s2}{\PYZdq{}}\PY{l+s+s2}{Año}\PY{l+s+s2}{\PYZdq{}}\PY{p}{,} \PY{n}{ylabel}\PY{o}{=}\PY{l+s+s2}{\PYZdq{}}\PY{l+s+s2}{Gasto (€)}\PY{l+s+s2}{\PYZdq{}}\PY{p}{,} 
                  \PY{n}{title}\PY{o}{=}\PY{l+s+s2}{\PYZdq{}}\PY{l+s+s2}{Gasto medio por persona (Nacional)}\PY{l+s+s2}{\PYZdq{}}\PY{p}{,} \PY{n}{xticks\PYZus{}rotation}\PY{o}{=}\PY{l+m+mi}{0}\PY{p}{,} \PY{n}{style}\PY{o}{=}\PY{l+s+s2}{\PYZdq{}}\PY{l+s+s2}{whitegrid}\PY{l+s+s2}{\PYZdq{}}\PY{p}{,} \PY{n}{color}\PY{o}{=}\PY{l+s+s2}{\PYZdq{}}\PY{l+s+s2}{dodgerblue}\PY{l+s+s2}{\PYZdq{}}\PY{p}{,} 
                  \PY{n}{figsize}\PY{o}{=}\PY{p}{(}\PY{l+m+mi}{10}\PY{p}{,} \PY{l+m+mi}{6}\PY{p}{)}\PY{p}{,} \PY{n}{label}\PY{o}{=}\PY{l+s+s2}{\PYZdq{}}\PY{l+s+s2}{Gasto por Persona}\PY{l+s+s2}{\PYZdq{}}\PY{p}{)}
\end{Verbatim}
\end{tcolorbox}

    \begin{center}
    \adjustimage{max size={0.9\linewidth}{0.9\paperheight}}{tfm_project_files/tfm_project_32_0.png}
    \end{center}
    { \hspace*{\fill} \\}
    
    El gasto total varía bastante a lo largo de los años, sufriendo una gran
caída en 2020 por la pandemia y posteriorment un gran rebote, ya que se
produjo un gran incremento del consumo. Es importante tener en cuenta
que el gasto agregado, al tener incluidos gastos no necesarios para la
subsistencia del individuo, no es tan indicativo del coste de vida en el
país o en una determinada region.

Para tener un mejor entendimiento a este respecto, podemos focalizarnos
en gastos específicos como alimentos, bebidas no alcohólicas, vivienda,
gas, electricidad, agua y combustibles, que englobaremos bajo el manto
de ``gastos básicos''.

    \begin{tcolorbox}[breakable, size=fbox, boxrule=1pt, pad at break*=1mm,colback=cellbackground, colframe=cellborder]
\prompt{In}{incolor}{10}{\boxspacing}
\begin{Verbatim}[commandchars=\\\{\}]
\PY{n}{gastos\PYZus{}basicos} \PY{o}{=} \PY{p}{[}\PY{l+s+s1}{\PYZsq{}}\PY{l+s+s1}{01 Alimentos y bebidas no alcohólicas}\PY{l+s+s1}{\PYZsq{}}\PY{p}{,}
                   \PY{l+s+s1}{\PYZsq{}}\PY{l+s+s1}{04 Vivienda, agua, electricidad, gas y otros combustibles}\PY{l+s+s1}{\PYZsq{}}\PY{p}{]}

\PY{n}{gasto\PYZus{}basico} \PY{o}{=} \PY{n}{gasto\PYZus{}hogares}\PY{p}{[}\PY{p}{(}\PY{n}{gasto\PYZus{}hogares}\PY{p}{[}\PY{l+s+s1}{\PYZsq{}}\PY{l+s+s1}{tipo\PYZus{}gasto}\PY{l+s+s1}{\PYZsq{}}\PY{p}{]} \PY{o}{==} \PY{l+s+s2}{\PYZdq{}}\PY{l+s+s2}{Gasto medio por persona}\PY{l+s+s2}{\PYZdq{}}\PY{p}{)} \PY{o}{\PYZam{}} 
                             \PY{p}{(}\PY{n}{gasto\PYZus{}hogares}\PY{p}{[}\PY{l+s+s1}{\PYZsq{}}\PY{l+s+s1}{tipo\PYZus{}dato}\PY{l+s+s1}{\PYZsq{}}\PY{p}{]} \PY{o}{==} \PY{l+s+s2}{\PYZdq{}}\PY{l+s+s2}{Dato base}\PY{l+s+s2}{\PYZdq{}}\PY{p}{)} \PY{o}{\PYZam{}} 
                             \PY{n}{gasto\PYZus{}hogares}\PY{o}{.}\PY{n}{grupo\PYZus{}gasto}\PY{o}{.}\PY{n}{isin}\PY{p}{(}\PY{n}{gastos\PYZus{}basicos}\PY{p}{)}\PY{p}{]}\PY{o}{.}\PY{n}{groupby}\PY{p}{(}\PY{p}{[}\PY{l+s+s2}{\PYZdq{}}\PY{l+s+s2}{ccaa}\PY{l+s+s2}{\PYZdq{}}\PY{p}{,} \PY{l+s+s2}{\PYZdq{}}\PY{l+s+s2}{periodo}\PY{l+s+s2}{\PYZdq{}}\PY{p}{]}\PY{p}{,} \PY{n}{as\PYZus{}index}\PY{o}{=}\PY{k+kc}{False}\PY{p}{)}\PY{o}{.}\PY{n}{sum}\PY{p}{(}\PY{n}{numeric\PYZus{}only}\PY{o}{=}\PY{k+kc}{True}\PY{p}{)}
\PY{n}{gasto\PYZus{}basico\PYZus{}nacional} \PY{o}{=} \PY{n}{gasto\PYZus{}basico}\PY{p}{[}\PY{n}{gasto\PYZus{}basico}\PY{p}{[}\PY{l+s+s1}{\PYZsq{}}\PY{l+s+s1}{ccaa}\PY{l+s+s1}{\PYZsq{}}\PY{p}{]} \PY{o}{==} \PY{l+s+s2}{\PYZdq{}}\PY{l+s+s2}{Total Nacional}\PY{l+s+s2}{\PYZdq{}}\PY{p}{]}
\PY{n}{p}\PY{o}{.}\PY{n}{create\PYZus{}basic\PYZus{}plot}\PY{p}{(}\PY{n}{gasto\PYZus{}basico\PYZus{}nacional}\PY{p}{[}\PY{l+s+s1}{\PYZsq{}}\PY{l+s+s1}{periodo}\PY{l+s+s1}{\PYZsq{}}\PY{p}{]}\PY{p}{,} \PY{n}{gasto\PYZus{}basico\PYZus{}nacional}\PY{p}{[}\PY{l+s+s1}{\PYZsq{}}\PY{l+s+s1}{Total}\PY{l+s+s1}{\PYZsq{}}\PY{p}{]}\PY{p}{,} \PY{n}{xlabel}\PY{o}{=}\PY{l+s+s2}{\PYZdq{}}\PY{l+s+s2}{Año}\PY{l+s+s2}{\PYZdq{}}\PY{p}{,} 
                  \PY{n}{ylabel}\PY{o}{=}\PY{l+s+s2}{\PYZdq{}}\PY{l+s+s2}{Gasto (€)}\PY{l+s+s2}{\PYZdq{}}\PY{p}{,} \PY{n}{title}\PY{o}{=}\PY{l+s+s2}{\PYZdq{}}\PY{l+s+s2}{Gasto básicos medios por persona (Nacional)}\PY{l+s+s2}{\PYZdq{}}\PY{p}{,} 
                  \PY{n}{xticks\PYZus{}rotation}\PY{o}{=}\PY{l+m+mi}{0}\PY{p}{,} \PY{n}{style}\PY{o}{=}\PY{l+s+s2}{\PYZdq{}}\PY{l+s+s2}{whitegrid}\PY{l+s+s2}{\PYZdq{}}\PY{p}{,} \PY{n}{color}\PY{o}{=}\PY{l+s+s2}{\PYZdq{}}\PY{l+s+s2}{dodgerblue}\PY{l+s+s2}{\PYZdq{}}\PY{p}{,} \PY{n}{figsize}\PY{o}{=}\PY{p}{(}\PY{l+m+mi}{10}\PY{p}{,} \PY{l+m+mi}{6}\PY{p}{)}\PY{p}{,} 
                  \PY{n}{label}\PY{o}{=}\PY{l+s+s2}{\PYZdq{}}\PY{l+s+s2}{Gasto por Persona}\PY{l+s+s2}{\PYZdq{}}\PY{p}{)}
\end{Verbatim}
\end{tcolorbox}

    \begin{center}
    \adjustimage{max size={0.9\linewidth}{0.9\paperheight}}{tfm_project_files/tfm_project_34_0.png}
    \end{center}
    { \hspace*{\fill} \\}
    
    \begin{tcolorbox}[breakable, size=fbox, boxrule=1pt, pad at break*=1mm,colback=cellbackground, colframe=cellborder]
\prompt{In}{incolor}{11}{\boxspacing}
\begin{Verbatim}[commandchars=\\\{\}]
\PY{n}{gasto\PYZus{}basico\PYZus{}nacional\PYZus{}smi} \PY{o}{=} \PY{n}{gasto\PYZus{}basico\PYZus{}nacional}\PY{o}{.}\PY{n}{rename}\PY{p}{(}\PY{n}{columns}\PY{o}{=}\PY{p}{\PYZob{}}\PY{l+s+s1}{\PYZsq{}}\PY{l+s+s1}{Total}\PY{l+s+s1}{\PYZsq{}}\PY{p}{:} \PY{l+s+s1}{\PYZsq{}}\PY{l+s+s1}{gasto\PYZus{}basico}\PY{l+s+s1}{\PYZsq{}}\PY{p}{\PYZcb{}}\PY{p}{)}\PY{o}{.}\PY{n}{merge}\PY{p}{(}\PY{n}{smi}\PY{p}{,} \PY{n}{how}\PY{o}{=}\PY{l+s+s1}{\PYZsq{}}\PY{l+s+s1}{right}\PY{l+s+s1}{\PYZsq{}}\PY{p}{,} \PY{n}{on}\PY{o}{=}\PY{l+s+s2}{\PYZdq{}}\PY{l+s+s2}{periodo}\PY{l+s+s2}{\PYZdq{}}\PY{p}{)}
\PY{n}{p}\PY{o}{.}\PY{n}{create\PYZus{}dual\PYZus{}plot}\PY{p}{(}\PY{n}{gasto\PYZus{}basico\PYZus{}nacional\PYZus{}smi}\PY{p}{[}\PY{l+s+s1}{\PYZsq{}}\PY{l+s+s1}{periodo}\PY{l+s+s1}{\PYZsq{}}\PY{p}{]}\PY{p}{,} \PY{n}{gasto\PYZus{}basico\PYZus{}nacional\PYZus{}smi}\PY{p}{[}\PY{l+s+s1}{\PYZsq{}}\PY{l+s+s1}{gasto\PYZus{}basico}\PY{l+s+s1}{\PYZsq{}}\PY{p}{]}\PY{p}{,} 
                 \PY{n}{gasto\PYZus{}basico\PYZus{}nacional\PYZus{}smi}\PY{p}{[}\PY{l+s+s1}{\PYZsq{}}\PY{l+s+s1}{smi\PYZus{}14}\PY{l+s+s1}{\PYZsq{}}\PY{p}{]}\PY{p}{,} \PY{n}{xlabel}\PY{o}{=}\PY{l+s+s2}{\PYZdq{}}\PY{l+s+s2}{Año}\PY{l+s+s2}{\PYZdq{}}\PY{p}{,} \PY{n}{ylabel1}\PY{o}{=}\PY{l+s+s2}{\PYZdq{}}\PY{l+s+s2}{Gasto basico(€)}\PY{l+s+s2}{\PYZdq{}}\PY{p}{,} 
                 \PY{n}{ylabel2}\PY{o}{=}\PY{l+s+s2}{\PYZdq{}}\PY{l+s+s2}{SMI (€)}\PY{l+s+s2}{\PYZdq{}} \PY{p}{,}\PY{n}{label1}\PY{o}{=} \PY{l+s+s2}{\PYZdq{}}\PY{l+s+s2}{Gastos básicos}\PY{l+s+s2}{\PYZdq{}}\PY{p}{,} \PY{n}{label2}\PY{o}{=}\PY{l+s+s2}{\PYZdq{}}\PY{l+s+s2}{SMI}\PY{l+s+s2}{\PYZdq{}}\PY{p}{,} \PY{n}{title}\PY{o}{=}\PY{l+s+s2}{\PYZdq{}}\PY{l+s+s2}{Gastos Básicos y SMI}\PY{l+s+s2}{\PYZdq{}}\PY{p}{,} 
                 \PY{n}{xticks\PYZus{}rotation}\PY{o}{=}\PY{l+m+mi}{0}\PY{p}{,} \PY{n}{style}\PY{o}{=}\PY{l+s+s2}{\PYZdq{}}\PY{l+s+s2}{whitegrid}\PY{l+s+s2}{\PYZdq{}}\PY{p}{,} \PY{n}{figsize}\PY{o}{=}\PY{p}{(}\PY{l+m+mi}{10}\PY{p}{,} \PY{l+m+mi}{6}\PY{p}{)}\PY{p}{,} \PY{n}{marker1}\PY{o}{=}\PY{k+kc}{None}\PY{p}{,} \PY{n}{marker2}\PY{o}{=}\PY{k+kc}{None}\PY{p}{,} 
                 \PY{n}{secondary\PYZus{}y}\PY{o}{=}\PY{k+kc}{True}\PY{p}{)}
\end{Verbatim}
\end{tcolorbox}

    \begin{center}
    \adjustimage{max size={0.9\linewidth}{0.9\paperheight}}{tfm_project_files/tfm_project_35_0.png}
    \end{center}
    { \hspace*{\fill} \\}
    
    Vemos que ahora el gasto tiene una evolución algo menos volátil y de
hecho sigue una evolución bastante más parecida al IPC general y también
tiene una clara correlación con el SMI, aunque dada la baja proporción
de trabajadores que percibe el salario mínimo, no necesariamente
hablamos de una relación causal. Veamos ahora las diferencias entre
comunidades autónomas.

    \begin{tcolorbox}[breakable, size=fbox, boxrule=1pt, pad at break*=1mm,colback=cellbackground, colframe=cellborder]
\prompt{In}{incolor}{12}{\boxspacing}
\begin{Verbatim}[commandchars=\\\{\}]
\PY{n}{gasto\PYZus{}basico\PYZus{}ccaa} \PY{o}{=} \PY{n}{gasto\PYZus{}basico}\PY{p}{[}\PY{p}{(}\PY{n}{gasto\PYZus{}basico}\PY{p}{[}\PY{l+s+s1}{\PYZsq{}}\PY{l+s+s1}{ccaa}\PY{l+s+s1}{\PYZsq{}}\PY{p}{]}\PY{o}{.}\PY{n}{isin}\PY{p}{(}\PY{n}{CCAA\PYZus{}mostrar}\PY{p}{)}\PY{p}{)}\PY{p}{]}\PY{o}{.}\PY{n}{groupby}\PY{p}{(}\PY{p}{[}\PY{l+s+s1}{\PYZsq{}}\PY{l+s+s1}{periodo}\PY{l+s+s1}{\PYZsq{}}\PY{p}{,}\PY{l+s+s1}{\PYZsq{}}\PY{l+s+s1}{ccaa}\PY{l+s+s1}{\PYZsq{}}\PY{p}{]}\PY{p}{,} \PY{n}{as\PYZus{}index}\PY{o}{=}\PY{k+kc}{False}\PY{p}{)}\PY{o}{.}\PY{n}{sum}\PY{p}{(}\PY{n}{numeric\PYZus{}only}\PY{o}{=}\PY{k+kc}{True}\PY{p}{)}
\PY{n}{p}\PY{o}{.}\PY{n}{create\PYZus{}multi\PYZus{}category\PYZus{}plot}\PY{p}{(}\PY{n}{data} \PY{o}{=} \PY{n}{gasto\PYZus{}basico\PYZus{}ccaa}\PY{p}{,} \PY{n}{x\PYZus{}col}\PY{o}{=}\PY{l+s+s2}{\PYZdq{}}\PY{l+s+s2}{periodo}\PY{l+s+s2}{\PYZdq{}}\PY{p}{,} \PY{n}{y\PYZus{}col}\PY{o}{=}\PY{l+s+s2}{\PYZdq{}}\PY{l+s+s2}{Total}\PY{l+s+s2}{\PYZdq{}}\PY{p}{,} \PY{n}{label} \PY{o}{=} \PY{l+s+s2}{\PYZdq{}}\PY{l+s+s2}{CCAA}\PY{l+s+s2}{\PYZdq{}}\PY{p}{,}\PY{n}{category\PYZus{}col}\PY{o}{=}\PY{l+s+s2}{\PYZdq{}}\PY{l+s+s2}{ccaa}\PY{l+s+s2}{\PYZdq{}}\PY{p}{,} 
                           \PY{n}{xlabel}\PY{o}{=}\PY{l+s+s2}{\PYZdq{}}\PY{l+s+s2}{Año}\PY{l+s+s2}{\PYZdq{}}\PY{p}{,} \PY{n}{ylabel}\PY{o}{=}\PY{l+s+s2}{\PYZdq{}}\PY{l+s+s2}{Gasto(€)}\PY{l+s+s2}{\PYZdq{}}\PY{p}{,} \PY{n}{title}\PY{o}{=}\PY{l+s+s2}{\PYZdq{}}\PY{l+s+s2}{Gastos básicos medios por persona (CCAA)}\PY{l+s+s2}{\PYZdq{}}\PY{p}{,} 
                           \PY{n}{xticks\PYZus{}rotation}\PY{o}{=}\PY{l+m+mi}{0}\PY{p}{,} \PY{n}{style}\PY{o}{=}\PY{l+s+s2}{\PYZdq{}}\PY{l+s+s2}{whitegrid}\PY{l+s+s2}{\PYZdq{}}\PY{p}{,} \PY{n}{figsize}\PY{o}{=}\PY{p}{(}\PY{l+m+mi}{12}\PY{p}{,} \PY{l+m+mi}{7}\PY{p}{)}\PY{p}{)}
\end{Verbatim}
\end{tcolorbox}

    \begin{center}
    \adjustimage{max size={0.9\linewidth}{0.9\paperheight}}{tfm_project_files/tfm_project_37_0.png}
    \end{center}
    { \hspace*{\fill} \\}
    
    La tendencia pese a ser general presenta variaciones mensuales
relevantes, así como diferencias notables entre las comunidades
autónomas y el gasto por persona que tiene cada una, encontrándonos
diferencias superiores al 50\%. Estos valores los usaremos más adelante
en este análisis para crear ratios útiles que nos permitan entender el
poder real de compra que que tiene el salario mínimo en las diferentes
regiones.

    \subsubsection{Salarios}\label{salarios}

    Los salarios son necesarios para entender la coyuntura de cada
territorio, por lo que es importante estudiar cómo han variado con el
tiempo y así poder identificar posibles grupos afectados por el salario
mínimo.

    \begin{tcolorbox}[breakable, size=fbox, boxrule=1pt, pad at break*=1mm,colback=cellbackground, colframe=cellborder]
\prompt{In}{incolor}{13}{\boxspacing}
\begin{Verbatim}[commandchars=\\\{\}]
\PY{n}{salarios\PYZus{}ocupacion} \PY{o}{=} \PY{n}{pd}\PY{o}{.}\PY{n}{read\PYZus{}csv}\PY{p}{(}\PY{l+s+s2}{\PYZdq{}}\PY{l+s+s2}{../../processed\PYZus{}data/salarios/ocupacion.csv}\PY{l+s+s2}{\PYZdq{}}\PY{p}{)}
\PY{n}{salarios\PYZus{}ocupacion\PYZus{}nacional} \PY{o}{=} \PY{n}{salarios\PYZus{}ocupacion}\PY{p}{[}\PY{p}{(}\PY{n}{salarios\PYZus{}ocupacion}\PY{p}{[}\PY{l+s+s1}{\PYZsq{}}\PY{l+s+s1}{ccaa}\PY{l+s+s1}{\PYZsq{}}\PY{p}{]} \PY{o}{==} \PY{l+s+s2}{\PYZdq{}}\PY{l+s+s2}{Total Nacional}\PY{l+s+s2}{\PYZdq{}}\PY{p}{)} \PY{o}{\PYZam{}} 
                                                 \PY{p}{(}\PY{n}{salarios\PYZus{}ocupacion}\PY{p}{[}\PY{l+s+s1}{\PYZsq{}}\PY{l+s+s1}{sexo}\PY{l+s+s1}{\PYZsq{}}\PY{p}{]} \PY{o}{==} \PY{l+s+s2}{\PYZdq{}}\PY{l+s+s2}{Ambos sexos}\PY{l+s+s2}{\PYZdq{}}\PY{p}{)}\PY{p}{]}
\PY{n}{p}\PY{o}{.}\PY{n}{create\PYZus{}multi\PYZus{}category\PYZus{}plot}\PY{p}{(}\PY{n}{salarios\PYZus{}ocupacion\PYZus{}nacional}\PY{p}{,} \PY{l+s+s2}{\PYZdq{}}\PY{l+s+s2}{periodo}\PY{l+s+s2}{\PYZdq{}}\PY{p}{,} \PY{l+s+s2}{\PYZdq{}}\PY{l+s+s2}{salario\PYZus{}año}\PY{l+s+s2}{\PYZdq{}}\PY{p}{,} \PY{l+s+s2}{\PYZdq{}}\PY{l+s+s2}{ocupacion}\PY{l+s+s2}{\PYZdq{}}\PY{p}{,}
                            \PY{n}{xlabel}\PY{o}{=}\PY{l+s+s2}{\PYZdq{}}\PY{l+s+s2}{Año}\PY{l+s+s2}{\PYZdq{}}\PY{p}{,} \PY{n}{ylabel}\PY{o}{=}\PY{l+s+s2}{\PYZdq{}}\PY{l+s+s2}{Salario (€)}\PY{l+s+s2}{\PYZdq{}}\PY{p}{,}\PY{n}{label}\PY{o}{=}\PY{l+s+s2}{\PYZdq{}}\PY{l+s+s2}{Nivel de Ocupación}\PY{l+s+s2}{\PYZdq{}} \PY{p}{,}
                            \PY{n}{xticks\PYZus{}rotation}\PY{o}{=}\PY{l+m+mi}{0}\PY{p}{,} \PY{n}{style}\PY{o}{=}\PY{l+s+s2}{\PYZdq{}}\PY{l+s+s2}{whitegrid}\PY{l+s+s2}{\PYZdq{}}\PY{p}{,} \PY{n}{figsize}\PY{o}{=}\PY{p}{(}\PY{l+m+mi}{12}\PY{p}{,} \PY{l+m+mi}{7}\PY{p}{)}\PY{p}{,}
                            \PY{n}{title}\PY{o}{=} \PY{l+s+s2}{\PYZdq{}}\PY{l+s+s2}{Salario Anual por Ocupación}\PY{l+s+s2}{\PYZdq{}}\PY{p}{)}
\end{Verbatim}
\end{tcolorbox}

    \begin{center}
    \adjustimage{max size={0.9\linewidth}{0.9\paperheight}}{tfm_project_files/tfm_project_41_0.png}
    \end{center}
    { \hspace*{\fill} \\}
    
    \begin{tcolorbox}[breakable, size=fbox, boxrule=1pt, pad at break*=1mm,colback=cellbackground, colframe=cellborder]
\prompt{In}{incolor}{14}{\boxspacing}
\begin{Verbatim}[commandchars=\\\{\}]
\PY{n}{p}\PY{o}{.}\PY{n}{create\PYZus{}multi\PYZus{}category\PYZus{}plot}\PY{p}{(}\PY{n}{salarios\PYZus{}ocupacion\PYZus{}nacional}\PY{p}{,} \PY{l+s+s2}{\PYZdq{}}\PY{l+s+s2}{periodo}\PY{l+s+s2}{\PYZdq{}}\PY{p}{,} \PY{l+s+s2}{\PYZdq{}}\PY{l+s+s2}{salario\PYZus{}hora}\PY{l+s+s2}{\PYZdq{}}\PY{p}{,} 
                           \PY{l+s+s2}{\PYZdq{}}\PY{l+s+s2}{ocupacion}\PY{l+s+s2}{\PYZdq{}}\PY{p}{,} \PY{n}{xlabel}\PY{o}{=}\PY{l+s+s2}{\PYZdq{}}\PY{l+s+s2}{Año}\PY{l+s+s2}{\PYZdq{}}\PY{p}{,} \PY{n}{ylabel}\PY{o}{=}\PY{l+s+s2}{\PYZdq{}}\PY{l+s+s2}{Salario/Hora (€)}\PY{l+s+s2}{\PYZdq{}}\PY{p}{,}\PY{n}{label}\PY{o}{=}\PY{l+s+s2}{\PYZdq{}}\PY{l+s+s2}{Nivel de Ocupación}\PY{l+s+s2}{\PYZdq{}} \PY{p}{,}
                           \PY{n}{xticks\PYZus{}rotation}\PY{o}{=}\PY{l+m+mi}{0}\PY{p}{,} \PY{n}{style}\PY{o}{=}\PY{l+s+s2}{\PYZdq{}}\PY{l+s+s2}{whitegrid}\PY{l+s+s2}{\PYZdq{}}\PY{p}{,} \PY{n}{figsize}\PY{o}{=}\PY{p}{(}\PY{l+m+mi}{12}\PY{p}{,} \PY{l+m+mi}{7}\PY{p}{)}\PY{p}{,} \PY{n}{title}\PY{o}{=} \PY{l+s+s2}{\PYZdq{}}\PY{l+s+s2}{Salario por Hora por Ocupación}\PY{l+s+s2}{\PYZdq{}}\PY{p}{)}
\end{Verbatim}
\end{tcolorbox}

    \begin{center}
    \adjustimage{max size={0.9\linewidth}{0.9\paperheight}}{tfm_project_files/tfm_project_42_0.png}
    \end{center}
    { \hspace*{\fill} \\}
    
    Observamos que no hay una diferencia particular en términos de
crecimiento salarial, lo que a priori puede indicar que no hay ningún
grupo especialmente afectado por el salario mínimo. Veamos como
evoluciona el salario de la ocupación más baja respecto al SMI.

    \begin{tcolorbox}[breakable, size=fbox, boxrule=1pt, pad at break*=1mm,colback=cellbackground, colframe=cellborder]
\prompt{In}{incolor}{15}{\boxspacing}
\begin{Verbatim}[commandchars=\\\{\}]
\PY{n}{salario\PYZus{}oc\PYZus{}nacional\PYZus{}baja\PYZus{}smi} \PY{o}{=} \PY{n}{salarios\PYZus{}ocupacion\PYZus{}nacional}\PY{p}{[}\PY{n}{salarios\PYZus{}ocupacion\PYZus{}nacional}\PY{p}{[}\PY{l+s+s1}{\PYZsq{}}\PY{l+s+s1}{ocupacion}\PY{l+s+s1}{\PYZsq{}}\PY{p}{]} \PY{o}{==} \PY{l+s+s2}{\PYZdq{}}\PY{l+s+s2}{Baja}\PY{l+s+s2}{\PYZdq{}}\PY{p}{]}\PY{o}{.}\PY{n}{merge}\PY{p}{(}\PY{n}{smi}\PY{p}{,} \PY{n}{how}\PY{o}{=}\PY{l+s+s2}{\PYZdq{}}\PY{l+s+s2}{left}\PY{l+s+s2}{\PYZdq{}}\PY{p}{,} \PY{n}{on}\PY{o}{=}\PY{l+s+s2}{\PYZdq{}}\PY{l+s+s2}{periodo}\PY{l+s+s2}{\PYZdq{}}\PY{p}{)}
\PY{n}{p}\PY{o}{.}\PY{n}{create\PYZus{}dual\PYZus{}plot}\PY{p}{(}\PY{n}{salario\PYZus{}oc\PYZus{}nacional\PYZus{}baja\PYZus{}smi}\PY{p}{[}\PY{l+s+s1}{\PYZsq{}}\PY{l+s+s1}{periodo}\PY{l+s+s1}{\PYZsq{}}\PY{p}{]}\PY{p}{,} \PY{n}{salario\PYZus{}oc\PYZus{}nacional\PYZus{}baja\PYZus{}smi}\PY{p}{[}\PY{l+s+s1}{\PYZsq{}}\PY{l+s+s1}{salario\PYZus{}año}\PY{l+s+s1}{\PYZsq{}}\PY{p}{]}\PY{p}{,} \PY{n}{salario\PYZus{}oc\PYZus{}nacional\PYZus{}baja\PYZus{}smi}\PY{p}{[}\PY{l+s+s1}{\PYZsq{}}\PY{l+s+s1}{smi\PYZus{}14}\PY{l+s+s1}{\PYZsq{}}\PY{p}{]}\PY{p}{,} 
                 \PY{n}{xlabel}\PY{o}{=}\PY{l+s+s2}{\PYZdq{}}\PY{l+s+s2}{Año}\PY{l+s+s2}{\PYZdq{}}\PY{p}{,} \PY{n}{ylabel1}\PY{o}{=}\PY{l+s+s2}{\PYZdq{}}\PY{l+s+s2}{Salario Anual (€)}\PY{l+s+s2}{\PYZdq{}}\PY{p}{,} \PY{n}{ylabel2}\PY{o}{=}\PY{l+s+s2}{\PYZdq{}}\PY{l+s+s2}{SMI (€)}\PY{l+s+s2}{\PYZdq{}} \PY{p}{,}\PY{n}{label1}\PY{o}{=} \PY{l+s+s2}{\PYZdq{}}\PY{l+s+s2}{Salario Anual}\PY{l+s+s2}{\PYZdq{}}\PY{p}{,} \PY{n}{label2}\PY{o}{=}\PY{l+s+s2}{\PYZdq{}}\PY{l+s+s2}{SMI}\PY{l+s+s2}{\PYZdq{}}\PY{p}{,} \PY{n}{title}\PY{o}{=}\PY{l+s+s2}{\PYZdq{}}\PY{l+s+s2}{Salario de ocupación baja y SMI}\PY{l+s+s2}{\PYZdq{}}\PY{p}{,}
                 \PY{n}{xticks\PYZus{}rotation}\PY{o}{=}\PY{l+m+mi}{0}\PY{p}{,} \PY{n}{style}\PY{o}{=}\PY{l+s+s2}{\PYZdq{}}\PY{l+s+s2}{whitegrid}\PY{l+s+s2}{\PYZdq{}}\PY{p}{,} \PY{n}{figsize}\PY{o}{=}\PY{p}{(}\PY{l+m+mi}{10}\PY{p}{,} \PY{l+m+mi}{6}\PY{p}{)}\PY{p}{,} \PY{n}{marker1}\PY{o}{=}\PY{k+kc}{None}\PY{p}{,} \PY{n}{marker2}\PY{o}{=}\PY{k+kc}{None}\PY{p}{,} \PY{n}{secondary\PYZus{}y}\PY{o}{=}\PY{k+kc}{True}\PY{p}{)}
\end{Verbatim}
\end{tcolorbox}

    \begin{center}
    \adjustimage{max size={0.9\linewidth}{0.9\paperheight}}{tfm_project_files/tfm_project_44_0.png}
    \end{center}
    { \hspace*{\fill} \\}
    
    \begin{tcolorbox}[breakable, size=fbox, boxrule=1pt, pad at break*=1mm,colback=cellbackground, colframe=cellborder]
\prompt{In}{incolor}{16}{\boxspacing}
\begin{Verbatim}[commandchars=\\\{\}]
\PY{n}{salario\PYZus{}oc\PYZus{}nacional\PYZus{}alta\PYZus{}smi} \PY{o}{=} \PY{n}{salarios\PYZus{}ocupacion\PYZus{}nacional}\PY{p}{[}\PY{n}{salarios\PYZus{}ocupacion\PYZus{}nacional}\PY{p}{[}\PY{l+s+s1}{\PYZsq{}}\PY{l+s+s1}{ocupacion}\PY{l+s+s1}{\PYZsq{}}\PY{p}{]} \PY{o}{==} \PY{l+s+s2}{\PYZdq{}}\PY{l+s+s2}{Alta}\PY{l+s+s2}{\PYZdq{}}\PY{p}{]}\PY{o}{.}\PY{n}{merge}\PY{p}{(}\PY{n}{smi}\PY{p}{,} \PY{n}{how}\PY{o}{=}\PY{l+s+s2}{\PYZdq{}}\PY{l+s+s2}{left}\PY{l+s+s2}{\PYZdq{}}\PY{p}{,} \PY{n}{on}\PY{o}{=}\PY{l+s+s2}{\PYZdq{}}\PY{l+s+s2}{periodo}\PY{l+s+s2}{\PYZdq{}}\PY{p}{)}
\PY{n}{p}\PY{o}{.}\PY{n}{create\PYZus{}dual\PYZus{}plot}\PY{p}{(}\PY{n}{salario\PYZus{}oc\PYZus{}nacional\PYZus{}alta\PYZus{}smi}\PY{p}{[}\PY{l+s+s1}{\PYZsq{}}\PY{l+s+s1}{periodo}\PY{l+s+s1}{\PYZsq{}}\PY{p}{]}\PY{p}{,} \PY{n}{salario\PYZus{}oc\PYZus{}nacional\PYZus{}alta\PYZus{}smi}\PY{p}{[}\PY{l+s+s1}{\PYZsq{}}\PY{l+s+s1}{salario\PYZus{}año}\PY{l+s+s1}{\PYZsq{}}\PY{p}{]}\PY{p}{,} \PY{n}{salario\PYZus{}oc\PYZus{}nacional\PYZus{}alta\PYZus{}smi}\PY{p}{[}\PY{l+s+s1}{\PYZsq{}}\PY{l+s+s1}{smi\PYZus{}14}\PY{l+s+s1}{\PYZsq{}}\PY{p}{]}\PY{p}{,} 
                 \PY{n}{xlabel}\PY{o}{=}\PY{l+s+s2}{\PYZdq{}}\PY{l+s+s2}{Año}\PY{l+s+s2}{\PYZdq{}}\PY{p}{,} \PY{n}{ylabel1}\PY{o}{=}\PY{l+s+s2}{\PYZdq{}}\PY{l+s+s2}{Salario Anual (€)}\PY{l+s+s2}{\PYZdq{}}\PY{p}{,} \PY{n}{ylabel2}\PY{o}{=}\PY{l+s+s2}{\PYZdq{}}\PY{l+s+s2}{SMI (€)}\PY{l+s+s2}{\PYZdq{}} \PY{p}{,}\PY{n}{label1}\PY{o}{=} \PY{l+s+s2}{\PYZdq{}}\PY{l+s+s2}{Salario Anual}\PY{l+s+s2}{\PYZdq{}}\PY{p}{,} \PY{n}{label2}\PY{o}{=}\PY{l+s+s2}{\PYZdq{}}\PY{l+s+s2}{SMI}\PY{l+s+s2}{\PYZdq{}}\PY{p}{,} \PY{n}{title}\PY{o}{=}\PY{l+s+s2}{\PYZdq{}}\PY{l+s+s2}{Salario de ocupación alta y SMI}\PY{l+s+s2}{\PYZdq{}}\PY{p}{,}
                 \PY{n}{xticks\PYZus{}rotation}\PY{o}{=}\PY{l+m+mi}{0}\PY{p}{,} \PY{n}{style}\PY{o}{=}\PY{l+s+s2}{\PYZdq{}}\PY{l+s+s2}{whitegrid}\PY{l+s+s2}{\PYZdq{}}\PY{p}{,} \PY{n}{figsize}\PY{o}{=}\PY{p}{(}\PY{l+m+mi}{10}\PY{p}{,} \PY{l+m+mi}{6}\PY{p}{)}\PY{p}{,} \PY{n}{marker1}\PY{o}{=}\PY{k+kc}{None}\PY{p}{,} \PY{n}{marker2}\PY{o}{=}\PY{k+kc}{None}\PY{p}{,} \PY{n}{secondary\PYZus{}y}\PY{o}{=}\PY{k+kc}{True} \PY{p}{)}
\end{Verbatim}
\end{tcolorbox}

    \begin{center}
    \adjustimage{max size={0.9\linewidth}{0.9\paperheight}}{tfm_project_files/tfm_project_45_0.png}
    \end{center}
    { \hspace*{\fill} \\}
    
    Aunque la evolución del salario de ocupacion baja parezca correlacionado
con el salario mínimo, también lo parece así el de la ocupación alta, lo
que puede descartar una causalidad. No obstante, hemos de tener en
cuenta que las ocupaciones bajas corresponden a puestos que serán en
general menos productivos y cuya productividad también crecerá en
general más despacio, por lo que es posible que el crecimiento de los
salarios en en sectores de nivel de ocupación alta vaya más relacionado
con la productividad y la de la ocupación baja más relacionado con el
movimiento del salario mínimo.

También es importante tener en cuenta que a nivel político la propuesta
de subida del salario mínimo gira en torno a llegar al 60\% del salario
medio, lo que implica algo de causalidad entre el nivel de salarios y la
subida de salarios mínimos a realizar, si bien es cierto que al depender
esta medida más de cuestiones políticas que de cuestiones económicas es
precipitado dar por supuesta esta relación.

Veamos también cómo evoluciona el salario de las ocupaciones bajas en
las comunidades autónomas que hemos mostrado antes:

    \begin{tcolorbox}[breakable, size=fbox, boxrule=1pt, pad at break*=1mm,colback=cellbackground, colframe=cellborder]
\prompt{In}{incolor}{17}{\boxspacing}
\begin{Verbatim}[commandchars=\\\{\}]
\PY{n}{salarios\PYZus{}ocupacion\PYZus{}ccaa} \PY{o}{=} \PY{n}{salarios\PYZus{}ocupacion}\PY{p}{[}\PY{p}{(}\PY{n}{salarios\PYZus{}ocupacion}\PY{p}{[}\PY{l+s+s1}{\PYZsq{}}\PY{l+s+s1}{ccaa}\PY{l+s+s1}{\PYZsq{}}\PY{p}{]} \PY{o}{!=} \PY{l+s+s2}{\PYZdq{}}\PY{l+s+s2}{Total Nacional}\PY{l+s+s2}{\PYZdq{}}\PY{p}{)} \PY{o}{\PYZam{}}
                                              \PY{p}{(}\PY{n}{salarios\PYZus{}ocupacion}\PY{p}{[}\PY{l+s+s1}{\PYZsq{}}\PY{l+s+s1}{sexo}\PY{l+s+s1}{\PYZsq{}}\PY{p}{]} \PY{o}{==} \PY{l+s+s2}{\PYZdq{}}\PY{l+s+s2}{Ambos sexos}\PY{l+s+s2}{\PYZdq{}}\PY{p}{)}\PY{p}{]}
\PY{n}{salarios\PYZus{}oc\PYZus{}baja\PYZus{}ccaa} \PY{o}{=} \PY{n}{salarios\PYZus{}ocupacion\PYZus{}ccaa}\PY{p}{[}\PY{p}{(}\PY{n}{salarios\PYZus{}ocupacion\PYZus{}ccaa}\PY{p}{[}\PY{l+s+s1}{\PYZsq{}}\PY{l+s+s1}{ocupacion}\PY{l+s+s1}{\PYZsq{}}\PY{p}{]} \PY{o}{==} \PY{l+s+s2}{\PYZdq{}}\PY{l+s+s2}{Baja}\PY{l+s+s2}{\PYZdq{}}\PY{p}{)} \PY{o}{\PYZam{}} 
                                                \PY{p}{(}\PY{n}{salarios\PYZus{}ocupacion\PYZus{}ccaa}\PY{p}{[}\PY{l+s+s1}{\PYZsq{}}\PY{l+s+s1}{ccaa}\PY{l+s+s1}{\PYZsq{}}\PY{p}{]}\PY{o}{.}\PY{n}{isin}\PY{p}{(}\PY{n}{CCAA\PYZus{}mostrar}\PY{p}{)}\PY{p}{)}\PY{p}{]}
\PY{n}{p}\PY{o}{.}\PY{n}{create\PYZus{}multi\PYZus{}category\PYZus{}plot}\PY{p}{(}\PY{n}{data} \PY{o}{=} \PY{n}{salarios\PYZus{}oc\PYZus{}baja\PYZus{}ccaa}\PY{p}{,} \PY{n}{x\PYZus{}col}\PY{o}{=}\PY{l+s+s2}{\PYZdq{}}\PY{l+s+s2}{periodo}\PY{l+s+s2}{\PYZdq{}}\PY{p}{,} \PY{n}{y\PYZus{}col}\PY{o}{=}\PY{l+s+s2}{\PYZdq{}}\PY{l+s+s2}{salario\PYZus{}año}\PY{l+s+s2}{\PYZdq{}}\PY{p}{,} \PY{n}{label} \PY{o}{=} \PY{l+s+s2}{\PYZdq{}}\PY{l+s+s2}{CCAA}\PY{l+s+s2}{\PYZdq{}}\PY{p}{,}
                           \PY{n}{category\PYZus{}col}\PY{o}{=}\PY{l+s+s2}{\PYZdq{}}\PY{l+s+s2}{ccaa}\PY{l+s+s2}{\PYZdq{}}\PY{p}{,} \PY{n}{xlabel}\PY{o}{=}\PY{l+s+s2}{\PYZdq{}}\PY{l+s+s2}{Año}\PY{l+s+s2}{\PYZdq{}}\PY{p}{,} \PY{n}{ylabel}\PY{o}{=}\PY{l+s+s2}{\PYZdq{}}\PY{l+s+s2}{Salario Anual (€)}\PY{l+s+s2}{\PYZdq{}}\PY{p}{,} \PY{n}{title}\PY{o}{=}\PY{l+s+s2}{\PYZdq{}}\PY{l+s+s2}{Salario de ocupacion baja (CCAA)}\PY{l+s+s2}{\PYZdq{}}\PY{p}{,} 
                           \PY{n}{xticks\PYZus{}rotation}\PY{o}{=}\PY{l+m+mi}{0}\PY{p}{,} \PY{n}{style}\PY{o}{=}\PY{l+s+s2}{\PYZdq{}}\PY{l+s+s2}{whitegrid}\PY{l+s+s2}{\PYZdq{}}\PY{p}{,} \PY{n}{figsize}\PY{o}{=}\PY{p}{(}\PY{l+m+mi}{12}\PY{p}{,} \PY{l+m+mi}{7}\PY{p}{)}\PY{p}{)}
\end{Verbatim}
\end{tcolorbox}

    \begin{center}
    \adjustimage{max size={0.9\linewidth}{0.9\paperheight}}{tfm_project_files/tfm_project_47_0.png}
    \end{center}
    { \hspace*{\fill} \\}
    
    Es interesante observar que en País Vasco y Navarra los salarios están
en constante creciemiento, mientras que para las otras comunidades
mostradas la tendencia es relativamente plana hasta 2015, a partir de
cuando se empieza a ver un crecimiento más sólido, coincidiendo con el
comienzo de los incrementos del salario mínmo más fuertes, lo que nos
indica que no todas las comunidades autónomas siguen una correlación
similar.

Es importante que de 2018 a 2019 (donde se produjo el mayor incremento
de salario mínimo hasta la fecha) no necesariamente tenemos que ver un
gran incremento del salario, puesto que una reducción del número de
horas trabajadas amortiguaría el incremento.

    Es interesante también, para enconctrar posibles grupos afectados,
estudiar la evolución de los salarios en función de la jornada y del
contrato.

    \begin{tcolorbox}[breakable, size=fbox, boxrule=1pt, pad at break*=1mm,colback=cellbackground, colframe=cellborder]
\prompt{In}{incolor}{18}{\boxspacing}
\begin{Verbatim}[commandchars=\\\{\}]
\PY{n}{salarios\PYZus{}jornada} \PY{o}{=} \PY{n}{pd}\PY{o}{.}\PY{n}{read\PYZus{}csv}\PY{p}{(}\PY{l+s+s2}{\PYZdq{}}\PY{l+s+s2}{../../processed\PYZus{}data/salarios/jornada.csv}\PY{l+s+s2}{\PYZdq{}}\PY{p}{)}
\PY{n}{salarios\PYZus{}jornada}\PY{p}{[}\PY{l+s+s2}{\PYZdq{}}\PY{l+s+s2}{salario\PYZus{}año}\PY{l+s+s2}{\PYZdq{}}\PY{p}{]} \PY{o}{=} \PY{n}{salarios\PYZus{}jornada}\PY{p}{[}\PY{l+s+s1}{\PYZsq{}}\PY{l+s+s1}{salario\PYZus{}mes}\PY{l+s+s1}{\PYZsq{}}\PY{p}{]}\PY{o}{*}\PY{l+m+mi}{12}
\PY{n}{salarios\PYZus{}jornada\PYZus{}nacional} \PY{o}{=} \PY{n}{salarios\PYZus{}jornada}\PY{p}{[}\PY{p}{(}\PY{n}{salarios\PYZus{}jornada}\PY{p}{[}\PY{l+s+s1}{\PYZsq{}}\PY{l+s+s1}{ccaa}\PY{l+s+s1}{\PYZsq{}}\PY{p}{]} \PY{o}{==} \PY{l+s+s2}{\PYZdq{}}\PY{l+s+s2}{Total Nacional}\PY{l+s+s2}{\PYZdq{}}\PY{p}{)} 
                                             \PY{o}{\PYZam{}} \PY{p}{(}\PY{n}{salarios\PYZus{}jornada}\PY{p}{[}\PY{l+s+s1}{\PYZsq{}}\PY{l+s+s1}{decil}\PY{l+s+s1}{\PYZsq{}}\PY{p}{]} \PY{o}{==} \PY{l+s+s2}{\PYZdq{}}\PY{l+s+s2}{Total decil}\PY{l+s+s2}{\PYZdq{}}\PY{p}{)}\PY{p}{]}
\PY{n}{p}\PY{o}{.}\PY{n}{create\PYZus{}multi\PYZus{}category\PYZus{}plot}\PY{p}{(}\PY{n}{data} \PY{o}{=} \PY{n}{salarios\PYZus{}jornada\PYZus{}nacional}\PY{p}{,} \PY{n}{x\PYZus{}col}\PY{o}{=}\PY{l+s+s2}{\PYZdq{}}\PY{l+s+s2}{periodo}\PY{l+s+s2}{\PYZdq{}}\PY{p}{,} \PY{n}{y\PYZus{}col}\PY{o}{=}\PY{l+s+s2}{\PYZdq{}}\PY{l+s+s2}{salario\PYZus{}año}\PY{l+s+s2}{\PYZdq{}}\PY{p}{,} 
                           \PY{n}{label} \PY{o}{=} \PY{l+s+s2}{\PYZdq{}}\PY{l+s+s2}{Tipo de Jornada}\PY{l+s+s2}{\PYZdq{}}\PY{p}{,}\PY{n}{category\PYZus{}col}\PY{o}{=}\PY{l+s+s2}{\PYZdq{}}\PY{l+s+s2}{jornada}\PY{l+s+s2}{\PYZdq{}}\PY{p}{,} \PY{n}{xlabel}\PY{o}{=}\PY{l+s+s2}{\PYZdq{}}\PY{l+s+s2}{Año}\PY{l+s+s2}{\PYZdq{}}\PY{p}{,} \PY{n}{ylabel}\PY{o}{=}\PY{l+s+s2}{\PYZdq{}}\PY{l+s+s2}{Salario Anual (€)}\PY{l+s+s2}{\PYZdq{}}\PY{p}{,} 
                           \PY{n}{title}\PY{o}{=}\PY{l+s+s2}{\PYZdq{}}\PY{l+s+s2}{Salario por tipo de Joranda (Nacional)}\PY{l+s+s2}{\PYZdq{}}\PY{p}{,} \PY{n}{xticks\PYZus{}rotation}\PY{o}{=}\PY{l+m+mi}{0}\PY{p}{,} \PY{n}{style}\PY{o}{=}\PY{l+s+s2}{\PYZdq{}}\PY{l+s+s2}{whitegrid}\PY{l+s+s2}{\PYZdq{}}\PY{p}{,} 
                           \PY{n}{figsize}\PY{o}{=}\PY{p}{(}\PY{l+m+mi}{12}\PY{p}{,} \PY{l+m+mi}{7}\PY{p}{)}\PY{p}{)}
\end{Verbatim}
\end{tcolorbox}

    \begin{center}
    \adjustimage{max size={0.9\linewidth}{0.9\paperheight}}{tfm_project_files/tfm_project_50_0.png}
    \end{center}
    { \hspace*{\fill} \\}
    
    De nuevo se puede extraer una conclusión similar a la que comentábamos
previamente, atendiendo a que el crecimiento de los sueldos se empieza a
notar después de 2015 para los trabajadores a jornada parcial. Veamos si
se puede apreciar algo por sector.

    \begin{tcolorbox}[breakable, size=fbox, boxrule=1pt, pad at break*=1mm,colback=cellbackground, colframe=cellborder]
\prompt{In}{incolor}{19}{\boxspacing}
\begin{Verbatim}[commandchars=\\\{\}]
\PY{n}{salarios\PYZus{}sector} \PY{o}{=} \PY{n}{pd}\PY{o}{.}\PY{n}{read\PYZus{}csv}\PY{p}{(}\PY{l+s+s2}{\PYZdq{}}\PY{l+s+s2}{../../processed\PYZus{}data/salarios/sector.csv}\PY{l+s+s2}{\PYZdq{}}\PY{p}{)}
\PY{n}{salarios\PYZus{}sector\PYZus{}nacional} \PY{o}{=} \PY{n}{salarios\PYZus{}sector}\PY{p}{[}\PY{p}{(}\PY{n}{salarios\PYZus{}sector}\PY{p}{[}\PY{l+s+s1}{\PYZsq{}}\PY{l+s+s1}{ccaa}\PY{l+s+s1}{\PYZsq{}}\PY{p}{]} \PY{o}{==} \PY{l+s+s2}{\PYZdq{}}\PY{l+s+s2}{Total Nacional}\PY{l+s+s2}{\PYZdq{}}\PY{p}{)} 
                                             \PY{o}{\PYZam{}} \PY{p}{(}\PY{n}{salarios\PYZus{}sector}\PY{p}{[}\PY{l+s+s1}{\PYZsq{}}\PY{l+s+s1}{sexo}\PY{l+s+s1}{\PYZsq{}}\PY{p}{]} \PY{o}{==} \PY{l+s+s2}{\PYZdq{}}\PY{l+s+s2}{Ambos sexos}\PY{l+s+s2}{\PYZdq{}}\PY{p}{)}\PY{p}{]}
\PY{n}{p}\PY{o}{.}\PY{n}{create\PYZus{}multi\PYZus{}category\PYZus{}plot}\PY{p}{(}\PY{n}{data} \PY{o}{=} \PY{n}{salarios\PYZus{}sector\PYZus{}nacional}\PY{p}{,} \PY{n}{x\PYZus{}col}\PY{o}{=}\PY{l+s+s2}{\PYZdq{}}\PY{l+s+s2}{periodo}\PY{l+s+s2}{\PYZdq{}}\PY{p}{,} \PY{n}{y\PYZus{}col}\PY{o}{=}\PY{l+s+s2}{\PYZdq{}}\PY{l+s+s2}{salario\PYZus{}año}\PY{l+s+s2}{\PYZdq{}}\PY{p}{,} 
                           \PY{n}{label} \PY{o}{=} \PY{l+s+s2}{\PYZdq{}}\PY{l+s+s2}{Sector}\PY{l+s+s2}{\PYZdq{}}\PY{p}{,}\PY{n}{category\PYZus{}col}\PY{o}{=}\PY{l+s+s2}{\PYZdq{}}\PY{l+s+s2}{sector}\PY{l+s+s2}{\PYZdq{}}\PY{p}{,} \PY{n}{xlabel}\PY{o}{=}\PY{l+s+s2}{\PYZdq{}}\PY{l+s+s2}{Año}\PY{l+s+s2}{\PYZdq{}}\PY{p}{,} \PY{n}{ylabel}\PY{o}{=}\PY{l+s+s2}{\PYZdq{}}\PY{l+s+s2}{Salario Anual (€)}\PY{l+s+s2}{\PYZdq{}}\PY{p}{,} 
                           \PY{n}{title}\PY{o}{=}\PY{l+s+s2}{\PYZdq{}}\PY{l+s+s2}{Salario por Sector (Nacional)}\PY{l+s+s2}{\PYZdq{}}\PY{p}{,} \PY{n}{xticks\PYZus{}rotation}\PY{o}{=}\PY{l+m+mi}{0}\PY{p}{,} \PY{n}{style}\PY{o}{=}\PY{l+s+s2}{\PYZdq{}}\PY{l+s+s2}{whitegrid}\PY{l+s+s2}{\PYZdq{}}\PY{p}{,} 
                           \PY{n}{figsize}\PY{o}{=}\PY{p}{(}\PY{l+m+mi}{12}\PY{p}{,} \PY{l+m+mi}{7}\PY{p}{)}\PY{p}{)}
\end{Verbatim}
\end{tcolorbox}

    \begin{center}
    \adjustimage{max size={0.9\linewidth}{0.9\paperheight}}{tfm_project_files/tfm_project_52_0.png}
    \end{center}
    { \hspace*{\fill} \\}
    
    De nuevo, observamos que los conjuntos de salarios más bajos empiezan a
tener un crecimiento sostenido tras 2015, mientras que en sectores mejor
pagados como la industria el crecimiento es más continuado en todo el
periodo analizado.

Aprovechando la información de la agencia tributaria, podemos echar un
vistazo a la evolución de los salarios comprendidos entre un número de
SMIs dados.

    \begin{tcolorbox}[breakable, size=fbox, boxrule=1pt, pad at break*=1mm,colback=cellbackground, colframe=cellborder]
\prompt{In}{incolor}{20}{\boxspacing}
\begin{Verbatim}[commandchars=\\\{\}]
\PY{n}{salarios\PYZus{}smis} \PY{o}{=} \PY{n}{pd}\PY{o}{.}\PY{n}{read\PYZus{}csv}\PY{p}{(}\PY{l+s+s2}{\PYZdq{}}\PY{l+s+s2}{../../processed\PYZus{}data/salarios/salarios\PYZus{}smis\PYZus{}aeat.csv}\PY{l+s+s2}{\PYZdq{}}\PY{p}{)}
\PY{n}{smis\PYZus{}to\PYZus{}plot} \PY{o}{=} \PY{p}{[}\PY{l+s+s1}{\PYZsq{}}\PY{l+s+s1}{0\PYZhy{}0.5}\PY{l+s+s1}{\PYZsq{}}\PY{p}{,} \PY{l+s+s1}{\PYZsq{}}\PY{l+s+s1}{0.5\PYZhy{}1}\PY{l+s+s1}{\PYZsq{}}\PY{p}{,} \PY{l+s+s1}{\PYZsq{}}\PY{l+s+s1}{1\PYZhy{}1.5}\PY{l+s+s1}{\PYZsq{}}\PY{p}{,} \PY{l+s+s1}{\PYZsq{}}\PY{l+s+s1}{1.5\PYZhy{}2}\PY{l+s+s1}{\PYZsq{}}\PY{p}{,} \PY{l+s+s1}{\PYZsq{}}\PY{l+s+s1}{2\PYZhy{}2.5}\PY{l+s+s1}{\PYZsq{}}\PY{p}{,} \PY{l+s+s1}{\PYZsq{}}\PY{l+s+s1}{2.5\PYZhy{}3}\PY{l+s+s1}{\PYZsq{}}\PY{p}{]}
\PY{n}{salarios\PYZus{}smis\PYZus{}nacional} \PY{o}{=} \PY{n}{salarios\PYZus{}smis}\PY{p}{[}\PY{p}{(}\PY{n}{salarios\PYZus{}smis}\PY{p}{[}\PY{l+s+s1}{\PYZsq{}}\PY{l+s+s1}{ccaa}\PY{l+s+s1}{\PYZsq{}}\PY{p}{]} \PY{o}{==} \PY{l+s+s2}{\PYZdq{}}\PY{l+s+s2}{Total}\PY{l+s+s2}{\PYZdq{}}\PY{p}{)} \PY{o}{\PYZam{}} 
                                       \PY{p}{(}\PY{n}{salarios\PYZus{}smis}\PY{p}{[}\PY{l+s+s1}{\PYZsq{}}\PY{l+s+s1}{smi}\PY{l+s+s1}{\PYZsq{}}\PY{p}{]}\PY{o}{.}\PY{n}{isin}\PY{p}{(}\PY{n}{smis\PYZus{}to\PYZus{}plot}\PY{p}{)}\PY{p}{)}\PY{p}{]}
\PY{n}{p}\PY{o}{.}\PY{n}{create\PYZus{}multi\PYZus{}category\PYZus{}plot}\PY{p}{(}\PY{n}{data} \PY{o}{=} \PY{n}{salarios\PYZus{}smis\PYZus{}nacional}\PY{p}{,} \PY{n}{x\PYZus{}col}\PY{o}{=}\PY{l+s+s2}{\PYZdq{}}\PY{l+s+s2}{periodo}\PY{l+s+s2}{\PYZdq{}}\PY{p}{,} \PY{n}{y\PYZus{}col}\PY{o}{=}\PY{l+s+s2}{\PYZdq{}}\PY{l+s+s2}{sma}\PY{l+s+s2}{\PYZdq{}}\PY{p}{,} 
                           \PY{n}{label} \PY{o}{=} \PY{l+s+s2}{\PYZdq{}}\PY{l+s+s2}{SMIs}\PY{l+s+s2}{\PYZdq{}}\PY{p}{,}\PY{n}{category\PYZus{}col}\PY{o}{=}\PY{l+s+s2}{\PYZdq{}}\PY{l+s+s2}{smi}\PY{l+s+s2}{\PYZdq{}}\PY{p}{,} \PY{n}{xlabel}\PY{o}{=}\PY{l+s+s2}{\PYZdq{}}\PY{l+s+s2}{Año}\PY{l+s+s2}{\PYZdq{}}\PY{p}{,} \PY{n}{ylabel}\PY{o}{=}\PY{l+s+s2}{\PYZdq{}}\PY{l+s+s2}{Salario Medio Anual (€)}\PY{l+s+s2}{\PYZdq{}}\PY{p}{,} 
                           \PY{n}{title}\PY{o}{=}\PY{l+s+s2}{\PYZdq{}}\PY{l+s+s2}{Salario según rango de SMIs (Nacional)}\PY{l+s+s2}{\PYZdq{}}\PY{p}{,} \PY{n}{xticks\PYZus{}rotation}\PY{o}{=}\PY{l+m+mi}{0}\PY{p}{,} 
                           \PY{n}{style}\PY{o}{=}\PY{l+s+s2}{\PYZdq{}}\PY{l+s+s2}{whitegrid}\PY{l+s+s2}{\PYZdq{}}\PY{p}{,} \PY{n}{figsize}\PY{o}{=}\PY{p}{(}\PY{l+m+mi}{12}\PY{p}{,} \PY{l+m+mi}{7}\PY{p}{)}\PY{p}{)}
\end{Verbatim}
\end{tcolorbox}

    \begin{center}
    \adjustimage{max size={0.9\linewidth}{0.9\paperheight}}{tfm_project_files/tfm_project_54_0.png}
    \end{center}
    { \hspace*{\fill} \\}
    
    \begin{tcolorbox}[breakable, size=fbox, boxrule=1pt, pad at break*=1mm,colback=cellbackground, colframe=cellborder]
\prompt{In}{incolor}{21}{\boxspacing}
\begin{Verbatim}[commandchars=\\\{\}]
\PY{n}{smis\PYZus{}to\PYZus{}plot} \PY{o}{=} \PY{p}{[}\PY{l+s+s1}{\PYZsq{}}\PY{l+s+s1}{0\PYZhy{}0.5}\PY{l+s+s1}{\PYZsq{}}\PY{p}{]}
\PY{n}{salarios\PYZus{}smis\PYZus{}ccaa} \PY{o}{=} \PY{n}{salarios\PYZus{}smis}\PY{p}{[}\PY{p}{(}\PY{n}{salarios\PYZus{}smis}\PY{p}{[}\PY{l+s+s1}{\PYZsq{}}\PY{l+s+s1}{ccaa}\PY{l+s+s1}{\PYZsq{}}\PY{p}{]}\PY{o}{.}\PY{n}{isin}\PY{p}{(}\PY{n}{CCAA\PYZus{}mostrar}\PY{p}{)}\PY{p}{)} \PY{o}{\PYZam{}} 
                                   \PY{p}{(}\PY{n}{salarios\PYZus{}smis}\PY{p}{[}\PY{l+s+s1}{\PYZsq{}}\PY{l+s+s1}{smi}\PY{l+s+s1}{\PYZsq{}}\PY{p}{]}\PY{o}{.}\PY{n}{isin}\PY{p}{(}\PY{n}{smis\PYZus{}to\PYZus{}plot}\PY{p}{)}\PY{p}{)}\PY{p}{]}
\PY{n}{salarios\PYZus{}smis\PYZus{}ccaa\PYZus{}total} \PY{o}{=} \PY{n}{salarios\PYZus{}smis}\PY{p}{[}\PY{p}{(}\PY{n}{salarios\PYZus{}smis}\PY{p}{[}\PY{l+s+s1}{\PYZsq{}}\PY{l+s+s1}{ccaa}\PY{l+s+s1}{\PYZsq{}}\PY{p}{]}\PY{o}{.}\PY{n}{isin}\PY{p}{(}\PY{n}{CCAA\PYZus{}mostrar}\PY{p}{)}\PY{p}{)} \PY{o}{\PYZam{}} 
                                         \PY{p}{(}\PY{n}{salarios\PYZus{}smis}\PY{p}{[}\PY{l+s+s1}{\PYZsq{}}\PY{l+s+s1}{smi}\PY{l+s+s1}{\PYZsq{}}\PY{p}{]}\PY{o}{==}\PY{l+s+s2}{\PYZdq{}}\PY{l+s+s2}{Total}\PY{l+s+s2}{\PYZdq{}}\PY{p}{)}\PY{p}{]}\PY{o}{.}\PY{n}{rename}\PY{p}{(}\PY{n}{columns}\PY{o}{=}\PY{p}{\PYZob{}}\PY{l+s+s1}{\PYZsq{}}\PY{l+s+s1}{asalariados}\PY{l+s+s1}{\PYZsq{}}\PY{p}{:} \PY{l+s+s1}{\PYZsq{}}\PY{l+s+s1}{asalariados\PYZus{}total}\PY{l+s+s1}{\PYZsq{}}\PY{p}{\PYZcb{}}\PY{p}{)}
\PY{n}{salarios\PYZus{}smis\PYZus{}ccaa} \PY{o}{=} \PY{n}{salarios\PYZus{}smis\PYZus{}ccaa}\PY{o}{.}\PY{n}{merge}\PY{p}{(}\PY{n}{salarios\PYZus{}smis\PYZus{}ccaa\PYZus{}total}\PY{p}{,} 
                                              \PY{n}{how}\PY{o}{=}\PY{l+s+s2}{\PYZdq{}}\PY{l+s+s2}{left}\PY{l+s+s2}{\PYZdq{}}\PY{p}{,} \PY{n}{on}\PY{o}{=}\PY{p}{[}\PY{l+s+s2}{\PYZdq{}}\PY{l+s+s2}{periodo}\PY{l+s+s2}{\PYZdq{}}\PY{p}{,} \PY{l+s+s2}{\PYZdq{}}\PY{l+s+s2}{ccaa}\PY{l+s+s2}{\PYZdq{}}\PY{p}{]}\PY{p}{)}
\PY{n}{salarios\PYZus{}smis\PYZus{}ccaa}\PY{p}{[}\PY{l+s+s1}{\PYZsq{}}\PY{l+s+s1}{prop\PYZus{}0\PYZus{}05}\PY{l+s+s1}{\PYZsq{}}\PY{p}{]} \PY{o}{=} \PY{n}{salarios\PYZus{}smis\PYZus{}ccaa}\PY{p}{[}\PY{l+s+s1}{\PYZsq{}}\PY{l+s+s1}{asalariados}\PY{l+s+s1}{\PYZsq{}}\PY{p}{]}\PY{o}{/}\PY{n}{salarios\PYZus{}smis\PYZus{}ccaa}\PY{p}{[}\PY{l+s+s1}{\PYZsq{}}\PY{l+s+s1}{asalariados\PYZus{}total}\PY{l+s+s1}{\PYZsq{}}\PY{p}{]}\PY{o}{*}\PY{l+m+mi}{100}

\PY{n}{p}\PY{o}{.}\PY{n}{create\PYZus{}multi\PYZus{}category\PYZus{}plot}\PY{p}{(}\PY{n}{data} \PY{o}{=} \PY{n}{salarios\PYZus{}smis\PYZus{}ccaa}\PY{p}{,} \PY{n}{x\PYZus{}col}\PY{o}{=}\PY{l+s+s2}{\PYZdq{}}\PY{l+s+s2}{periodo}\PY{l+s+s2}{\PYZdq{}}\PY{p}{,} \PY{n}{y\PYZus{}col}\PY{o}{=}\PY{l+s+s2}{\PYZdq{}}\PY{l+s+s2}{prop\PYZus{}0\PYZus{}05}\PY{l+s+s2}{\PYZdq{}}\PY{p}{,} \PY{n}{label} \PY{o}{=} \PY{l+s+s2}{\PYZdq{}}\PY{l+s+s2}{CCAA}\PY{l+s+s2}{\PYZdq{}}\PY{p}{,}
                           \PY{n}{category\PYZus{}col}\PY{o}{=}\PY{l+s+s2}{\PYZdq{}}\PY{l+s+s2}{ccaa}\PY{l+s+s2}{\PYZdq{}}\PY{p}{,} \PY{n}{xlabel}\PY{o}{=}\PY{l+s+s2}{\PYZdq{}}\PY{l+s+s2}{Año}\PY{l+s+s2}{\PYZdq{}}\PY{p}{,} 
                           \PY{n}{ylabel}\PY{o}{=}\PY{l+s+s2}{\PYZdq{}}\PY{l+s+si}{\PYZpc{} d}\PY{l+s+s2}{e Asalariados}\PY{l+s+s2}{\PYZdq{}}\PY{p}{,} \PY{n}{title}\PY{o}{=}\PY{l+s+s2}{\PYZdq{}}\PY{l+s+s2}{Asalariados en el rango 0\PYZhy{}0.5 SMIs}\PY{l+s+s2}{\PYZdq{}}\PY{p}{,} 
                           \PY{n}{xticks\PYZus{}rotation}\PY{o}{=}\PY{l+m+mi}{0}\PY{p}{,} \PY{n}{style}\PY{o}{=}\PY{l+s+s2}{\PYZdq{}}\PY{l+s+s2}{whitegrid}\PY{l+s+s2}{\PYZdq{}}\PY{p}{,} \PY{n}{figsize}\PY{o}{=}\PY{p}{(}\PY{l+m+mi}{12}\PY{p}{,} \PY{l+m+mi}{7}\PY{p}{)}\PY{p}{)}
\end{Verbatim}
\end{tcolorbox}

    \begin{center}
    \adjustimage{max size={0.9\linewidth}{0.9\paperheight}}{tfm_project_files/tfm_project_55_0.png}
    \end{center}
    { \hspace*{\fill} \\}
    
    Como era esperado, la evolución es igual en todos los estratos, ya que a
medida que aumenta el salario mínimo por definición el salario medio de
un rango en salarios mínimos subirán. En lo que respecta a las
comunidades autónoma se aprecian diferencias nada despreciables en el
porcentaje de trabajadores que cobran en cierto rango de salarios
mínimos.

Esta variable tiene algo de limitaciones, pues no contiene información
de País Vasco y Navarra debido a su régimen particular, pero nos aporta
información muy útil de los posibles afectados por la subida del salario
mínimo.

    \subsubsection{Empleos}\label{empleos}

    El número de ocupados es otra de las métricas que puede ser afectadas
por un incremento del salario mínimo, por lo que es importante estudiar
el cambio de estos en el tiempo.

    \begin{tcolorbox}[breakable, size=fbox, boxrule=1pt, pad at break*=1mm,colback=cellbackground, colframe=cellborder]
\prompt{In}{incolor}{22}{\boxspacing}
\begin{Verbatim}[commandchars=\\\{\}]
\PY{n}{ocupados\PYZus{}jornada} \PY{o}{=} \PY{n}{pd}\PY{o}{.}\PY{n}{read\PYZus{}csv}\PY{p}{(}\PY{l+s+s2}{\PYZdq{}}\PY{l+s+s2}{../../processed\PYZus{}data/trabajo/ocupados\PYZus{}jornada.csv}\PY{l+s+s2}{\PYZdq{}}\PY{p}{)}
\PY{c+c1}{\PYZsh{}Agrupamos por año haciendo la media}
\PY{n}{ocupados\PYZus{}jornada} \PY{o}{=} \PY{n}{ocupados\PYZus{}jornada}\PY{o}{.}\PY{n}{groupby}\PY{p}{(}\PY{p}{[}\PY{l+s+s1}{\PYZsq{}}\PY{l+s+s1}{sexo}\PY{l+s+s1}{\PYZsq{}}\PY{p}{,} \PY{l+s+s1}{\PYZsq{}}\PY{l+s+s1}{ccaa}\PY{l+s+s1}{\PYZsq{}}\PY{p}{,} \PY{l+s+s1}{\PYZsq{}}\PY{l+s+s1}{tipo\PYZus{}jornada}\PY{l+s+s1}{\PYZsq{}}\PY{p}{,} \PY{l+s+s1}{\PYZsq{}}\PY{l+s+s1}{unidad}\PY{l+s+s1}{\PYZsq{}}\PY{p}{,} \PY{l+s+s1}{\PYZsq{}}\PY{l+s+s1}{periodo}\PY{l+s+s1}{\PYZsq{}}\PY{p}{]}\PY{p}{,} \PY{n}{as\PYZus{}index}\PY{o}{=}\PY{k+kc}{False}\PY{p}{)}\PY{o}{.}\PY{n}{mean}\PY{p}{(}\PY{n}{numeric\PYZus{}only}\PY{o}{=}\PY{k+kc}{True}\PY{p}{)}
\PY{n}{ocupados\PYZus{}jornada\PYZus{}nacional} \PY{o}{=} \PY{n}{ocupados\PYZus{}jornada}\PY{p}{[}\PY{p}{(}\PY{n}{ocupados\PYZus{}jornada}\PY{p}{[}\PY{l+s+s1}{\PYZsq{}}\PY{l+s+s1}{ccaa}\PY{l+s+s1}{\PYZsq{}}\PY{p}{]} \PY{o}{==} \PY{l+s+s2}{\PYZdq{}}\PY{l+s+s2}{Total Nacional}\PY{l+s+s2}{\PYZdq{}}\PY{p}{)} \PY{o}{\PYZam{}} \PY{p}{(}\PY{n}{ocupados\PYZus{}jornada}\PY{p}{[}\PY{l+s+s1}{\PYZsq{}}\PY{l+s+s1}{sexo}\PY{l+s+s1}{\PYZsq{}}\PY{p}{]} \PY{o}{==} \PY{l+s+s2}{\PYZdq{}}\PY{l+s+s2}{Ambos sexos}\PY{l+s+s2}{\PYZdq{}}\PY{p}{)}
                                             \PY{o}{\PYZam{}} \PY{p}{(}\PY{n}{ocupados\PYZus{}jornada}\PY{p}{[}\PY{l+s+s1}{\PYZsq{}}\PY{l+s+s1}{unidad}\PY{l+s+s1}{\PYZsq{}}\PY{p}{]} \PY{o}{==} \PY{l+s+s2}{\PYZdq{}}\PY{l+s+s2}{Valor absoluto}\PY{l+s+s2}{\PYZdq{}}\PY{p}{)}\PY{p}{]}
\PY{n}{p}\PY{o}{.}\PY{n}{create\PYZus{}multi\PYZus{}category\PYZus{}plot}\PY{p}{(}\PY{n}{data} \PY{o}{=} \PY{n}{ocupados\PYZus{}jornada\PYZus{}nacional}\PY{p}{,} \PY{n}{x\PYZus{}col}\PY{o}{=}\PY{l+s+s2}{\PYZdq{}}\PY{l+s+s2}{periodo}\PY{l+s+s2}{\PYZdq{}}\PY{p}{,} \PY{n}{y\PYZus{}col}\PY{o}{=}\PY{l+s+s2}{\PYZdq{}}\PY{l+s+s2}{Total}\PY{l+s+s2}{\PYZdq{}}\PY{p}{,} 
                           \PY{n}{label} \PY{o}{=} \PY{l+s+s2}{\PYZdq{}}\PY{l+s+s2}{Tipo de Jornada}\PY{l+s+s2}{\PYZdq{}}\PY{p}{,}\PY{n}{category\PYZus{}col}\PY{o}{=}\PY{l+s+s2}{\PYZdq{}}\PY{l+s+s2}{tipo\PYZus{}jornada}\PY{l+s+s2}{\PYZdq{}}\PY{p}{,} \PY{n}{xlabel}\PY{o}{=}\PY{l+s+s2}{\PYZdq{}}\PY{l+s+s2}{Año}\PY{l+s+s2}{\PYZdq{}}\PY{p}{,} 
                           \PY{n}{ylabel}\PY{o}{=}\PY{l+s+s2}{\PYZdq{}}\PY{l+s+s2}{Ocupados (Miles de Personas)}\PY{l+s+s2}{\PYZdq{}}\PY{p}{,} \PY{n}{title}\PY{o}{=}\PY{l+s+s2}{\PYZdq{}}\PY{l+s+s2}{Ocupados por tipo de Jornada (Nacional)}\PY{l+s+s2}{\PYZdq{}}\PY{p}{,} 
                           \PY{n}{xticks\PYZus{}rotation}\PY{o}{=}\PY{l+m+mi}{0}\PY{p}{,} \PY{n}{style}\PY{o}{=}\PY{l+s+s2}{\PYZdq{}}\PY{l+s+s2}{whitegrid}\PY{l+s+s2}{\PYZdq{}}\PY{p}{,} \PY{n}{figsize}\PY{o}{=}\PY{p}{(}\PY{l+m+mi}{12}\PY{p}{,} \PY{l+m+mi}{7}\PY{p}{)}\PY{p}{)}
\end{Verbatim}
\end{tcolorbox}

    \begin{center}
    \adjustimage{max size={0.9\linewidth}{0.9\paperheight}}{tfm_project_files/tfm_project_59_0.png}
    \end{center}
    { \hspace*{\fill} \\}
    
    Se aprecia de manera clara que el número de ocupados crece especialmente
en tiempo completo, siendo el crecimiento del trabajo a tiempo parcial
algo más residual en el tiempo, si bien es cierto qu elos porcentajes de
cada uno sobre el total tienen una mayor variación. Veamos cómo
evoluciona el porcentaje de ocupados a tiempo parcial dentro de cada
comunidad autónoma.

    \begin{tcolorbox}[breakable, size=fbox, boxrule=1pt, pad at break*=1mm,colback=cellbackground, colframe=cellborder]
\prompt{In}{incolor}{23}{\boxspacing}
\begin{Verbatim}[commandchars=\\\{\}]
\PY{n}{ocupados\PYZus{}jornada\PYZus{}ccaa} \PY{o}{=} \PY{n}{ocupados\PYZus{}jornada}\PY{p}{[}\PY{p}{(}\PY{n}{ocupados\PYZus{}jornada}\PY{p}{[}\PY{l+s+s1}{\PYZsq{}}\PY{l+s+s1}{ccaa}\PY{l+s+s1}{\PYZsq{}}\PY{p}{]}\PY{o}{.}\PY{n}{isin}\PY{p}{(}\PY{n}{CCAA\PYZus{}mostrar}\PY{p}{)}\PY{p}{)} \PY{o}{\PYZam{}} 
                                         \PY{p}{(}\PY{n}{ocupados\PYZus{}jornada}\PY{p}{[}\PY{l+s+s1}{\PYZsq{}}\PY{l+s+s1}{sexo}\PY{l+s+s1}{\PYZsq{}}\PY{p}{]} \PY{o}{==} \PY{l+s+s2}{\PYZdq{}}\PY{l+s+s2}{Ambos sexos}\PY{l+s+s2}{\PYZdq{}}\PY{p}{)}
                                             \PY{o}{\PYZam{}} \PY{p}{(}\PY{n}{ocupados\PYZus{}jornada}\PY{p}{[}\PY{l+s+s1}{\PYZsq{}}\PY{l+s+s1}{unidad}\PY{l+s+s1}{\PYZsq{}}\PY{p}{]} \PY{o}{==} \PY{l+s+s2}{\PYZdq{}}\PY{l+s+s2}{Porcentaje}\PY{l+s+s2}{\PYZdq{}}\PY{p}{)} \PY{o}{\PYZam{}} 
                                             \PY{p}{(}\PY{n}{ocupados\PYZus{}jornada}\PY{p}{[}\PY{l+s+s1}{\PYZsq{}}\PY{l+s+s1}{tipo\PYZus{}jornada}\PY{l+s+s1}{\PYZsq{}}\PY{p}{]} \PY{o}{==} \PY{l+s+s2}{\PYZdq{}}\PY{l+s+s2}{Jornada a tiempo parcial}\PY{l+s+s2}{\PYZdq{}}\PY{p}{)}\PY{p}{]}
\PY{n}{p}\PY{o}{.}\PY{n}{create\PYZus{}multi\PYZus{}category\PYZus{}plot}\PY{p}{(}\PY{n}{data} \PY{o}{=} \PY{n}{ocupados\PYZus{}jornada\PYZus{}ccaa}\PY{p}{,} \PY{n}{x\PYZus{}col}\PY{o}{=}\PY{l+s+s2}{\PYZdq{}}\PY{l+s+s2}{periodo}\PY{l+s+s2}{\PYZdq{}}\PY{p}{,} \PY{n}{y\PYZus{}col}\PY{o}{=}\PY{l+s+s2}{\PYZdq{}}\PY{l+s+s2}{Total}\PY{l+s+s2}{\PYZdq{}}\PY{p}{,} \PY{n}{label} \PY{o}{=} \PY{l+s+s2}{\PYZdq{}}\PY{l+s+s2}{CCAA}\PY{l+s+s2}{\PYZdq{}}\PY{p}{,}
                           \PY{n}{category\PYZus{}col}\PY{o}{=}\PY{l+s+s2}{\PYZdq{}}\PY{l+s+s2}{ccaa}\PY{l+s+s2}{\PYZdq{}}\PY{p}{,} \PY{n}{xlabel}\PY{o}{=}\PY{l+s+s2}{\PYZdq{}}\PY{l+s+s2}{Año}\PY{l+s+s2}{\PYZdq{}}\PY{p}{,} \PY{n}{ylabel}\PY{o}{=}\PY{l+s+s2}{\PYZdq{}}\PY{l+s+s2}{Porcentaje (}\PY{l+s+s2}{\PYZpc{}}\PY{l+s+s2}{)}\PY{l+s+s2}{\PYZdq{}}\PY{p}{,} 
                           \PY{n}{title}\PY{o}{=}\PY{l+s+s2}{\PYZdq{}}\PY{l+s+s2}{Ocupacion a tiempo parcial (CCAA)}\PY{l+s+s2}{\PYZdq{}}\PY{p}{,} \PY{n}{xticks\PYZus{}rotation}\PY{o}{=}\PY{l+m+mi}{0}\PY{p}{,}
                           \PY{n}{style}\PY{o}{=}\PY{l+s+s2}{\PYZdq{}}\PY{l+s+s2}{whitegrid}\PY{l+s+s2}{\PYZdq{}}\PY{p}{,} \PY{n}{figsize}\PY{o}{=}\PY{p}{(}\PY{l+m+mi}{12}\PY{p}{,} \PY{l+m+mi}{7}\PY{p}{)}\PY{p}{)}
\end{Verbatim}
\end{tcolorbox}

    \begin{center}
    \adjustimage{max size={0.9\linewidth}{0.9\paperheight}}{tfm_project_files/tfm_project_61_0.png}
    \end{center}
    { \hspace*{\fill} \\}
    
    \begin{tcolorbox}[breakable, size=fbox, boxrule=1pt, pad at break*=1mm,colback=cellbackground, colframe=cellborder]
\prompt{In}{incolor}{24}{\boxspacing}
\begin{Verbatim}[commandchars=\\\{\}]
\PY{n}{ocupados\PYZus{}jornada\PYZus{}ccaa} \PY{o}{=} \PY{n}{ocupados\PYZus{}jornada}\PY{p}{[}\PY{p}{(}\PY{n}{ocupados\PYZus{}jornada}\PY{p}{[}\PY{l+s+s1}{\PYZsq{}}\PY{l+s+s1}{ccaa}\PY{l+s+s1}{\PYZsq{}}\PY{p}{]}\PY{o}{.}\PY{n}{isin}\PY{p}{(}\PY{n}{CCAA\PYZus{}mostrar}\PY{p}{)}\PY{p}{)} \PY{o}{\PYZam{}} 
                                         \PY{p}{(}\PY{n}{ocupados\PYZus{}jornada}\PY{p}{[}\PY{l+s+s1}{\PYZsq{}}\PY{l+s+s1}{sexo}\PY{l+s+s1}{\PYZsq{}}\PY{p}{]} \PY{o}{==} \PY{l+s+s2}{\PYZdq{}}\PY{l+s+s2}{Ambos sexos}\PY{l+s+s2}{\PYZdq{}}\PY{p}{)} \PY{o}{\PYZam{}} 
                                         \PY{p}{(}\PY{n}{ocupados\PYZus{}jornada}\PY{p}{[}\PY{l+s+s1}{\PYZsq{}}\PY{l+s+s1}{unidad}\PY{l+s+s1}{\PYZsq{}}\PY{p}{]} \PY{o}{==} \PY{l+s+s2}{\PYZdq{}}\PY{l+s+s2}{Valor absoluto}\PY{l+s+s2}{\PYZdq{}}\PY{p}{)} \PY{o}{\PYZam{}} 
                                         \PY{p}{(}\PY{n}{ocupados\PYZus{}jornada}\PY{p}{[}\PY{l+s+s1}{\PYZsq{}}\PY{l+s+s1}{tipo\PYZus{}jornada}\PY{l+s+s1}{\PYZsq{}}\PY{p}{]} \PY{o}{==} \PY{l+s+s2}{\PYZdq{}}\PY{l+s+s2}{Total}\PY{l+s+s2}{\PYZdq{}}\PY{p}{)}\PY{p}{]}
\PY{n}{p}\PY{o}{.}\PY{n}{create\PYZus{}multi\PYZus{}category\PYZus{}plot}\PY{p}{(}\PY{n}{data} \PY{o}{=} \PY{n}{ocupados\PYZus{}jornada\PYZus{}ccaa}\PY{p}{,} \PY{n}{x\PYZus{}col}\PY{o}{=}\PY{l+s+s2}{\PYZdq{}}\PY{l+s+s2}{periodo}\PY{l+s+s2}{\PYZdq{}}\PY{p}{,} \PY{n}{y\PYZus{}col}\PY{o}{=}\PY{l+s+s2}{\PYZdq{}}\PY{l+s+s2}{Total}\PY{l+s+s2}{\PYZdq{}}\PY{p}{,} \PY{n}{label} \PY{o}{=} \PY{l+s+s2}{\PYZdq{}}\PY{l+s+s2}{CCAA}\PY{l+s+s2}{\PYZdq{}}\PY{p}{,}
                           \PY{n}{category\PYZus{}col}\PY{o}{=}\PY{l+s+s2}{\PYZdq{}}\PY{l+s+s2}{ccaa}\PY{l+s+s2}{\PYZdq{}}\PY{p}{,} \PY{n}{xlabel}\PY{o}{=}\PY{l+s+s2}{\PYZdq{}}\PY{l+s+s2}{Año}\PY{l+s+s2}{\PYZdq{}}\PY{p}{,} \PY{n}{ylabel}\PY{o}{=}\PY{l+s+s2}{\PYZdq{}}\PY{l+s+s2}{Ocupados (Miles de personas)}\PY{l+s+s2}{\PYZdq{}}\PY{p}{,} 
                           \PY{n}{title}\PY{o}{=}\PY{l+s+s2}{\PYZdq{}}\PY{l+s+s2}{Ocupacion (CCAA)}\PY{l+s+s2}{\PYZdq{}}\PY{p}{,} \PY{n}{xticks\PYZus{}rotation}\PY{o}{=}\PY{l+m+mi}{0}\PY{p}{,} \PY{n}{style}\PY{o}{=}\PY{l+s+s2}{\PYZdq{}}\PY{l+s+s2}{whitegrid}\PY{l+s+s2}{\PYZdq{}}\PY{p}{,} 
                           \PY{n}{figsize}\PY{o}{=}\PY{p}{(}\PY{l+m+mi}{12}\PY{p}{,} \PY{l+m+mi}{7}\PY{p}{)}\PY{p}{)}
\end{Verbatim}
\end{tcolorbox}

    \begin{center}
    \adjustimage{max size={0.9\linewidth}{0.9\paperheight}}{tfm_project_files/tfm_project_62_0.png}
    \end{center}
    { \hspace*{\fill} \\}
    
    De nuevo, la tendencia es conjunta, pero la diferencia entre comunidades
autónomas es notable. En los años de mayores subida del salario mínimo
no se observa a priori ninguna variación que nos pueda indicar un
impacto del salario mínimo en dichos años. El número total de ocupados
tampoco parece indicarnos nada concreto.

    \subsubsection{PIB}\label{pib}

    El producto interior bruto (PIB), aparte de ser un buen indicador del
tamaño de la región, es útil para determinar la riqueza de dicha región
al dividirlo entre su número de habitantes (lo que sería el PIB per
cápita).

    \begin{tcolorbox}[breakable, size=fbox, boxrule=1pt, pad at break*=1mm,colback=cellbackground, colframe=cellborder]
\prompt{In}{incolor}{25}{\boxspacing}
\begin{Verbatim}[commandchars=\\\{\}]
\PY{n}{pib\PYZus{}abs} \PY{o}{=} \PY{n}{pd}\PY{o}{.}\PY{n}{read\PYZus{}csv}\PY{p}{(}\PY{l+s+s2}{\PYZdq{}}\PY{l+s+s2}{../../processed\PYZus{}data/pib/pib\PYZus{}abs.csv}\PY{l+s+s2}{\PYZdq{}}\PY{p}{)}
\PY{n}{pib\PYZus{}per\PYZus{}capita} \PY{o}{=} \PY{n}{pd}\PY{o}{.}\PY{n}{read\PYZus{}csv}\PY{p}{(}\PY{l+s+s2}{\PYZdq{}}\PY{l+s+s2}{../../processed\PYZus{}data/pib/pib\PYZus{}per\PYZus{}capita.csv}\PY{l+s+s2}{\PYZdq{}}\PY{p}{)}
\PY{n}{pib\PYZus{}per\PYZus{}capita\PYZus{}ccaa} \PY{o}{=} \PY{n}{pib\PYZus{}per\PYZus{}capita}\PY{p}{[}\PY{n}{pib\PYZus{}per\PYZus{}capita}\PY{p}{[}\PY{l+s+s1}{\PYZsq{}}\PY{l+s+s1}{ccaa}\PY{l+s+s1}{\PYZsq{}}\PY{p}{]}\PY{o}{.}\PY{n}{isin}\PY{p}{(}\PY{n}{CCAA\PYZus{}mostrar}\PY{p}{)} \PY{o}{\PYZam{}} 
                                     \PY{p}{(}\PY{n}{pib\PYZus{}per\PYZus{}capita}\PY{p}{[}\PY{l+s+s1}{\PYZsq{}}\PY{l+s+s1}{tipo\PYZus{}dato}\PY{l+s+s1}{\PYZsq{}}\PY{p}{]} \PY{o}{==} \PY{l+s+s2}{\PYZdq{}}\PY{l+s+s2}{Valor}\PY{l+s+s2}{\PYZdq{}}\PY{p}{)}\PY{p}{]}
\PY{n}{p}\PY{o}{.}\PY{n}{create\PYZus{}multi\PYZus{}category\PYZus{}plot}\PY{p}{(}\PY{n}{data} \PY{o}{=} \PY{n}{pib\PYZus{}per\PYZus{}capita\PYZus{}ccaa}\PY{p}{,} \PY{n}{x\PYZus{}col}\PY{o}{=}\PY{l+s+s2}{\PYZdq{}}\PY{l+s+s2}{periodo}\PY{l+s+s2}{\PYZdq{}}\PY{p}{,} \PY{n}{y\PYZus{}col}\PY{o}{=}\PY{l+s+s2}{\PYZdq{}}\PY{l+s+s2}{valor}\PY{l+s+s2}{\PYZdq{}}\PY{p}{,} \PY{n}{label} \PY{o}{=} \PY{l+s+s2}{\PYZdq{}}\PY{l+s+s2}{CCAA}\PY{l+s+s2}{\PYZdq{}}\PY{p}{,}
                           \PY{n}{category\PYZus{}col}\PY{o}{=}\PY{l+s+s2}{\PYZdq{}}\PY{l+s+s2}{ccaa}\PY{l+s+s2}{\PYZdq{}}\PY{p}{,} \PY{n}{xlabel}\PY{o}{=}\PY{l+s+s2}{\PYZdq{}}\PY{l+s+s2}{Año}\PY{l+s+s2}{\PYZdq{}}\PY{p}{,} \PY{n}{ylabel}\PY{o}{=}\PY{l+s+s2}{\PYZdq{}}\PY{l+s+s2}{PIB per Cápita (€)}\PY{l+s+s2}{\PYZdq{}}\PY{p}{,} \PY{n}{title}\PY{o}{=}\PY{l+s+s2}{\PYZdq{}}\PY{l+s+s2}{PIB per Cápita (CCAA)}\PY{l+s+s2}{\PYZdq{}}\PY{p}{,} 
                           \PY{n}{xticks\PYZus{}rotation}\PY{o}{=}\PY{l+m+mi}{0}\PY{p}{,} \PY{n}{style}\PY{o}{=}\PY{l+s+s2}{\PYZdq{}}\PY{l+s+s2}{whitegrid}\PY{l+s+s2}{\PYZdq{}}\PY{p}{,} \PY{n}{figsize}\PY{o}{=}\PY{p}{(}\PY{l+m+mi}{12}\PY{p}{,} \PY{l+m+mi}{7}\PY{p}{)}\PY{p}{)}
\end{Verbatim}
\end{tcolorbox}

    \begin{center}
    \adjustimage{max size={0.9\linewidth}{0.9\paperheight}}{tfm_project_files/tfm_project_66_0.png}
    \end{center}
    { \hspace*{\fill} \\}
    
    Observamos que las diferencias entre comunidades autónomas no son nada
despreciables, pues Madrid dobla a lo largo de todo el periodo el PIB
per capita de extremadura. Así mismo se ve además que el crecimiento de
estas regiones más pobres es claramente inferior.

    \subsubsection{Paro}\label{paro}

    El paro nos indica el número de personas que no tienen empleo pero que
estás buscando empleo activamente. Esta variable es de utilidad, pues
variaciones en el paro pueden ser un buen predictor que acompañe al
salario mínimo así como un buen resultado a estudiar a consecuencia de
un incremento del SMI.

    \begin{tcolorbox}[breakable, size=fbox, boxrule=1pt, pad at break*=1mm,colback=cellbackground, colframe=cellborder]
\prompt{In}{incolor}{26}{\boxspacing}
\begin{Verbatim}[commandchars=\\\{\}]
\PY{n}{paro} \PY{o}{=} \PY{n}{pd}\PY{o}{.}\PY{n}{read\PYZus{}csv}\PY{p}{(}\PY{l+s+s2}{\PYZdq{}}\PY{l+s+s2}{../../processed\PYZus{}data/paro/parados.csv}\PY{l+s+s2}{\PYZdq{}}\PY{p}{)}
\PY{n}{paro\PYZus{}ccaa} \PY{o}{=} \PY{n}{paro}\PY{p}{[}\PY{p}{(}\PY{n}{paro}\PY{p}{[}\PY{l+s+s1}{\PYZsq{}}\PY{l+s+s1}{ccaa}\PY{l+s+s1}{\PYZsq{}}\PY{p}{]}\PY{o}{.}\PY{n}{isin}\PY{p}{(}\PY{n}{CCAA\PYZus{}mostrar}\PY{p}{)}\PY{p}{)} \PY{o}{\PYZam{}} 
                 \PY{p}{(}\PY{n}{paro}\PY{p}{[}\PY{l+s+s1}{\PYZsq{}}\PY{l+s+s1}{sexo}\PY{l+s+s1}{\PYZsq{}}\PY{p}{]} \PY{o}{==} \PY{l+s+s2}{\PYZdq{}}\PY{l+s+s2}{Ambos sexos}\PY{l+s+s2}{\PYZdq{}}\PY{p}{)} \PY{o}{\PYZam{}} 
                 \PY{p}{(}\PY{n}{paro}\PY{p}{[}\PY{l+s+s1}{\PYZsq{}}\PY{l+s+s1}{edad}\PY{l+s+s1}{\PYZsq{}}\PY{p}{]} \PY{o}{==} \PY{l+s+s2}{\PYZdq{}}\PY{l+s+s2}{Total}\PY{l+s+s2}{\PYZdq{}}\PY{p}{)}\PY{p}{]}
\PY{n}{p}\PY{o}{.}\PY{n}{create\PYZus{}multi\PYZus{}category\PYZus{}plot}\PY{p}{(}\PY{n}{data} \PY{o}{=} \PY{n}{paro\PYZus{}ccaa}\PY{p}{,} \PY{n}{x\PYZus{}col}\PY{o}{=}\PY{l+s+s2}{\PYZdq{}}\PY{l+s+s2}{periodo}\PY{l+s+s2}{\PYZdq{}}\PY{p}{,} \PY{n}{y\PYZus{}col}\PY{o}{=}\PY{l+s+s2}{\PYZdq{}}\PY{l+s+s2}{tasa\PYZus{}paro\PYZus{}total}\PY{l+s+s2}{\PYZdq{}}\PY{p}{,} 
                           \PY{n}{label} \PY{o}{=} \PY{l+s+s2}{\PYZdq{}}\PY{l+s+s2}{CCAA}\PY{l+s+s2}{\PYZdq{}}\PY{p}{,}\PY{n}{category\PYZus{}col}\PY{o}{=}\PY{l+s+s2}{\PYZdq{}}\PY{l+s+s2}{ccaa}\PY{l+s+s2}{\PYZdq{}}\PY{p}{,} \PY{n}{xlabel}\PY{o}{=}\PY{l+s+s2}{\PYZdq{}}\PY{l+s+s2}{Año}\PY{l+s+s2}{\PYZdq{}}\PY{p}{,} \PY{n}{ylabel}\PY{o}{=}\PY{l+s+s2}{\PYZdq{}}\PY{l+s+s2}{Tasa de paro (}\PY{l+s+s2}{\PYZpc{}}\PY{l+s+s2}{)}\PY{l+s+s2}{\PYZdq{}}\PY{p}{,} 
                           \PY{n}{title}\PY{o}{=}\PY{l+s+s2}{\PYZdq{}}\PY{l+s+s2}{Tasa de paro (CCAA)}\PY{l+s+s2}{\PYZdq{}}\PY{p}{,} \PY{n}{xticks\PYZus{}rotation}\PY{o}{=}\PY{l+m+mi}{0}\PY{p}{,} \PY{n}{style}\PY{o}{=}\PY{l+s+s2}{\PYZdq{}}\PY{l+s+s2}{whitegrid}\PY{l+s+s2}{\PYZdq{}}\PY{p}{,} 
                           \PY{n}{figsize}\PY{o}{=}\PY{p}{(}\PY{l+m+mi}{12}\PY{p}{,} \PY{l+m+mi}{7}\PY{p}{)}\PY{p}{)}
\end{Verbatim}
\end{tcolorbox}

    \begin{center}
    \adjustimage{max size={0.9\linewidth}{0.9\paperheight}}{tfm_project_files/tfm_project_70_0.png}
    \end{center}
    { \hspace*{\fill} \\}
    
    De nuevo, las diferencias entre comunidades autónomas son destacables.
Aprovechando que tenemos el desglose por por edad y sexo, veamos cómo ha
evolucionado el paro según estas dos variables.

    \begin{tcolorbox}[breakable, size=fbox, boxrule=1pt, pad at break*=1mm,colback=cellbackground, colframe=cellborder]
\prompt{In}{incolor}{27}{\boxspacing}
\begin{Verbatim}[commandchars=\\\{\}]
\PY{n}{paro\PYZus{}edad} \PY{o}{=} \PY{n}{paro}\PY{p}{[}\PY{p}{(}\PY{n}{paro}\PY{p}{[}\PY{l+s+s1}{\PYZsq{}}\PY{l+s+s1}{ccaa}\PY{l+s+s1}{\PYZsq{}}\PY{p}{]}\PY{o}{==}\PY{l+s+s2}{\PYZdq{}}\PY{l+s+s2}{Total Nacional}\PY{l+s+s2}{\PYZdq{}}\PY{p}{)} \PY{o}{\PYZam{}} 
                 \PY{p}{(}\PY{n}{paro}\PY{p}{[}\PY{l+s+s1}{\PYZsq{}}\PY{l+s+s1}{sexo}\PY{l+s+s1}{\PYZsq{}}\PY{p}{]} \PY{o}{==} \PY{l+s+s2}{\PYZdq{}}\PY{l+s+s2}{Ambos sexos}\PY{l+s+s2}{\PYZdq{}}\PY{p}{)} \PY{p}{]}
\PY{n}{p}\PY{o}{.}\PY{n}{create\PYZus{}multi\PYZus{}category\PYZus{}plot}\PY{p}{(}\PY{n}{data} \PY{o}{=} \PY{n}{paro\PYZus{}edad}\PY{p}{,} \PY{n}{x\PYZus{}col}\PY{o}{=}\PY{l+s+s2}{\PYZdq{}}\PY{l+s+s2}{periodo}\PY{l+s+s2}{\PYZdq{}}\PY{p}{,} \PY{n}{y\PYZus{}col}\PY{o}{=}\PY{l+s+s2}{\PYZdq{}}\PY{l+s+s2}{tasa\PYZus{}paro\PYZus{}total}\PY{l+s+s2}{\PYZdq{}}\PY{p}{,} 
                           \PY{n}{label} \PY{o}{=} \PY{l+s+s2}{\PYZdq{}}\PY{l+s+s2}{Edad}\PY{l+s+s2}{\PYZdq{}}\PY{p}{,}\PY{n}{category\PYZus{}col}\PY{o}{=}\PY{l+s+s2}{\PYZdq{}}\PY{l+s+s2}{edad}\PY{l+s+s2}{\PYZdq{}}\PY{p}{,} \PY{n}{xlabel}\PY{o}{=}\PY{l+s+s2}{\PYZdq{}}\PY{l+s+s2}{Año}\PY{l+s+s2}{\PYZdq{}}\PY{p}{,} \PY{n}{ylabel}\PY{o}{=}\PY{l+s+s2}{\PYZdq{}}\PY{l+s+s2}{Tasa de paro (}\PY{l+s+s2}{\PYZpc{}}\PY{l+s+s2}{)}\PY{l+s+s2}{\PYZdq{}}\PY{p}{,} 
                           \PY{n}{title}\PY{o}{=}\PY{l+s+s2}{\PYZdq{}}\PY{l+s+s2}{Tasa de paro por edades}\PY{l+s+s2}{\PYZdq{}}\PY{p}{,} \PY{n}{xticks\PYZus{}rotation}\PY{o}{=}\PY{l+m+mi}{0}\PY{p}{,} \PY{n}{style}\PY{o}{=}\PY{l+s+s2}{\PYZdq{}}\PY{l+s+s2}{whitegrid}\PY{l+s+s2}{\PYZdq{}}\PY{p}{,} 
                           \PY{n}{figsize}\PY{o}{=}\PY{p}{(}\PY{l+m+mi}{12}\PY{p}{,} \PY{l+m+mi}{7}\PY{p}{)}\PY{p}{)}
\end{Verbatim}
\end{tcolorbox}

    \begin{center}
    \adjustimage{max size={0.9\linewidth}{0.9\paperheight}}{tfm_project_files/tfm_project_72_0.png}
    \end{center}
    { \hspace*{\fill} \\}
    
    \begin{tcolorbox}[breakable, size=fbox, boxrule=1pt, pad at break*=1mm,colback=cellbackground, colframe=cellborder]
\prompt{In}{incolor}{28}{\boxspacing}
\begin{Verbatim}[commandchars=\\\{\}]
\PY{n}{paro\PYZus{}sexo} \PY{o}{=} \PY{n}{paro}\PY{p}{[}\PY{p}{(}\PY{n}{paro}\PY{p}{[}\PY{l+s+s1}{\PYZsq{}}\PY{l+s+s1}{ccaa}\PY{l+s+s1}{\PYZsq{}}\PY{p}{]}\PY{o}{==}\PY{l+s+s2}{\PYZdq{}}\PY{l+s+s2}{Total Nacional}\PY{l+s+s2}{\PYZdq{}}\PY{p}{)} \PY{o}{\PYZam{}} 
                 \PY{p}{(}\PY{n}{paro}\PY{p}{[}\PY{l+s+s1}{\PYZsq{}}\PY{l+s+s1}{edad}\PY{l+s+s1}{\PYZsq{}}\PY{p}{]} \PY{o}{==} \PY{l+s+s2}{\PYZdq{}}\PY{l+s+s2}{Total}\PY{l+s+s2}{\PYZdq{}}\PY{p}{)} \PY{p}{]}
\PY{n}{p}\PY{o}{.}\PY{n}{create\PYZus{}multi\PYZus{}category\PYZus{}plot}\PY{p}{(}\PY{n}{data} \PY{o}{=} \PY{n}{paro\PYZus{}sexo}\PY{p}{,} \PY{n}{x\PYZus{}col}\PY{o}{=}\PY{l+s+s2}{\PYZdq{}}\PY{l+s+s2}{periodo}\PY{l+s+s2}{\PYZdq{}}\PY{p}{,} \PY{n}{y\PYZus{}col}\PY{o}{=}\PY{l+s+s2}{\PYZdq{}}\PY{l+s+s2}{tasa\PYZus{}paro\PYZus{}total}\PY{l+s+s2}{\PYZdq{}}\PY{p}{,} 
                           \PY{n}{label} \PY{o}{=} \PY{l+s+s2}{\PYZdq{}}\PY{l+s+s2}{Sexo}\PY{l+s+s2}{\PYZdq{}}\PY{p}{,}\PY{n}{category\PYZus{}col}\PY{o}{=}\PY{l+s+s2}{\PYZdq{}}\PY{l+s+s2}{sexo}\PY{l+s+s2}{\PYZdq{}}\PY{p}{,} \PY{n}{xlabel}\PY{o}{=}\PY{l+s+s2}{\PYZdq{}}\PY{l+s+s2}{Año}\PY{l+s+s2}{\PYZdq{}}\PY{p}{,} \PY{n}{ylabel}\PY{o}{=}\PY{l+s+s2}{\PYZdq{}}\PY{l+s+s2}{Tasa de paro (}\PY{l+s+s2}{\PYZpc{}}\PY{l+s+s2}{)}\PY{l+s+s2}{\PYZdq{}}\PY{p}{,} 
                           \PY{n}{title}\PY{o}{=}\PY{l+s+s2}{\PYZdq{}}\PY{l+s+s2}{Tasa de paro por sexo}\PY{l+s+s2}{\PYZdq{}}\PY{p}{,} \PY{n}{xticks\PYZus{}rotation}\PY{o}{=}\PY{l+m+mi}{0}\PY{p}{,} \PY{n}{style}\PY{o}{=}\PY{l+s+s2}{\PYZdq{}}\PY{l+s+s2}{whitegrid}\PY{l+s+s2}{\PYZdq{}}\PY{p}{,} 
                           \PY{n}{figsize}\PY{o}{=}\PY{p}{(}\PY{l+m+mi}{12}\PY{p}{,} \PY{l+m+mi}{7}\PY{p}{)}\PY{p}{)}
\end{Verbatim}
\end{tcolorbox}

    \begin{center}
    \adjustimage{max size={0.9\linewidth}{0.9\paperheight}}{tfm_project_files/tfm_project_73_0.png}
    \end{center}
    { \hspace*{\fill} \\}
    
    \begin{tcolorbox}[breakable, size=fbox, boxrule=1pt, pad at break*=1mm,colback=cellbackground, colframe=cellborder]
\prompt{In}{incolor}{29}{\boxspacing}
\begin{Verbatim}[commandchars=\\\{\}]
\PY{n}{paro}\PY{p}{[}\PY{l+s+s1}{\PYZsq{}}\PY{l+s+s1}{sexo\PYZhy{}edad}\PY{l+s+s1}{\PYZsq{}}\PY{p}{]} \PY{o}{=} \PY{n}{paro}\PY{p}{[}\PY{l+s+s1}{\PYZsq{}}\PY{l+s+s1}{sexo}\PY{l+s+s1}{\PYZsq{}}\PY{p}{]} \PY{o}{+} \PY{l+s+s2}{\PYZdq{}}\PY{l+s+s2}{\PYZhy{}}\PY{l+s+s2}{\PYZdq{}} \PY{o}{+} \PY{n}{paro}\PY{p}{[}\PY{l+s+s1}{\PYZsq{}}\PY{l+s+s1}{edad}\PY{l+s+s1}{\PYZsq{}}\PY{p}{]}
\PY{n}{paro\PYZus{}sexo\PYZus{}edad} \PY{o}{=} \PY{n}{paro}\PY{p}{[}\PY{p}{(}\PY{n}{paro}\PY{p}{[}\PY{l+s+s1}{\PYZsq{}}\PY{l+s+s1}{ccaa}\PY{l+s+s1}{\PYZsq{}}\PY{p}{]}\PY{o}{==}\PY{l+s+s2}{\PYZdq{}}\PY{l+s+s2}{Total Nacional}\PY{l+s+s2}{\PYZdq{}}\PY{p}{)} \PY{o}{\PYZam{}} 
                      \PY{p}{(}\PY{n}{paro}\PY{p}{[}\PY{l+s+s1}{\PYZsq{}}\PY{l+s+s1}{sexo}\PY{l+s+s1}{\PYZsq{}}\PY{p}{]}\PY{o}{.}\PY{n}{isin}\PY{p}{(}\PY{p}{[}\PY{l+s+s1}{\PYZsq{}}\PY{l+s+s1}{Hombres}\PY{l+s+s1}{\PYZsq{}}\PY{p}{,} \PY{l+s+s1}{\PYZsq{}}\PY{l+s+s1}{Mujeres}\PY{l+s+s1}{\PYZsq{}}\PY{p}{]}\PY{p}{)}\PY{p}{)} \PY{o}{\PYZam{}} 
                      \PY{p}{(}\PY{n}{paro}\PY{p}{[}\PY{l+s+s1}{\PYZsq{}}\PY{l+s+s1}{edad}\PY{l+s+s1}{\PYZsq{}}\PY{p}{]}\PY{o}{.}\PY{n}{isin}\PY{p}{(}\PY{p}{[}\PY{l+s+s1}{\PYZsq{}}\PY{l+s+s1}{De 16 a 19 años}\PY{l+s+s1}{\PYZsq{}}\PY{p}{,} \PY{l+s+s1}{\PYZsq{}}\PY{l+s+s1}{De 20 a 24 años}\PY{l+s+s1}{\PYZsq{}}\PY{p}{,} \PY{l+s+s1}{\PYZsq{}}\PY{l+s+s1}{De 25 a 54 años}\PY{l+s+s1}{\PYZsq{}}\PY{p}{,} \PY{l+s+s1}{\PYZsq{}}\PY{l+s+s1}{55 y más años}\PY{l+s+s1}{\PYZsq{}}\PY{p}{]}\PY{p}{)}\PY{p}{)} \PY{p}{]}
\PY{n}{p}\PY{o}{.}\PY{n}{create\PYZus{}multi\PYZus{}category\PYZus{}plot}\PY{p}{(}\PY{n}{data} \PY{o}{=} \PY{n}{paro\PYZus{}sexo\PYZus{}edad}\PY{p}{,} \PY{n}{x\PYZus{}col}\PY{o}{=}\PY{l+s+s2}{\PYZdq{}}\PY{l+s+s2}{periodo}\PY{l+s+s2}{\PYZdq{}}\PY{p}{,} 
                            \PY{n}{y\PYZus{}col}\PY{o}{=}\PY{l+s+s2}{\PYZdq{}}\PY{l+s+s2}{tasa\PYZus{}paro\PYZus{}total}\PY{l+s+s2}{\PYZdq{}}\PY{p}{,} \PY{n}{label} \PY{o}{=} \PY{l+s+s2}{\PYZdq{}}\PY{l+s+s2}{Sexo y Edad}\PY{l+s+s2}{\PYZdq{}}\PY{p}{,}
                           \PY{n}{category\PYZus{}col}\PY{o}{=}\PY{l+s+s2}{\PYZdq{}}\PY{l+s+s2}{sexo\PYZhy{}edad}\PY{l+s+s2}{\PYZdq{}}\PY{p}{,} \PY{n}{xlabel}\PY{o}{=}\PY{l+s+s2}{\PYZdq{}}\PY{l+s+s2}{Año}\PY{l+s+s2}{\PYZdq{}}\PY{p}{,} 
                           \PY{n}{ylabel}\PY{o}{=}\PY{l+s+s2}{\PYZdq{}}\PY{l+s+s2}{Tasa de paro (}\PY{l+s+s2}{\PYZpc{}}\PY{l+s+s2}{)}\PY{l+s+s2}{\PYZdq{}}\PY{p}{,} \PY{n}{title}\PY{o}{=}\PY{l+s+s2}{\PYZdq{}}\PY{l+s+s2}{Tasa de paro por sexo y edad}\PY{l+s+s2}{\PYZdq{}}\PY{p}{,} 
                           \PY{n}{xticks\PYZus{}rotation}\PY{o}{=}\PY{l+m+mi}{0}\PY{p}{,} \PY{n}{style}\PY{o}{=}\PY{l+s+s2}{\PYZdq{}}\PY{l+s+s2}{whitegrid}\PY{l+s+s2}{\PYZdq{}}\PY{p}{,} \PY{n}{figsize}\PY{o}{=}\PY{p}{(}\PY{l+m+mi}{12}\PY{p}{,} \PY{l+m+mi}{7}\PY{p}{)}\PY{p}{)}
\end{Verbatim}
\end{tcolorbox}

    \begin{center}
    \adjustimage{max size={0.9\linewidth}{0.9\paperheight}}{tfm_project_files/tfm_project_74_0.png}
    \end{center}
    { \hspace*{\fill} \\}
    
    Parece claro que los jóvenes son los más afectados por el desempleo,
sufriendo los incrementos más grandes a lo largo del tiempo, y
alcanzando los valores más altos. A priori este debería ser de los
posibles grupos afectados por el salario mínimo, pues un incremento en
el mismo podría provocar una mayor dificultad de inserción en el mercado
laboral.

En lo que respesta a sexo, parece que el grupo más afectado son las
mujeres dentro de todas las edades.

    Otro factor relevante a tener en cuenta en el análisis es qué porcentaje
de parados son de larga duración, algo que puede pasar desapercibido en
un análisis superficial y que puede ser un indicativo de efectos a más
largo plazo como la dificultad de reinserción.

    \begin{tcolorbox}[breakable, size=fbox, boxrule=1pt, pad at break*=1mm,colback=cellbackground, colframe=cellborder]
\prompt{In}{incolor}{30}{\boxspacing}
\begin{Verbatim}[commandchars=\\\{\}]
\PY{n}{paro\PYZus{}duracion} \PY{o}{=} \PY{n}{pd}\PY{o}{.}\PY{n}{read\PYZus{}csv}\PY{p}{(}\PY{l+s+s2}{\PYZdq{}}\PY{l+s+s2}{../../processed\PYZus{}data/paro/parados\PYZus{}tiempo.csv}\PY{l+s+s2}{\PYZdq{}}\PY{p}{)}
\PY{n}{paro\PYZus{}tiempo\PYZus{}busqueda} \PY{o}{=} \PY{n}{paro\PYZus{}duracion}\PY{p}{[}\PY{p}{(}\PY{n}{paro\PYZus{}duracion}\PY{p}{[}\PY{l+s+s1}{\PYZsq{}}\PY{l+s+s1}{ccaa}\PY{l+s+s1}{\PYZsq{}}\PY{p}{]}\PY{o}{==}\PY{l+s+s2}{\PYZdq{}}\PY{l+s+s2}{Total Nacional}\PY{l+s+s2}{\PYZdq{}}\PY{p}{)} \PY{o}{\PYZam{}} 
                                     \PY{p}{(}\PY{n}{paro\PYZus{}duracion}\PY{p}{[}\PY{l+s+s1}{\PYZsq{}}\PY{l+s+s1}{sexo}\PY{l+s+s1}{\PYZsq{}}\PY{p}{]} \PY{o}{==} \PY{l+s+s2}{\PYZdq{}}\PY{l+s+s2}{Ambos sexos}\PY{l+s+s2}{\PYZdq{}}\PY{p}{)} \PY{o}{\PYZam{}} 
                                     \PY{p}{(}\PY{n}{paro\PYZus{}duracion}\PY{p}{[}\PY{l+s+s1}{\PYZsq{}}\PY{l+s+s1}{tiempo\PYZus{}busqueda}\PY{l+s+s1}{\PYZsq{}}\PY{p}{]}\PY{o}{!=}\PY{l+s+s2}{\PYZdq{}}\PY{l+s+s2}{Total}\PY{l+s+s2}{\PYZdq{}}\PY{p}{)}\PY{p}{]}
\PY{n}{p}\PY{o}{.}\PY{n}{create\PYZus{}multi\PYZus{}category\PYZus{}plot}\PY{p}{(}\PY{n}{data} \PY{o}{=} \PY{n}{paro\PYZus{}tiempo\PYZus{}busqueda}\PY{p}{,} \PY{n}{x\PYZus{}col}\PY{o}{=}\PY{l+s+s2}{\PYZdq{}}\PY{l+s+s2}{periodo}\PY{l+s+s2}{\PYZdq{}}\PY{p}{,} \PY{n}{y\PYZus{}col}\PY{o}{=}\PY{l+s+s2}{\PYZdq{}}\PY{l+s+s2}{porcentaje\PYZus{}tipo\PYZus{}paro}\PY{l+s+s2}{\PYZdq{}}\PY{p}{,} 
                           \PY{n}{label} \PY{o}{=} \PY{l+s+s2}{\PYZdq{}}\PY{l+s+s2}{Tiempo de búsqueda}\PY{l+s+s2}{\PYZdq{}}\PY{p}{,}\PY{n}{category\PYZus{}col}\PY{o}{=}\PY{l+s+s2}{\PYZdq{}}\PY{l+s+s2}{tiempo\PYZus{}busqueda}\PY{l+s+s2}{\PYZdq{}}\PY{p}{,} \PY{n}{xlabel}\PY{o}{=}\PY{l+s+s2}{\PYZdq{}}\PY{l+s+s2}{Año}\PY{l+s+s2}{\PYZdq{}}\PY{p}{,} 
                           \PY{n}{ylabel}\PY{o}{=}\PY{l+s+s2}{\PYZdq{}}\PY{l+s+s2}{Porcentaje de parados (}\PY{l+s+s2}{\PYZpc{}}\PY{l+s+s2}{)}\PY{l+s+s2}{\PYZdq{}}\PY{p}{,} \PY{n}{title}\PY{o}{=}\PY{l+s+s2}{\PYZdq{}}\PY{l+s+s2}{Parados según su tiempo de búsqueda}\PY{l+s+s2}{\PYZdq{}}\PY{p}{,} 
                           \PY{n}{xticks\PYZus{}rotation}\PY{o}{=}\PY{l+m+mi}{0}\PY{p}{,} \PY{n}{style}\PY{o}{=}\PY{l+s+s2}{\PYZdq{}}\PY{l+s+s2}{whitegrid}\PY{l+s+s2}{\PYZdq{}}\PY{p}{,} \PY{n}{figsize}\PY{o}{=}\PY{p}{(}\PY{l+m+mi}{12}\PY{p}{,} \PY{l+m+mi}{7}\PY{p}{)}\PY{p}{)}
\end{Verbatim}
\end{tcolorbox}

    \begin{center}
    \adjustimage{max size={0.9\linewidth}{0.9\paperheight}}{tfm_project_files/tfm_project_77_0.png}
    \end{center}
    { \hspace*{\fill} \\}
    
    \begin{tcolorbox}[breakable, size=fbox, boxrule=1pt, pad at break*=1mm,colback=cellbackground, colframe=cellborder]
\prompt{In}{incolor}{31}{\boxspacing}
\begin{Verbatim}[commandchars=\\\{\}]
\PY{n}{paro\PYZus{}tiempo\PYZus{}largo\PYZus{}ccaa}\PY{o}{=} \PY{n}{paro\PYZus{}duracion}\PY{p}{[}\PY{p}{(}\PY{n}{paro\PYZus{}duracion}\PY{p}{[}\PY{l+s+s1}{\PYZsq{}}\PY{l+s+s1}{ccaa}\PY{l+s+s1}{\PYZsq{}}\PY{p}{]}\PY{o}{.}\PY{n}{isin}\PY{p}{(}\PY{n}{CCAA\PYZus{}mostrar}\PY{p}{)}\PY{p}{)} \PY{o}{\PYZam{}} 
                                      \PY{p}{(}\PY{n}{paro\PYZus{}duracion}\PY{p}{[}\PY{l+s+s1}{\PYZsq{}}\PY{l+s+s1}{sexo}\PY{l+s+s1}{\PYZsq{}}\PY{p}{]} \PY{o}{==} \PY{l+s+s2}{\PYZdq{}}\PY{l+s+s2}{Ambos sexos}\PY{l+s+s2}{\PYZdq{}}\PY{p}{)} 
                                      \PY{o}{\PYZam{}} \PY{p}{(}\PY{n}{paro\PYZus{}duracion}\PY{p}{[}\PY{l+s+s1}{\PYZsq{}}\PY{l+s+s1}{tiempo\PYZus{}busqueda}\PY{l+s+s1}{\PYZsq{}}\PY{p}{]}\PY{o}{==}\PY{l+s+s2}{\PYZdq{}}\PY{l+s+s2}{2 años o más}\PY{l+s+s2}{\PYZdq{}}\PY{p}{)}\PY{p}{]}
\PY{n}{p}\PY{o}{.}\PY{n}{create\PYZus{}multi\PYZus{}category\PYZus{}plot}\PY{p}{(}\PY{n}{data} \PY{o}{=} \PY{n}{paro\PYZus{}tiempo\PYZus{}largo\PYZus{}ccaa}\PY{p}{,} \PY{n}{x\PYZus{}col}\PY{o}{=}\PY{l+s+s2}{\PYZdq{}}\PY{l+s+s2}{periodo}\PY{l+s+s2}{\PYZdq{}}\PY{p}{,} \PY{n}{y\PYZus{}col}\PY{o}{=}\PY{l+s+s2}{\PYZdq{}}\PY{l+s+s2}{porcentaje\PYZus{}tipo\PYZus{}paro}\PY{l+s+s2}{\PYZdq{}}\PY{p}{,} \PY{n}{label} \PY{o}{=} \PY{l+s+s2}{\PYZdq{}}\PY{l+s+s2}{CCAA}\PY{l+s+s2}{\PYZdq{}}\PY{p}{,}\PY{n}{category\PYZus{}col}\PY{o}{=}\PY{l+s+s2}{\PYZdq{}}\PY{l+s+s2}{ccaa}\PY{l+s+s2}{\PYZdq{}}\PY{p}{,}
                            \PY{n}{xlabel}\PY{o}{=}\PY{l+s+s2}{\PYZdq{}}\PY{l+s+s2}{Año}\PY{l+s+s2}{\PYZdq{}}\PY{p}{,} \PY{n}{ylabel}\PY{o}{=}\PY{l+s+s2}{\PYZdq{}}\PY{l+s+s2}{Porcentaje de parados (}\PY{l+s+s2}{\PYZpc{}}\PY{l+s+s2}{)}\PY{l+s+s2}{\PYZdq{}}\PY{p}{,} 
                           \PY{n}{title}\PY{o}{=}\PY{l+s+s2}{\PYZdq{}}\PY{l+s+s2}{Parados de larga duración por comunidad autónoma}\PY{l+s+s2}{\PYZdq{}}\PY{p}{,} \PY{n}{xticks\PYZus{}rotation}\PY{o}{=}\PY{l+m+mi}{0}\PY{p}{,} \PY{n}{style}\PY{o}{=}\PY{l+s+s2}{\PYZdq{}}\PY{l+s+s2}{whitegrid}\PY{l+s+s2}{\PYZdq{}}\PY{p}{,} 
                           \PY{n}{figsize}\PY{o}{=}\PY{p}{(}\PY{l+m+mi}{12}\PY{p}{,} \PY{l+m+mi}{7}\PY{p}{)}\PY{p}{)}
\end{Verbatim}
\end{tcolorbox}

    \begin{center}
    \adjustimage{max size={0.9\linewidth}{0.9\paperheight}}{tfm_project_files/tfm_project_78_0.png}
    \end{center}
    { \hspace*{\fill} \\}
    
    Al igual que con todas las variables previamente observadas, vemos que
la evolución por comunidad autónoma sigue la misma tendencia, aunque las
diferencias entre cada una no se pueden pasar por alto.

    \subsubsection{Productividad}\label{productividad}

    La productividad es un factor importante al estudiar el salario mínimo.
Una alta productividad da más margen para incrementos salariales,
mientras que una baja productividad puede fomentar una mayor cantidad de
despidos.

El dataset escogido de productividad, obtenido del Observatorio de
Productividad y Competitividad de España (OPCE) contiene además
información útil sobre qué componentes son los que contribuyen a la
productividad

    \begin{tcolorbox}[breakable, size=fbox, boxrule=1pt, pad at break*=1mm,colback=cellbackground, colframe=cellborder]
\prompt{In}{incolor}{32}{\boxspacing}
\begin{Verbatim}[commandchars=\\\{\}]
\PY{n}{productividad} \PY{o}{=} \PY{n}{pd}\PY{o}{.}\PY{n}{read\PYZus{}csv}\PY{p}{(}\PY{l+s+s2}{\PYZdq{}}\PY{l+s+s2}{../../processed\PYZus{}data/productividad/productividad\PYZus{}ccaa.csv}\PY{l+s+s2}{\PYZdq{}}\PY{p}{)}
\PY{n}{productividad} \PY{o}{=} \PY{n}{productividad}\PY{p}{[}\PY{n}{productividad}\PY{o}{.}\PY{n}{periodo}\PY{o}{\PYZgt{}}\PY{o}{=}\PY{l+m+mi}{2008}\PY{p}{]}
\PY{n}{productividad\PYZus{}hora} \PY{o}{=} \PY{n}{productividad}\PY{p}{[}\PY{p}{(}\PY{n}{productividad}\PY{p}{[}\PY{l+s+s1}{\PYZsq{}}\PY{l+s+s1}{variable}\PY{l+s+s1}{\PYZsq{}}\PY{p}{]} \PY{o}{==} \PY{l+s+s1}{\PYZsq{}}\PY{l+s+s1}{Productividad del trabajo por hora trabajada}\PY{l+s+s1}{\PYZsq{}}\PY{p}{)} \PY{o}{\PYZam{}} 
                                   \PY{p}{(}\PY{n}{productividad}\PY{o}{.}\PY{n}{unidad} \PY{o}{==} \PY{l+s+s2}{\PYZdq{}}\PY{l+s+s2}{Euros de 2015 por hora trabajada}\PY{l+s+s2}{\PYZdq{}}\PY{p}{)}\PY{p}{]}
\PY{n}{productividad\PYZus{}hora\PYZus{}ccaa} \PY{o}{=} \PY{n}{productividad\PYZus{}hora}\PY{p}{[}\PY{n}{productividad\PYZus{}hora}\PY{p}{[}\PY{l+s+s1}{\PYZsq{}}\PY{l+s+s1}{ccaa}\PY{l+s+s1}{\PYZsq{}}\PY{p}{]}\PY{o}{.}\PY{n}{isin}\PY{p}{(}\PY{n}{CCAA\PYZus{}mostrar}\PY{p}{)}\PY{p}{]}
\PY{n}{p}\PY{o}{.}\PY{n}{create\PYZus{}multi\PYZus{}category\PYZus{}plot}\PY{p}{(}\PY{n}{data} \PY{o}{=} \PY{n}{productividad\PYZus{}hora\PYZus{}ccaa}\PY{p}{,} \PY{n}{x\PYZus{}col}\PY{o}{=}\PY{l+s+s2}{\PYZdq{}}\PY{l+s+s2}{periodo}\PY{l+s+s2}{\PYZdq{}}\PY{p}{,} \PY{n}{y\PYZus{}col}\PY{o}{=}\PY{l+s+s2}{\PYZdq{}}\PY{l+s+s2}{total}\PY{l+s+s2}{\PYZdq{}}\PY{p}{,} \PY{n}{label} \PY{o}{=} \PY{l+s+s2}{\PYZdq{}}\PY{l+s+s2}{CCAA}\PY{l+s+s2}{\PYZdq{}}
                           \PY{p}{,}\PY{n}{category\PYZus{}col}\PY{o}{=}\PY{l+s+s2}{\PYZdq{}}\PY{l+s+s2}{ccaa}\PY{l+s+s2}{\PYZdq{}}\PY{p}{,} \PY{n}{xlabel}\PY{o}{=}\PY{l+s+s2}{\PYZdq{}}\PY{l+s+s2}{Año}\PY{l+s+s2}{\PYZdq{}}\PY{p}{,} \PY{n}{ylabel}\PY{o}{=}\PY{l+s+s2}{\PYZdq{}}\PY{l+s+s2}{Productividad (€)}\PY{l+s+s2}{\PYZdq{}}\PY{p}{,} 
                           \PY{n}{title}\PY{o}{=}\PY{l+s+s2}{\PYZdq{}}\PY{l+s+s2}{Productividad del trabajo por hora (€ de 2015)}\PY{l+s+s2}{\PYZdq{}}\PY{p}{,} \PY{n}{xticks\PYZus{}rotation}\PY{o}{=}\PY{l+m+mi}{0}\PY{p}{,}
                           \PY{n}{style}\PY{o}{=}\PY{l+s+s2}{\PYZdq{}}\PY{l+s+s2}{whitegrid}\PY{l+s+s2}{\PYZdq{}}\PY{p}{,} \PY{n}{figsize}\PY{o}{=}\PY{p}{(}\PY{l+m+mi}{12}\PY{p}{,} \PY{l+m+mi}{7}\PY{p}{)}\PY{p}{)}
\end{Verbatim}
\end{tcolorbox}

    \begin{center}
    \adjustimage{max size={0.9\linewidth}{0.9\paperheight}}{tfm_project_files/tfm_project_82_0.png}
    \end{center}
    { \hspace*{\fill} \\}
    
    \begin{tcolorbox}[breakable, size=fbox, boxrule=1pt, pad at break*=1mm,colback=cellbackground, colframe=cellborder]
\prompt{In}{incolor}{33}{\boxspacing}
\begin{Verbatim}[commandchars=\\\{\}]
\PY{n}{productividad\PYZus{}ocupado} \PY{o}{=} \PY{n}{productividad}\PY{p}{[}\PY{p}{(}\PY{n}{productividad}\PY{p}{[}\PY{l+s+s1}{\PYZsq{}}\PY{l+s+s1}{variable}\PY{l+s+s1}{\PYZsq{}}\PY{p}{]} \PY{o}{==} \PY{l+s+s1}{\PYZsq{}}\PY{l+s+s1}{Productividad del trabajo por ocupado}\PY{l+s+s1}{\PYZsq{}}\PY{p}{)} \PY{o}{\PYZam{}} 
                                      \PY{p}{(}\PY{n}{productividad}\PY{o}{.}\PY{n}{unidad} \PY{o}{==} \PY{l+s+s2}{\PYZdq{}}\PY{l+s+s2}{Euros de 2015 por persona ocupada}\PY{l+s+s2}{\PYZdq{}}\PY{p}{)}\PY{p}{]}
\PY{n}{productividad\PYZus{}ocupado\PYZus{}ccaa} \PY{o}{=} \PY{n}{productividad\PYZus{}ocupado}\PY{p}{[}\PY{n}{productividad\PYZus{}ocupado}\PY{p}{[}\PY{l+s+s1}{\PYZsq{}}\PY{l+s+s1}{ccaa}\PY{l+s+s1}{\PYZsq{}}\PY{p}{]}\PY{o}{.}\PY{n}{isin}\PY{p}{(}\PY{n}{CCAA\PYZus{}mostrar}\PY{p}{)}\PY{p}{]}
\PY{n}{p}\PY{o}{.}\PY{n}{create\PYZus{}multi\PYZus{}category\PYZus{}plot}\PY{p}{(}\PY{n}{data} \PY{o}{=} \PY{n}{productividad\PYZus{}ocupado\PYZus{}ccaa}\PY{p}{,} \PY{n}{x\PYZus{}col}\PY{o}{=}\PY{l+s+s2}{\PYZdq{}}\PY{l+s+s2}{periodo}\PY{l+s+s2}{\PYZdq{}}\PY{p}{,} \PY{n}{y\PYZus{}col}\PY{o}{=}\PY{l+s+s2}{\PYZdq{}}\PY{l+s+s2}{total}\PY{l+s+s2}{\PYZdq{}}\PY{p}{,} \PY{n}{label} \PY{o}{=} \PY{l+s+s2}{\PYZdq{}}\PY{l+s+s2}{CCAA}\PY{l+s+s2}{\PYZdq{}}\PY{p}{,}\PY{n}{category\PYZus{}col}\PY{o}{=}\PY{l+s+s2}{\PYZdq{}}\PY{l+s+s2}{ccaa}\PY{l+s+s2}{\PYZdq{}}\PY{p}{,} 
                           \PY{n}{xlabel}\PY{o}{=}\PY{l+s+s2}{\PYZdq{}}\PY{l+s+s2}{Año}\PY{l+s+s2}{\PYZdq{}}\PY{p}{,} \PY{n}{ylabel}\PY{o}{=}\PY{l+s+s2}{\PYZdq{}}\PY{l+s+s2}{Productividad (€)}\PY{l+s+s2}{\PYZdq{}}\PY{p}{,} 
                           \PY{n}{title}\PY{o}{=}\PY{l+s+s2}{\PYZdq{}}\PY{l+s+s2}{Productividad por ocupado (€ de 2015)}\PY{l+s+s2}{\PYZdq{}}\PY{p}{,} 
                           \PY{n}{xticks\PYZus{}rotation}\PY{o}{=}\PY{l+m+mi}{0}\PY{p}{,} \PY{n}{style}\PY{o}{=}\PY{l+s+s2}{\PYZdq{}}\PY{l+s+s2}{whitegrid}\PY{l+s+s2}{\PYZdq{}}\PY{p}{,} \PY{n}{figsize}\PY{o}{=}\PY{p}{(}\PY{l+m+mi}{12}\PY{p}{,} \PY{l+m+mi}{7}\PY{p}{)}\PY{p}{)}
\end{Verbatim}
\end{tcolorbox}

    \begin{center}
    \adjustimage{max size={0.9\linewidth}{0.9\paperheight}}{tfm_project_files/tfm_project_83_0.png}
    \end{center}
    { \hspace*{\fill} \\}
    
    \begin{tcolorbox}[breakable, size=fbox, boxrule=1pt, pad at break*=1mm,colback=cellbackground, colframe=cellborder]
\prompt{In}{incolor}{34}{\boxspacing}
\begin{Verbatim}[commandchars=\\\{\}]
\PY{c+c1}{\PYZsh{}Obtenemos el promedio de horas por empleo para ver la evolución}
\PY{n}{empleo} \PY{o}{=} \PY{p}{(}\PY{n}{productividad}\PY{p}{[}\PY{p}{(}\PY{n}{productividad}\PY{o}{.}\PY{n}{variable} \PY{o}{==} \PY{l+s+s2}{\PYZdq{}}\PY{l+s+s2}{Empleo total}\PY{l+s+s2}{\PYZdq{}}\PY{p}{)} \PY{o}{\PYZam{}} 
                       \PY{p}{(}\PY{n}{productividad}\PY{o}{.}\PY{n}{unidad} \PY{o}{==} \PY{l+s+s2}{\PYZdq{}}\PY{l+s+s2}{Miles de personas}\PY{l+s+s2}{\PYZdq{}}\PY{p}{)}\PY{p}{]}
                       \PY{o}{.}\PY{n}{drop}\PY{p}{(}\PY{p}{[}\PY{l+s+s1}{\PYZsq{}}\PY{l+s+s1}{variable}\PY{l+s+s1}{\PYZsq{}}\PY{p}{,} \PY{l+s+s1}{\PYZsq{}}\PY{l+s+s1}{unidad}\PY{l+s+s1}{\PYZsq{}}\PY{p}{]}\PY{p}{,} \PY{n}{axis}\PY{o}{=}\PY{l+m+mi}{1}\PY{p}{)}
                       \PY{o}{.}\PY{n}{rename}\PY{p}{(}\PY{n}{columns} \PY{o}{=} \PY{p}{\PYZob{}}\PY{l+s+s1}{\PYZsq{}}\PY{l+s+s1}{total}\PY{l+s+s1}{\PYZsq{}}\PY{p}{:} \PY{l+s+s1}{\PYZsq{}}\PY{l+s+s1}{empleo}\PY{l+s+s1}{\PYZsq{}}\PY{p}{\PYZcb{}}\PY{p}{)}\PY{p}{)}
\PY{n}{horas\PYZus{}trabajadas} \PY{o}{=} \PY{p}{(}\PY{n}{productividad}\PY{p}{[}\PY{p}{(}\PY{n}{productividad}\PY{o}{.}\PY{n}{variable} \PY{o}{==} \PY{l+s+s2}{\PYZdq{}}\PY{l+s+s2}{Horas trabajadas totales}\PY{l+s+s2}{\PYZdq{}}\PY{p}{)} \PY{o}{\PYZam{}} 
                                 \PY{p}{(}\PY{n}{productividad}\PY{o}{.}\PY{n}{unidad} \PY{o}{==} \PY{l+s+s2}{\PYZdq{}}\PY{l+s+s2}{Millones de horas}\PY{l+s+s2}{\PYZdq{}}\PY{p}{)}\PY{p}{]}
                                 \PY{o}{.}\PY{n}{drop}\PY{p}{(}\PY{p}{[}\PY{l+s+s1}{\PYZsq{}}\PY{l+s+s1}{variable}\PY{l+s+s1}{\PYZsq{}}\PY{p}{,} \PY{l+s+s1}{\PYZsq{}}\PY{l+s+s1}{unidad}\PY{l+s+s1}{\PYZsq{}}\PY{p}{]}\PY{p}{,} \PY{n}{axis}\PY{o}{=}\PY{l+m+mi}{1}\PY{p}{)}
                                 \PY{o}{.}\PY{n}{rename}\PY{p}{(}\PY{n}{columns} \PY{o}{=} \PY{p}{\PYZob{}}\PY{l+s+s1}{\PYZsq{}}\PY{l+s+s1}{total}\PY{l+s+s1}{\PYZsq{}}\PY{p}{:} \PY{l+s+s1}{\PYZsq{}}\PY{l+s+s1}{horas\PYZus{}trabajadas}\PY{l+s+s1}{\PYZsq{}}\PY{p}{\PYZcb{}}\PY{p}{)}\PY{p}{)}

\PY{n}{empleo\PYZus{}hora} \PY{o}{=} \PY{n}{pd}\PY{o}{.}\PY{n}{merge}\PY{p}{(}\PY{n}{empleo}\PY{p}{,} \PY{n}{horas\PYZus{}trabajadas}\PY{p}{,} \PY{n}{on}\PY{o}{=}\PY{p}{[}\PY{l+s+s1}{\PYZsq{}}\PY{l+s+s1}{periodo}\PY{l+s+s1}{\PYZsq{}}\PY{p}{,} \PY{l+s+s1}{\PYZsq{}}\PY{l+s+s1}{ccaa}\PY{l+s+s1}{\PYZsq{}}\PY{p}{]}\PY{p}{)}
\PY{n}{empleo\PYZus{}hora}\PY{p}{[}\PY{l+s+s1}{\PYZsq{}}\PY{l+s+s1}{empleo\PYZus{}hora}\PY{l+s+s1}{\PYZsq{}}\PY{p}{]} \PY{o}{=} \PY{n}{empleo\PYZus{}hora}\PY{o}{.}\PY{n}{horas\PYZus{}trabajadas}\PY{o}{/}\PY{n}{empleo\PYZus{}hora}\PY{o}{.}\PY{n}{empleo}
\PY{n}{p}\PY{o}{.}\PY{n}{create\PYZus{}multi\PYZus{}category\PYZus{}plot}\PY{p}{(}\PY{n}{data} \PY{o}{=} \PY{n}{empleo\PYZus{}hora}\PY{p}{[}\PY{n}{empleo\PYZus{}hora}\PY{p}{[}\PY{l+s+s1}{\PYZsq{}}\PY{l+s+s1}{ccaa}\PY{l+s+s1}{\PYZsq{}}\PY{p}{]}\PY{o}{.}\PY{n}{isin}\PY{p}{(}\PY{n}{CCAA\PYZus{}mostrar}\PY{p}{)}\PY{p}{]}\PY{p}{,} 
                             \PY{n}{x\PYZus{}col}\PY{o}{=}\PY{l+s+s2}{\PYZdq{}}\PY{l+s+s2}{periodo}\PY{l+s+s2}{\PYZdq{}}\PY{p}{,} \PY{n}{y\PYZus{}col}\PY{o}{=}\PY{l+s+s2}{\PYZdq{}}\PY{l+s+s2}{empleo\PYZus{}hora}\PY{l+s+s2}{\PYZdq{}}\PY{p}{,} \PY{n}{label} \PY{o}{=} \PY{l+s+s2}{\PYZdq{}}\PY{l+s+s2}{CCAA}\PY{l+s+s2}{\PYZdq{}}\PY{p}{,}\PY{n}{category\PYZus{}col}\PY{o}{=}\PY{l+s+s2}{\PYZdq{}}\PY{l+s+s2}{ccaa}\PY{l+s+s2}{\PYZdq{}}\PY{p}{,} 
                           \PY{n}{xlabel}\PY{o}{=}\PY{l+s+s2}{\PYZdq{}}\PY{l+s+s2}{Año}\PY{l+s+s2}{\PYZdq{}}\PY{p}{,} \PY{n}{ylabel}\PY{o}{=}\PY{l+s+s2}{\PYZdq{}}\PY{l+s+s2}{Miles de horas}\PY{l+s+s2}{\PYZdq{}}\PY{p}{,} \PY{n}{title}\PY{o}{=}\PY{l+s+s2}{\PYZdq{}}\PY{l+s+s2}{Horas trabajadas por empleo}\PY{l+s+s2}{\PYZdq{}}\PY{p}{,} 
                           \PY{n}{xticks\PYZus{}rotation}\PY{o}{=}\PY{l+m+mi}{0}\PY{p}{,} \PY{n}{style}\PY{o}{=}\PY{l+s+s2}{\PYZdq{}}\PY{l+s+s2}{whitegrid}\PY{l+s+s2}{\PYZdq{}}\PY{p}{,} \PY{n}{figsize}\PY{o}{=}\PY{p}{(}\PY{l+m+mi}{12}\PY{p}{,} \PY{l+m+mi}{7}\PY{p}{)}\PY{p}{)}
\end{Verbatim}
\end{tcolorbox}

    \begin{center}
    \adjustimage{max size={0.9\linewidth}{0.9\paperheight}}{tfm_project_files/tfm_project_84_0.png}
    \end{center}
    { \hspace*{\fill} \\}
    
    De estas gráficas se puede extraer la conclusión de que comunidades
autónomas como Canarias tienen una mayor proporción de empleo poco
productivo, pues pese a trabajar de media más horas que un vasco la
productividad por hora trabajada a partir de 2015 acaba siendo menor.

    \subsubsection{Empresas}\label{empresas}

    Las empresas y tamaño nos pueden informar de qué tan resistente puede
ser una economía dada a un incremento del salario mínimo, pues empresas
pequeñas con menos recursos tendrán menos capacidad de hacer frente a
los gastos necesarios y consecuentemente podrán optar más por el despido
o por cerrar.

    \begin{tcolorbox}[breakable, size=fbox, boxrule=1pt, pad at break*=1mm,colback=cellbackground, colframe=cellborder]
\prompt{In}{incolor}{35}{\boxspacing}
\begin{Verbatim}[commandchars=\\\{\}]
\PY{n}{empresas} \PY{o}{=} \PY{n}{pd}\PY{o}{.}\PY{n}{read\PYZus{}csv}\PY{p}{(}\PY{l+s+s2}{\PYZdq{}}\PY{l+s+s2}{../../processed\PYZus{}data/empresas/empresas.csv}\PY{l+s+s2}{\PYZdq{}}\PY{p}{)}
\PY{n}{empresas\PYZus{}todo} \PY{o}{=} \PY{p}{(}\PY{n}{empresas}\PY{p}{[}\PY{p}{(}\PY{n}{empresas}\PY{o}{.}\PY{n}{ccaa} \PY{o}{==} \PY{l+s+s2}{\PYZdq{}}\PY{l+s+s2}{Total Nacional}\PY{l+s+s2}{\PYZdq{}}\PY{p}{)} \PY{o}{\PYZam{}} 
                         \PY{p}{(}\PY{n}{empresas}\PY{o}{.}\PY{n}{actividad\PYZus{}principal} \PY{o}{==} \PY{l+s+s2}{\PYZdq{}}\PY{l+s+s2}{Total CNAE}\PY{l+s+s2}{\PYZdq{}}\PY{p}{)}
                           \PY{o}{\PYZam{}} \PY{p}{(}\PY{n}{empresas}\PY{o}{.}\PY{n}{estrato\PYZus{}asalariados} \PY{o}{!=} \PY{l+s+s2}{\PYZdq{}}\PY{l+s+s2}{Total}\PY{l+s+s2}{\PYZdq{}}\PY{p}{)}\PY{p}{]}
                           \PY{o}{.}\PY{n}{groupby}\PY{p}{(}\PY{p}{[}\PY{l+s+s1}{\PYZsq{}}\PY{l+s+s1}{periodo}\PY{l+s+s1}{\PYZsq{}}\PY{p}{,} \PY{l+s+s1}{\PYZsq{}}\PY{l+s+s1}{estrato\PYZus{}asalariados\PYZus{}grupo}\PY{l+s+s1}{\PYZsq{}}\PY{p}{]}\PY{p}{,} 
                                    \PY{n}{as\PYZus{}index}\PY{o}{=}\PY{k+kc}{False}\PY{p}{)}\PY{o}{.}\PY{n}{agg}\PY{p}{(}\PY{p}{\PYZob{}}\PY{l+s+s1}{\PYZsq{}}\PY{l+s+s1}{total\PYZus{}empresas}\PY{l+s+s1}{\PYZsq{}}\PY{p}{:} \PY{l+s+s1}{\PYZsq{}}\PY{l+s+s1}{sum}\PY{l+s+s1}{\PYZsq{}}\PY{p}{\PYZcb{}}\PY{p}{)}\PY{p}{)}
\PY{n}{p}\PY{o}{.}\PY{n}{create\PYZus{}multi\PYZus{}category\PYZus{}plot}\PY{p}{(}\PY{n}{data} \PY{o}{=} \PY{n}{empresas\PYZus{}todo}\PY{p}{,} \PY{n}{x\PYZus{}col}\PY{o}{=}\PY{l+s+s2}{\PYZdq{}}\PY{l+s+s2}{periodo}\PY{l+s+s2}{\PYZdq{}}\PY{p}{,} 
                             \PY{n}{y\PYZus{}col}\PY{o}{=}\PY{l+s+s2}{\PYZdq{}}\PY{l+s+s2}{total\PYZus{}empresas}\PY{l+s+s2}{\PYZdq{}}\PY{p}{,} 
                             \PY{n}{label} \PY{o}{=} \PY{l+s+s2}{\PYZdq{}}\PY{l+s+s2}{Estrato de Asalariados}\PY{l+s+s2}{\PYZdq{}}\PY{p}{,}
                           \PY{n}{category\PYZus{}col}\PY{o}{=}\PY{l+s+s2}{\PYZdq{}}\PY{l+s+s2}{estrato\PYZus{}asalariados\PYZus{}grupo}\PY{l+s+s2}{\PYZdq{}}\PY{p}{,} 
                           \PY{n}{xlabel}\PY{o}{=}\PY{l+s+s2}{\PYZdq{}}\PY{l+s+s2}{Año}\PY{l+s+s2}{\PYZdq{}}\PY{p}{,} \PY{n}{ylabel}\PY{o}{=}\PY{l+s+s2}{\PYZdq{}}\PY{l+s+s2}{Número de empresas}\PY{l+s+s2}{\PYZdq{}}\PY{p}{,} 
                           \PY{n}{title}\PY{o}{=}\PY{l+s+s2}{\PYZdq{}}\PY{l+s+s2}{Empresas por Estrato de Asalariados}\PY{l+s+s2}{\PYZdq{}}\PY{p}{,} 
                           \PY{n}{xticks\PYZus{}rotation}\PY{o}{=}\PY{l+m+mi}{0}\PY{p}{,} \PY{n}{style}\PY{o}{=}\PY{l+s+s2}{\PYZdq{}}\PY{l+s+s2}{whitegrid}\PY{l+s+s2}{\PYZdq{}}\PY{p}{,} \PY{n}{figsize}\PY{o}{=}\PY{p}{(}\PY{l+m+mi}{12}\PY{p}{,} \PY{l+m+mi}{7}\PY{p}{)}\PY{p}{)}
\end{Verbatim}
\end{tcolorbox}

    \begin{center}
    \adjustimage{max size={0.9\linewidth}{0.9\paperheight}}{tfm_project_files/tfm_project_88_0.png}
    \end{center}
    { \hspace*{\fill} \\}
    
    \begin{tcolorbox}[breakable, size=fbox, boxrule=1pt, pad at break*=1mm,colback=cellbackground, colframe=cellborder]
\prompt{In}{incolor}{36}{\boxspacing}
\begin{Verbatim}[commandchars=\\\{\}]
\PY{n}{empresas\PYZus{}pequeñas\PYZus{}ccaa} \PY{o}{=} \PY{p}{(}\PY{n}{empresas}\PY{p}{[}\PY{n}{empresas}\PY{o}{.}\PY{n}{ccaa}\PY{o}{.}\PY{n}{isin}\PY{p}{(}\PY{n}{CCAA\PYZus{}mostrar}\PY{p}{)} \PY{o}{\PYZam{}} 
                                  \PY{p}{(}\PY{n}{empresas}\PY{o}{.}\PY{n}{estrato\PYZus{}asalariados}\PY{o}{.}\PY{n}{isin}\PY{p}{(}\PY{p}{[}\PY{l+s+s1}{\PYZsq{}}\PY{l+s+s1}{De 1 a 2}\PY{l+s+s1}{\PYZsq{}}\PY{p}{,} \PY{l+s+s1}{\PYZsq{}}\PY{l+s+s1}{De 3 a 5}\PY{l+s+s1}{\PYZsq{}}\PY{p}{,} \PY{l+s+s1}{\PYZsq{}}\PY{l+s+s1}{De 6 a 9}\PY{l+s+s1}{\PYZsq{}}\PY{p}{,} \PY{l+s+s1}{\PYZsq{}}\PY{l+s+s1}{De 10 a 19}\PY{l+s+s1}{\PYZsq{}}\PY{p}{]}\PY{p}{)}\PY{p}{)} 
                                  \PY{o}{\PYZam{}} \PY{p}{(}\PY{n}{empresas}\PY{o}{.}\PY{n}{actividad\PYZus{}principal} \PY{o}{==} \PY{l+s+s2}{\PYZdq{}}\PY{l+s+s2}{Total CNAE}\PY{l+s+s2}{\PYZdq{}}\PY{p}{)}\PY{p}{]}
                                  \PY{o}{.}\PY{n}{groupby}\PY{p}{(}\PY{p}{[}\PY{l+s+s1}{\PYZsq{}}\PY{l+s+s1}{periodo}\PY{l+s+s1}{\PYZsq{}}\PY{p}{,} \PY{l+s+s1}{\PYZsq{}}\PY{l+s+s1}{ccaa}\PY{l+s+s1}{\PYZsq{}}\PY{p}{]}\PY{p}{,} \PY{n}{as\PYZus{}index}\PY{o}{=}\PY{k+kc}{False}\PY{p}{)}
                                  \PY{o}{.}\PY{n}{agg}\PY{p}{(}\PY{p}{\PYZob{}}\PY{l+s+s1}{\PYZsq{}}\PY{l+s+s1}{total\PYZus{}empresas}\PY{l+s+s1}{\PYZsq{}}\PY{p}{:} \PY{l+s+s1}{\PYZsq{}}\PY{l+s+s1}{sum}\PY{l+s+s1}{\PYZsq{}}\PY{p}{\PYZcb{}}\PY{p}{)}\PY{p}{)}
\PY{n}{total\PYZus{}empresas} \PY{o}{=} \PY{p}{(}\PY{n}{empresas}\PY{p}{[}\PY{p}{(}\PY{n}{empresas}\PY{o}{.}\PY{n}{estrato\PYZus{}asalariados} \PY{o}{==} \PY{l+s+s2}{\PYZdq{}}\PY{l+s+s2}{Total}\PY{l+s+s2}{\PYZdq{}}\PY{p}{)} \PY{o}{\PYZam{}} 
                          \PY{p}{(}\PY{n}{empresas}\PY{o}{.}\PY{n}{actividad\PYZus{}principal} \PY{o}{==} \PY{l+s+s2}{\PYZdq{}}\PY{l+s+s2}{Total CNAE}\PY{l+s+s2}{\PYZdq{}}\PY{p}{)} \PY{p}{]}
                          \PY{o}{.}\PY{n}{rename}\PY{p}{(}\PY{n}{columns}\PY{o}{=}\PY{p}{\PYZob{}}\PY{l+s+s1}{\PYZsq{}}\PY{l+s+s1}{total\PYZus{}empresas}\PY{l+s+s1}{\PYZsq{}}\PY{p}{:} \PY{l+s+s1}{\PYZsq{}}\PY{l+s+s1}{total\PYZus{}empresas\PYZus{}total}\PY{l+s+s1}{\PYZsq{}}\PY{p}{\PYZcb{}}\PY{p}{)}\PY{p}{)}
\PY{n}{empresas\PYZus{}pequeñas\PYZus{}ccaa} \PY{o}{=} \PY{n}{empresas\PYZus{}pequeñas\PYZus{}ccaa}\PY{o}{.}\PY{n}{merge}\PY{p}{(}\PY{n}{total\PYZus{}empresas}\PY{p}{,} 
                                                      \PY{n}{on} \PY{o}{=} \PY{p}{[}\PY{l+s+s1}{\PYZsq{}}\PY{l+s+s1}{periodo}\PY{l+s+s1}{\PYZsq{}}\PY{p}{,} \PY{l+s+s1}{\PYZsq{}}\PY{l+s+s1}{ccaa}\PY{l+s+s1}{\PYZsq{}}\PY{p}{]}\PY{p}{,} 
                                                      \PY{n}{how}\PY{o}{=}\PY{l+s+s2}{\PYZdq{}}\PY{l+s+s2}{left}\PY{l+s+s2}{\PYZdq{}}\PY{p}{)}
\PY{n}{empresas\PYZus{}pequeñas\PYZus{}ccaa}\PY{p}{[}\PY{l+s+s1}{\PYZsq{}}\PY{l+s+s1}{porcentaje\PYZus{}empresas\PYZus{}pequeñas}\PY{l+s+s1}{\PYZsq{}}\PY{p}{]} \PY{o}{=} \PY{n}{empresas\PYZus{}pequeñas\PYZus{}ccaa}\PY{p}{[}\PY{l+s+s1}{\PYZsq{}}\PY{l+s+s1}{total\PYZus{}empresas}\PY{l+s+s1}{\PYZsq{}}\PY{p}{]}\PY{o}{/}\PY{n}{empresas\PYZus{}pequeñas\PYZus{}ccaa}\PY{p}{[}\PY{l+s+s1}{\PYZsq{}}\PY{l+s+s1}{total\PYZus{}empresas\PYZus{}total}\PY{l+s+s1}{\PYZsq{}}\PY{p}{]}
\PY{n}{p}\PY{o}{.}\PY{n}{create\PYZus{}multi\PYZus{}category\PYZus{}plot}\PY{p}{(}\PY{n}{data} \PY{o}{=} \PY{n}{empresas\PYZus{}pequeñas\PYZus{}ccaa}\PY{p}{,} \PY{n}{x\PYZus{}col}\PY{o}{=}\PY{l+s+s2}{\PYZdq{}}\PY{l+s+s2}{periodo}\PY{l+s+s2}{\PYZdq{}}\PY{p}{,} 
                             \PY{n}{y\PYZus{}col}\PY{o}{=}\PY{l+s+s2}{\PYZdq{}}\PY{l+s+s2}{porcentaje\PYZus{}empresas\PYZus{}pequeñas}\PY{l+s+s2}{\PYZdq{}}\PY{p}{,} 
                             \PY{n}{label} \PY{o}{=} \PY{l+s+s2}{\PYZdq{}}\PY{l+s+s2}{CCAA}\PY{l+s+s2}{\PYZdq{}}\PY{p}{,}\PY{n}{category\PYZus{}col}\PY{o}{=}\PY{l+s+s2}{\PYZdq{}}\PY{l+s+s2}{ccaa}\PY{l+s+s2}{\PYZdq{}}\PY{p}{,} 
                           \PY{n}{xlabel}\PY{o}{=}\PY{l+s+s2}{\PYZdq{}}\PY{l+s+s2}{Año}\PY{l+s+s2}{\PYZdq{}}\PY{p}{,} \PY{n}{ylabel}\PY{o}{=}\PY{l+s+s2}{\PYZdq{}}\PY{l+s+s2}{Porcentaje (}\PY{l+s+s2}{\PYZpc{}}\PY{l+s+s2}{)}\PY{l+s+s2}{\PYZdq{}}\PY{p}{,} 
                           \PY{n}{title}\PY{o}{=}\PY{l+s+s2}{\PYZdq{}}\PY{l+s+s2}{Proporción de empresas pequeñas (1 a 20 trabajadores) por CCAA}\PY{l+s+s2}{\PYZdq{}}\PY{p}{,} 
                           \PY{n}{xticks\PYZus{}rotation}\PY{o}{=}\PY{l+m+mi}{0}\PY{p}{,} \PY{n}{style}\PY{o}{=}\PY{l+s+s2}{\PYZdq{}}\PY{l+s+s2}{whitegrid}\PY{l+s+s2}{\PYZdq{}}\PY{p}{,} 
                           \PY{n}{figsize}\PY{o}{=}\PY{p}{(}\PY{l+m+mi}{12}\PY{p}{,} \PY{l+m+mi}{7}\PY{p}{)}\PY{p}{)}
\end{Verbatim}
\end{tcolorbox}

    \begin{center}
    \adjustimage{max size={0.9\linewidth}{0.9\paperheight}}{tfm_project_files/tfm_project_89_0.png}
    \end{center}
    { \hspace*{\fill} \\}
    
    \begin{tcolorbox}[breakable, size=fbox, boxrule=1pt, pad at break*=1mm,colback=cellbackground, colframe=cellborder]
\prompt{In}{incolor}{37}{\boxspacing}
\begin{Verbatim}[commandchars=\\\{\}]
\PY{n}{empresas\PYZus{}pequeñas\PYZus{}ccaa} \PY{o}{=} \PY{p}{(}\PY{n}{empresas}\PY{p}{[}\PY{n}{empresas}\PY{o}{.}\PY{n}{ccaa}\PY{o}{.}\PY{n}{isin}\PY{p}{(}\PY{n}{CCAA\PYZus{}mostrar}\PY{p}{)} \PY{o}{\PYZam{}} 
                            \PY{o}{\PYZti{}}\PY{p}{(}\PY{n}{empresas}\PY{o}{.}\PY{n}{estrato\PYZus{}asalariados}\PY{o}{.}\PY{n}{isin}\PY{p}{(}\PY{p}{[}\PY{l+s+s1}{\PYZsq{}}\PY{l+s+s1}{Sin asalariados}\PY{l+s+s1}{\PYZsq{}}\PY{p}{,}
                                                                  \PY{l+s+s1}{\PYZsq{}}\PY{l+s+s1}{De 1 a 2}\PY{l+s+s1}{\PYZsq{}}\PY{p}{,} 
                                                                  \PY{l+s+s1}{\PYZsq{}}\PY{l+s+s1}{De 3 a 5}\PY{l+s+s1}{\PYZsq{}}\PY{p}{,} 
                                                                  \PY{l+s+s1}{\PYZsq{}}\PY{l+s+s1}{De 6 a 9}\PY{l+s+s1}{\PYZsq{}}\PY{p}{,} 
                                                                  \PY{l+s+s1}{\PYZsq{}}\PY{l+s+s1}{De 10 a 19}\PY{l+s+s1}{\PYZsq{}}\PY{p}{]}\PY{p}{)}\PY{p}{)}  
                                 \PY{o}{\PYZam{}} \PY{p}{(}\PY{n}{empresas}\PY{o}{.}\PY{n}{actividad\PYZus{}principal} \PY{o}{==} \PY{l+s+s2}{\PYZdq{}}\PY{l+s+s2}{Total CNAE}\PY{l+s+s2}{\PYZdq{}}\PY{p}{)}\PY{p}{]}
                                  \PY{o}{.}\PY{n}{groupby}\PY{p}{(}\PY{p}{[}\PY{l+s+s1}{\PYZsq{}}\PY{l+s+s1}{periodo}\PY{l+s+s1}{\PYZsq{}}\PY{p}{,} \PY{l+s+s1}{\PYZsq{}}\PY{l+s+s1}{ccaa}\PY{l+s+s1}{\PYZsq{}}\PY{p}{]}\PY{p}{,} \PY{n}{as\PYZus{}index}\PY{o}{=}\PY{k+kc}{False}\PY{p}{)}
                                  \PY{o}{.}\PY{n}{agg}\PY{p}{(}\PY{p}{\PYZob{}}\PY{l+s+s1}{\PYZsq{}}\PY{l+s+s1}{total\PYZus{}empresas}\PY{l+s+s1}{\PYZsq{}}\PY{p}{:} \PY{l+s+s1}{\PYZsq{}}\PY{l+s+s1}{sum}\PY{l+s+s1}{\PYZsq{}}\PY{p}{\PYZcb{}}\PY{p}{)}\PY{p}{)}
\PY{n}{total\PYZus{}empresas} \PY{o}{=} \PY{p}{(}\PY{n}{empresas}\PY{p}{[}\PY{p}{(}\PY{n}{empresas}\PY{o}{.}\PY{n}{estrato\PYZus{}asalariados} \PY{o}{==} \PY{l+s+s2}{\PYZdq{}}\PY{l+s+s2}{Total}\PY{l+s+s2}{\PYZdq{}}\PY{p}{)} 
                          \PY{o}{\PYZam{}} \PY{p}{(}\PY{n}{empresas}\PY{o}{.}\PY{n}{actividad\PYZus{}principal} \PY{o}{==} \PY{l+s+s2}{\PYZdq{}}\PY{l+s+s2}{Total CNAE}\PY{l+s+s2}{\PYZdq{}}\PY{p}{)} \PY{p}{]}
                          \PY{o}{.}\PY{n}{rename}\PY{p}{(}\PY{n}{columns}\PY{o}{=}\PY{p}{\PYZob{}}\PY{l+s+s1}{\PYZsq{}}\PY{l+s+s1}{total\PYZus{}empresas}\PY{l+s+s1}{\PYZsq{}}\PY{p}{:} \PY{l+s+s1}{\PYZsq{}}\PY{l+s+s1}{total\PYZus{}empresas\PYZus{}total}\PY{l+s+s1}{\PYZsq{}}\PY{p}{\PYZcb{}}\PY{p}{)}\PY{p}{)}
\PY{n}{empresas\PYZus{}pequeñas\PYZus{}ccaa} \PY{o}{=} \PY{n}{empresas\PYZus{}pequeñas\PYZus{}ccaa}\PY{o}{.}\PY{n}{merge}\PY{p}{(}\PY{n}{total\PYZus{}empresas}\PY{p}{,} \PY{n}{on} \PY{o}{=} \PY{p}{[}\PY{l+s+s1}{\PYZsq{}}\PY{l+s+s1}{periodo}\PY{l+s+s1}{\PYZsq{}}\PY{p}{,} \PY{l+s+s1}{\PYZsq{}}\PY{l+s+s1}{ccaa}\PY{l+s+s1}{\PYZsq{}}\PY{p}{]}\PY{p}{,} \PY{n}{how}\PY{o}{=}\PY{l+s+s2}{\PYZdq{}}\PY{l+s+s2}{left}\PY{l+s+s2}{\PYZdq{}}\PY{p}{)}
\PY{n}{empresas\PYZus{}pequeñas\PYZus{}ccaa}\PY{p}{[}\PY{l+s+s1}{\PYZsq{}}\PY{l+s+s1}{porcentaje\PYZus{}empresas\PYZus{}pequeñas}\PY{l+s+s1}{\PYZsq{}}\PY{p}{]} \PY{o}{=} \PY{n}{empresas\PYZus{}pequeñas\PYZus{}ccaa}\PY{p}{[}\PY{l+s+s1}{\PYZsq{}}\PY{l+s+s1}{total\PYZus{}empresas}\PY{l+s+s1}{\PYZsq{}}\PY{p}{]}\PY{o}{/}\PY{n}{empresas\PYZus{}pequeñas\PYZus{}ccaa}\PY{p}{[}\PY{l+s+s1}{\PYZsq{}}\PY{l+s+s1}{total\PYZus{}empresas\PYZus{}total}\PY{l+s+s1}{\PYZsq{}}\PY{p}{]}
\PY{n}{p}\PY{o}{.}\PY{n}{create\PYZus{}multi\PYZus{}category\PYZus{}plot}\PY{p}{(}\PY{n}{data} \PY{o}{=} \PY{n}{empresas\PYZus{}pequeñas\PYZus{}ccaa}\PY{p}{,} 
                             \PY{n}{x\PYZus{}col}\PY{o}{=}\PY{l+s+s2}{\PYZdq{}}\PY{l+s+s2}{periodo}\PY{l+s+s2}{\PYZdq{}}\PY{p}{,} \PY{n}{y\PYZus{}col}\PY{o}{=}\PY{l+s+s2}{\PYZdq{}}\PY{l+s+s2}{porcentaje\PYZus{}empresas\PYZus{}pequeñas}\PY{l+s+s2}{\PYZdq{}}\PY{p}{,} 
                           \PY{n}{label} \PY{o}{=} \PY{l+s+s2}{\PYZdq{}}\PY{l+s+s2}{CCAA}\PY{l+s+s2}{\PYZdq{}}\PY{p}{,}\PY{n}{category\PYZus{}col}\PY{o}{=}\PY{l+s+s2}{\PYZdq{}}\PY{l+s+s2}{ccaa}\PY{l+s+s2}{\PYZdq{}}\PY{p}{,} 
                           \PY{n}{xlabel}\PY{o}{=}\PY{l+s+s2}{\PYZdq{}}\PY{l+s+s2}{Año}\PY{l+s+s2}{\PYZdq{}}\PY{p}{,} \PY{n}{ylabel}\PY{o}{=}\PY{l+s+s2}{\PYZdq{}}\PY{l+s+s2}{Porcentaje (}\PY{l+s+s2}{\PYZpc{}}\PY{l+s+s2}{)}\PY{l+s+s2}{\PYZdq{}}\PY{p}{,} 
                           \PY{n}{title}\PY{o}{=}\PY{l+s+s2}{\PYZdq{}}\PY{l+s+s2}{Proporción de empresas con \PYZgt{}20 trabajadores por CCAA}\PY{l+s+s2}{\PYZdq{}}\PY{p}{,} 
                           \PY{n}{xticks\PYZus{}rotation}\PY{o}{=}\PY{l+m+mi}{0}\PY{p}{,} \PY{n}{style}\PY{o}{=}\PY{l+s+s2}{\PYZdq{}}\PY{l+s+s2}{whitegrid}\PY{l+s+s2}{\PYZdq{}}\PY{p}{,} 
                           \PY{n}{figsize}\PY{o}{=}\PY{p}{(}\PY{l+m+mi}{12}\PY{p}{,} \PY{l+m+mi}{7}\PY{p}{)}\PY{p}{)}
\end{Verbatim}
\end{tcolorbox}

    \begin{center}
    \adjustimage{max size={0.9\linewidth}{0.9\paperheight}}{tfm_project_files/tfm_project_90_0.png}
    \end{center}
    { \hspace*{\fill} \\}
    
    Se observa que el mercado está dominado por pequeñas y medianas empresas
y por empresas sin asalariados, mientras que las grandes empresas tienen
una presencia prácticamente despreciable. A nivel de comunidad autónoma
se aprecia que la proporción de empresas con pocos empleados es dispar
entre comunidades autónomas, y que esta disparidad no guarda un
ordenamiento parecido al que hayamos observado en el pib o la
productividad.

En lo que respecta a las empresas con más de 20 empleados, estas suponen
un valor residual que no explica el decrecimiento porcentual de las
empresas pequeñas a partir de 2015, y este vendría principalmente
explicado por el incremento de empresas sin asalariados.

    \subsubsection{Pobreza y Desigualdad}\label{pobreza-y-desigualdad}

    La pobreza y la desigualdad suponen dos factores que nivel conceptual no
deberían tener efecto como variable de entrada en un modelo en ninguna
de las variables mencionadas previamente, pero sí que son una variable
de salida interesante a estudiar, pues los incrementos de salario mínimo
se justifican bajo la promeso de reducir desigualdad y pobreza.

    \begin{tcolorbox}[breakable, size=fbox, boxrule=1pt, pad at break*=1mm,colback=cellbackground, colframe=cellborder]
\prompt{In}{incolor}{38}{\boxspacing}
\begin{Verbatim}[commandchars=\\\{\}]
\PY{n}{pobreza} \PY{o}{=} \PY{n}{pd}\PY{o}{.}\PY{n}{read\PYZus{}csv}\PY{p}{(}\PY{l+s+s2}{\PYZdq{}}\PY{l+s+s2}{../../processed\PYZus{}data/pobreza/riesgo\PYZus{}pobreza.csv}\PY{l+s+s2}{\PYZdq{}}\PY{p}{)}
\PY{n}{pobreza\PYZus{}nacional} \PY{o}{=} \PY{n}{pobreza}\PY{p}{[}\PY{n}{pobreza}\PY{p}{[}\PY{l+s+s1}{\PYZsq{}}\PY{l+s+s1}{ccaa}\PY{l+s+s1}{\PYZsq{}}\PY{p}{]} \PY{o}{==} \PY{l+s+s2}{\PYZdq{}}\PY{l+s+s2}{Total Nacional}\PY{l+s+s2}{\PYZdq{}}\PY{p}{]}
\PY{n}{p}\PY{o}{.}\PY{n}{create\PYZus{}multi\PYZus{}category\PYZus{}plot}\PY{p}{(}\PY{n}{data} \PY{o}{=} \PY{n}{pobreza\PYZus{}nacional}\PY{p}{,} \PY{n}{x\PYZus{}col}\PY{o}{=}\PY{l+s+s2}{\PYZdq{}}\PY{l+s+s2}{periodo}\PY{l+s+s2}{\PYZdq{}}\PY{p}{,} \PY{n}{y\PYZus{}col}\PY{o}{=}\PY{l+s+s2}{\PYZdq{}}\PY{l+s+s2}{total}\PY{l+s+s2}{\PYZdq{}}\PY{p}{,} 
                           \PY{n}{label} \PY{o}{=} \PY{l+s+s2}{\PYZdq{}}\PY{l+s+s2}{Tasa}\PY{l+s+s2}{\PYZdq{}}\PY{p}{,}\PY{n}{category\PYZus{}col}\PY{o}{=}\PY{l+s+s2}{\PYZdq{}}\PY{l+s+s2}{riesgo\PYZus{}pobreza}\PY{l+s+s2}{\PYZdq{}}\PY{p}{,} 
                           \PY{n}{xlabel}\PY{o}{=}\PY{l+s+s2}{\PYZdq{}}\PY{l+s+s2}{Año}\PY{l+s+s2}{\PYZdq{}}\PY{p}{,} \PY{n}{ylabel}\PY{o}{=}\PY{l+s+s2}{\PYZdq{}}\PY{l+s+s2}{Porcentaje (}\PY{l+s+s2}{\PYZpc{}}\PY{l+s+s2}{)}\PY{l+s+s2}{\PYZdq{}}\PY{p}{,} 
                           \PY{n}{title}\PY{o}{=}\PY{l+s+s2}{\PYZdq{}}\PY{l+s+s2}{Tasa de riesgo de pobreza nacional}\PY{l+s+s2}{\PYZdq{}}\PY{p}{,} 
                           \PY{n}{xticks\PYZus{}rotation}\PY{o}{=}\PY{l+m+mi}{0}\PY{p}{,} \PY{n}{style}\PY{o}{=}\PY{l+s+s2}{\PYZdq{}}\PY{l+s+s2}{whitegrid}\PY{l+s+s2}{\PYZdq{}}\PY{p}{,} \PY{n}{figsize}\PY{o}{=}\PY{p}{(}\PY{l+m+mi}{12}\PY{p}{,} \PY{l+m+mi}{7}\PY{p}{)}\PY{p}{)}
\end{Verbatim}
\end{tcolorbox}

    \begin{center}
    \adjustimage{max size={0.9\linewidth}{0.9\paperheight}}{tfm_project_files/tfm_project_94_0.png}
    \end{center}
    { \hspace*{\fill} \\}
    
    Vemos que a nivel nacional el riesgo de pobreza parece no tener relación
con las subidas de salario mínimo. De hecho, tras 2015 comienza una
tendencia decreciente (exceptuando por 2021, que corresponde al riesgo
de pobreza en el año de pandemia), coincidiendo con la tendencia
creciente del salario mínimo.

Veamos cómo es la evolución a nivel de comunidad autónoma con alquiler
imputado.

    \begin{tcolorbox}[breakable, size=fbox, boxrule=1pt, pad at break*=1mm,colback=cellbackground, colframe=cellborder]
\prompt{In}{incolor}{40}{\boxspacing}
\begin{Verbatim}[commandchars=\\\{\}]
\PY{n}{pobreza\PYZus{}ccaa} \PY{o}{=} \PY{n}{pobreza}\PY{p}{[}\PY{p}{(}\PY{n}{pobreza}\PY{p}{[}\PY{l+s+s1}{\PYZsq{}}\PY{l+s+s1}{riesgo\PYZus{}pobreza}\PY{l+s+s1}{\PYZsq{}}\PY{p}{]} \PY{o}{==} \PY{l+s+s2}{\PYZdq{}}\PY{l+s+s2}{Tasa de riesgo de pobreza (con alquiler imputado) (renta del año anterior a la entrevista)}\PY{l+s+s2}{\PYZdq{}}\PY{p}{)}
                        \PY{o}{\PYZam{}} \PY{p}{(}\PY{n}{pobreza}\PY{p}{[}\PY{l+s+s1}{\PYZsq{}}\PY{l+s+s1}{ccaa}\PY{l+s+s1}{\PYZsq{}}\PY{p}{]}\PY{o}{.}\PY{n}{isin}\PY{p}{(}\PY{n}{CCAA\PYZus{}mostrar}\PY{p}{)}\PY{p}{)}\PY{p}{]}
\PY{n}{p}\PY{o}{.}\PY{n}{create\PYZus{}multi\PYZus{}category\PYZus{}plot}\PY{p}{(}\PY{n}{data} \PY{o}{=} \PY{n}{pobreza\PYZus{}ccaa}\PY{p}{,} \PY{n}{x\PYZus{}col}\PY{o}{=}\PY{l+s+s2}{\PYZdq{}}\PY{l+s+s2}{periodo}\PY{l+s+s2}{\PYZdq{}}\PY{p}{,} \PY{n}{y\PYZus{}col}\PY{o}{=}\PY{l+s+s2}{\PYZdq{}}\PY{l+s+s2}{total}\PY{l+s+s2}{\PYZdq{}}\PY{p}{,} \PY{n}{label} \PY{o}{=} \PY{l+s+s2}{\PYZdq{}}\PY{l+s+s2}{CCAA}\PY{l+s+s2}{\PYZdq{}}\PY{p}{,}\PY{n}{category\PYZus{}col}\PY{o}{=}\PY{l+s+s2}{\PYZdq{}}\PY{l+s+s2}{ccaa}\PY{l+s+s2}{\PYZdq{}}\PY{p}{,} 
                           \PY{n}{xlabel}\PY{o}{=}\PY{l+s+s2}{\PYZdq{}}\PY{l+s+s2}{Año}\PY{l+s+s2}{\PYZdq{}}\PY{p}{,} \PY{n}{ylabel}\PY{o}{=}\PY{l+s+s2}{\PYZdq{}}\PY{l+s+s2}{Porcentaje (}\PY{l+s+s2}{\PYZpc{}}\PY{l+s+s2}{)}\PY{l+s+s2}{\PYZdq{}}\PY{p}{,} \PY{n}{title}\PY{o}{=}\PY{l+s+s2}{\PYZdq{}}\PY{l+s+s2}{Tasa de riesgo de pobreza por CCAA}\PY{l+s+s2}{\PYZdq{}}\PY{p}{,} 
                           \PY{n}{xticks\PYZus{}rotation}\PY{o}{=}\PY{l+m+mi}{0}\PY{p}{,} \PY{n}{style}\PY{o}{=}\PY{l+s+s2}{\PYZdq{}}\PY{l+s+s2}{whitegrid}\PY{l+s+s2}{\PYZdq{}}\PY{p}{,} \PY{n}{figsize}\PY{o}{=}\PY{p}{(}\PY{l+m+mi}{12}\PY{p}{,} \PY{l+m+mi}{7}\PY{p}{)}\PY{p}{)}
\end{Verbatim}
\end{tcolorbox}

    \begin{center}
    \adjustimage{max size={0.9\linewidth}{0.9\paperheight}}{tfm_project_files/tfm_project_96_0.png}
    \end{center}
    { \hspace*{\fill} \\}
    
    Es interesante observar que las comunidades más pobres no solo tienen
tasas más altas, sino que estas son considerablemente más volátiles en
el tiempo.

Veamos cómo evoluciona la desigualdad medida en :

\begin{itemize}
\item
  \textbf{Índice de Gini:} Mide cuán distribuido está la renta a través
  de la curva de Lorenz, siendo 0 una perfecta igualdad y 100 el caso en
  el que una persona tiene toda la riqueza evaluada.
\item
  \textbf{Índice S80/S20}: Es el cociente entre el ingreso del 20\% más
  rico de la población entre el 20\% más pobre de la población. Se
  diferencia del índice Gini en que es más específico para ver la
  diferencia entre clases más ricas y más pobres en lugar de analizar
  una distribución global.
\end{itemize}

    \begin{tcolorbox}[breakable, size=fbox, boxrule=1pt, pad at break*=1mm,colback=cellbackground, colframe=cellborder]
\prompt{In}{incolor}{41}{\boxspacing}
\begin{Verbatim}[commandchars=\\\{\}]
\PY{n}{desigualdad} \PY{o}{=} \PY{n}{pd}\PY{o}{.}\PY{n}{read\PYZus{}csv}\PY{p}{(}\PY{l+s+s2}{\PYZdq{}}\PY{l+s+s2}{../../processed\PYZus{}data/pobreza/desigualdad.csv}\PY{l+s+s2}{\PYZdq{}}\PY{p}{)}
\PY{n}{desigualdad\PYZus{}nacional\PYZus{}gini} \PY{o}{=} \PY{n}{desigualdad}\PY{p}{[}\PY{p}{(}\PY{n}{desigualdad}\PY{p}{[}\PY{l+s+s1}{\PYZsq{}}\PY{l+s+s1}{ccaa}\PY{l+s+s1}{\PYZsq{}}\PY{p}{]} \PY{o}{==} \PY{l+s+s2}{\PYZdq{}}\PY{l+s+s2}{Total Nacional}\PY{l+s+s2}{\PYZdq{}}\PY{p}{)} \PY{o}{\PYZam{}} 
                                        \PY{p}{(}\PY{n}{desigualdad}\PY{p}{[}\PY{l+s+s1}{\PYZsq{}}\PY{l+s+s1}{desigualdad}\PY{l+s+s1}{\PYZsq{}}\PY{p}{]} \PY{o}{==} \PY{l+s+s2}{\PYZdq{}}\PY{l+s+s2}{Gini}\PY{l+s+s2}{\PYZdq{}}\PY{p}{)}\PY{p}{]}
\PY{n}{desigualdad\PYZus{}nacional\PYZus{}8020} \PY{o}{=} \PY{n}{desigualdad}\PY{p}{[}\PY{p}{(}\PY{n}{desigualdad}\PY{p}{[}\PY{l+s+s1}{\PYZsq{}}\PY{l+s+s1}{ccaa}\PY{l+s+s1}{\PYZsq{}}\PY{p}{]} \PY{o}{==} \PY{l+s+s2}{\PYZdq{}}\PY{l+s+s2}{Total Nacional}\PY{l+s+s2}{\PYZdq{}}\PY{p}{)} \PY{o}{\PYZam{}} 
                                        \PY{p}{(}\PY{n}{desigualdad}\PY{p}{[}\PY{l+s+s1}{\PYZsq{}}\PY{l+s+s1}{desigualdad}\PY{l+s+s1}{\PYZsq{}}\PY{p}{]} \PY{o}{==} \PY{l+s+s2}{\PYZdq{}}\PY{l+s+s2}{Distribución de la renta S80/S20}\PY{l+s+s2}{\PYZdq{}}\PY{p}{)}\PY{p}{]}
\PY{n}{p}\PY{o}{.}\PY{n}{create\PYZus{}dual\PYZus{}plot}\PY{p}{(}\PY{n}{desigualdad\PYZus{}nacional\PYZus{}gini}\PY{p}{[}\PY{l+s+s1}{\PYZsq{}}\PY{l+s+s1}{periodo}\PY{l+s+s1}{\PYZsq{}}\PY{p}{]}\PY{p}{,} \PY{n}{desigualdad\PYZus{}nacional\PYZus{}gini}\PY{p}{[}\PY{l+s+s1}{\PYZsq{}}\PY{l+s+s1}{total}\PY{l+s+s1}{\PYZsq{}}\PY{p}{]}\PY{p}{,} 
                 \PY{n}{desigualdad\PYZus{}nacional\PYZus{}8020}\PY{p}{[}\PY{l+s+s1}{\PYZsq{}}\PY{l+s+s1}{total}\PY{l+s+s1}{\PYZsq{}}\PY{p}{]}\PY{p}{,} 
                 \PY{n}{xlabel}\PY{o}{=}\PY{l+s+s2}{\PYZdq{}}\PY{l+s+s2}{Año}\PY{l+s+s2}{\PYZdq{}}\PY{p}{,} \PY{n}{ylabel1}\PY{o}{=}\PY{l+s+s2}{\PYZdq{}}\PY{l+s+s2}{Gini}\PY{l+s+s2}{\PYZdq{}}\PY{p}{,} \PY{n}{ylabel2}\PY{o}{=}\PY{l+s+s2}{\PYZdq{}}\PY{l+s+s2}{S20/S80}\PY{l+s+s2}{\PYZdq{}} \PY{p}{,}
                 \PY{n}{label1}\PY{o}{=} \PY{l+s+s2}{\PYZdq{}}\PY{l+s+s2}{Gini}\PY{l+s+s2}{\PYZdq{}}\PY{p}{,} \PY{n}{label2}\PY{o}{=}\PY{l+s+s2}{\PYZdq{}}\PY{l+s+s2}{S80/S20}\PY{l+s+s2}{\PYZdq{}}\PY{p}{,} 
                 \PY{n}{title}\PY{o}{=}\PY{l+s+s2}{\PYZdq{}}\PY{l+s+s2}{Desigualdad Nacional Media en Índice de Gini y S20/S80}\PY{l+s+s2}{\PYZdq{}}\PY{p}{,}
                 \PY{n}{xticks\PYZus{}rotation}\PY{o}{=}\PY{l+m+mi}{0}\PY{p}{,} \PY{n}{style}\PY{o}{=}\PY{l+s+s2}{\PYZdq{}}\PY{l+s+s2}{whitegrid}\PY{l+s+s2}{\PYZdq{}}\PY{p}{,} \PY{n}{figsize}\PY{o}{=}\PY{p}{(}\PY{l+m+mi}{10}\PY{p}{,} \PY{l+m+mi}{6}\PY{p}{)}\PY{p}{,} \PY{n}{marker1}\PY{o}{=}\PY{k+kc}{None}\PY{p}{,} 
                 \PY{n}{marker2}\PY{o}{=}\PY{k+kc}{None}\PY{p}{,} \PY{n}{secondary\PYZus{}y}\PY{o}{=}\PY{k+kc}{True} \PY{p}{)}
\end{Verbatim}
\end{tcolorbox}

    \begin{center}
    \adjustimage{max size={0.9\linewidth}{0.9\paperheight}}{tfm_project_files/tfm_project_98_0.png}
    \end{center}
    { \hspace*{\fill} \\}
    
    \begin{tcolorbox}[breakable, size=fbox, boxrule=1pt, pad at break*=1mm,colback=cellbackground, colframe=cellborder]
\prompt{In}{incolor}{42}{\boxspacing}
\begin{Verbatim}[commandchars=\\\{\}]
\PY{n}{desigualdad\PYZus{}gini\PYZus{}ccaa} \PY{o}{=} \PY{n}{desigualdad}\PY{p}{[}\PY{p}{(}\PY{n}{desigualdad}\PY{p}{[}\PY{l+s+s1}{\PYZsq{}}\PY{l+s+s1}{desigualdad}\PY{l+s+s1}{\PYZsq{}}\PY{p}{]} \PY{o}{==} \PY{l+s+s2}{\PYZdq{}}\PY{l+s+s2}{Gini}\PY{l+s+s2}{\PYZdq{}}\PY{p}{)} 
                                    \PY{o}{\PYZam{}} \PY{p}{(}\PY{n}{desigualdad}\PY{p}{[}\PY{l+s+s1}{\PYZsq{}}\PY{l+s+s1}{ccaa}\PY{l+s+s1}{\PYZsq{}}\PY{p}{]}\PY{o}{.}\PY{n}{isin}\PY{p}{(}\PY{n}{CCAA\PYZus{}mostrar}\PY{p}{)}\PY{p}{)}\PY{p}{]}
\PY{n}{p}\PY{o}{.}\PY{n}{create\PYZus{}multi\PYZus{}category\PYZus{}plot}\PY{p}{(}\PY{n}{data} \PY{o}{=} \PY{n}{desigualdad\PYZus{}gini\PYZus{}ccaa}\PY{p}{,} \PY{n}{x\PYZus{}col}\PY{o}{=}\PY{l+s+s2}{\PYZdq{}}\PY{l+s+s2}{periodo}\PY{l+s+s2}{\PYZdq{}}\PY{p}{,} 
                             \PY{n}{y\PYZus{}col}\PY{o}{=}\PY{l+s+s2}{\PYZdq{}}\PY{l+s+s2}{total}\PY{l+s+s2}{\PYZdq{}}\PY{p}{,} \PY{n}{label} \PY{o}{=} \PY{l+s+s2}{\PYZdq{}}\PY{l+s+s2}{CCAA}\PY{l+s+s2}{\PYZdq{}}\PY{p}{,}\PY{n}{category\PYZus{}col}\PY{o}{=}\PY{l+s+s2}{\PYZdq{}}\PY{l+s+s2}{ccaa}\PY{l+s+s2}{\PYZdq{}}\PY{p}{,} 
                           \PY{n}{xlabel}\PY{o}{=}\PY{l+s+s2}{\PYZdq{}}\PY{l+s+s2}{Año}\PY{l+s+s2}{\PYZdq{}}\PY{p}{,} \PY{n}{ylabel}\PY{o}{=}\PY{l+s+s2}{\PYZdq{}}\PY{l+s+s2}{Índice Gini}\PY{l+s+s2}{\PYZdq{}}\PY{p}{,} 
                           \PY{n}{title}\PY{o}{=}\PY{l+s+s2}{\PYZdq{}}\PY{l+s+s2}{Índice de Gini por comunidad autónoma}\PY{l+s+s2}{\PYZdq{}}\PY{p}{,} 
                           \PY{n}{xticks\PYZus{}rotation}\PY{o}{=}\PY{l+m+mi}{0}\PY{p}{,} \PY{n}{style}\PY{o}{=}\PY{l+s+s2}{\PYZdq{}}\PY{l+s+s2}{whitegrid}\PY{l+s+s2}{\PYZdq{}}\PY{p}{,} \PY{n}{figsize}\PY{o}{=}\PY{p}{(}\PY{l+m+mi}{12}\PY{p}{,} \PY{l+m+mi}{7}\PY{p}{)}\PY{p}{)}
\end{Verbatim}
\end{tcolorbox}

    \begin{center}
    \adjustimage{max size={0.9\linewidth}{0.9\paperheight}}{tfm_project_files/tfm_project_99_0.png}
    \end{center}
    { \hspace*{\fill} \\}
    
    Los dos índices tienen un evolución muy diferenciada: el índice de Gini
muestra una evolución algo más plana o a veces decreciente para
comunidades como Navarra, mientras que en el caso del índice S80/S20 a
partir de 2015 tiene correlación inversa con el salario mínimo. Estos
resultados podrían llevarnos rápidamente a concluir que el salario
mínimo disminuye la desigualdad, pues la evolución del salario mínimo
real está inversamente correlado con la desigualdad, si bien es cierto
que a nivel autonómico la variabilidad es alta. No obstante hay que
tener en cuenta dos matices:

\begin{itemize}
\tightlist
\item
  La desigualdad nos indica la brecha entre ricos y pobres, pero no nos
  da una idea clara de cómo viven estos últimos, lo que en última
  instancia tendría más peso que la desigualdad en sí.
\item
  El resto de factores económicos puede estar condicionando también la
  evolución positiva de la desigualdad: mayor productividad, mayor PIB
  per cápita o menor paro pueden conducir también a la reducción de la
  desigualdad, por lo que debemos ser precavidos con estos resultados
  iniciales.
\end{itemize}

Con el fin de resolver la primera cuestión, podemos echar un vistazo a
la carencia material, que nos indica el porcentaje de personas que
sufren de determinadas carencias como incapacidad de mantener la viviend
a una temperatura adecuada o inpagos en sus facturas.

    \begin{tcolorbox}[breakable, size=fbox, boxrule=1pt, pad at break*=1mm,colback=cellbackground, colframe=cellborder]
\prompt{In}{incolor}{43}{\boxspacing}
\begin{Verbatim}[commandchars=\\\{\}]
\PY{n}{carencia} \PY{o}{=} \PY{n}{pd}\PY{o}{.}\PY{n}{read\PYZus{}csv}\PY{p}{(}\PY{l+s+s2}{\PYZdq{}}\PY{l+s+s2}{../../processed\PYZus{}data/pobreza/carencia\PYZus{}material.csv}\PY{l+s+s2}{\PYZdq{}}\PY{p}{)}
\PY{n}{carencias\PYZus{}severas} \PY{o}{=} \PY{p}{[}\PY{l+s+s1}{\PYZsq{}}\PY{l+s+s1}{No puede permitirse mantener la vivienda con una temperatura adecuada}\PY{l+s+s1}{\PYZsq{}}\PY{p}{,}
                      \PY{l+s+s1}{\PYZsq{}}\PY{l+s+s1}{No puede permitirse una comida de carne, pollo o pescado al menos cada dos días}\PY{l+s+s1}{\PYZsq{}}\PY{p}{,}
                        \PY{l+s+s1}{\PYZsq{}}\PY{l+s+s1}{Ha tenido retrasos en el pago de gastos relacionados con la vivienda principal (hipoteca o alquiler, recibos de gas, comunidad...) o en compras a plazos en los últimos 12 meses}\PY{l+s+s1}{\PYZsq{}}\PY{p}{]}
\PY{n}{carencia\PYZus{}nacional} \PY{o}{=} \PY{n}{carencia}\PY{p}{[}\PY{p}{(}\PY{n}{carencia}\PY{p}{[}\PY{l+s+s1}{\PYZsq{}}\PY{l+s+s1}{ccaa}\PY{l+s+s1}{\PYZsq{}}\PY{p}{]} \PY{o}{==} \PY{l+s+s2}{\PYZdq{}}\PY{l+s+s2}{Total Nacional}\PY{l+s+s2}{\PYZdq{}}\PY{p}{)} \PY{o}{\PYZam{}} 
                             \PY{p}{(}\PY{n}{carencia}\PY{p}{[}\PY{l+s+s1}{\PYZsq{}}\PY{l+s+s1}{carencia\PYZus{}material}\PY{l+s+s1}{\PYZsq{}}\PY{p}{]}\PY{o}{.}\PY{n}{isin}\PY{p}{(}\PY{n}{carencias\PYZus{}severas}\PY{p}{)}\PY{p}{)}\PY{p}{]}
\PY{n}{carencia\PYZus{}nacional}\PY{p}{[}\PY{l+s+s1}{\PYZsq{}}\PY{l+s+s1}{carencia\PYZus{}material}\PY{l+s+s1}{\PYZsq{}}\PY{p}{]} \PY{o}{=} \PY{n}{carencia\PYZus{}nacional}\PY{p}{[}\PY{l+s+s1}{\PYZsq{}}\PY{l+s+s1}{carencia\PYZus{}material}\PY{l+s+s1}{\PYZsq{}}\PY{p}{]}\PY{o}{.}\PY{n}{replace}\PY{p}{(}\PY{l+s+s1}{\PYZsq{}}\PY{l+s+s1}{Ha tenido retrasos en el pago de gastos relacionados con la vivienda principal (hipoteca o alquiler, recibos de gas, comunidad...) o en compras a plazos en los últimos 12 meses}\PY{l+s+s1}{\PYZsq{}}\PY{p}{,}
                                                                                       \PY{l+s+s1}{\PYZsq{}}\PY{l+s+s1}{Ha tenido retrasos en el pago de gastos relacionados con la vivienda principal}\PY{l+s+s1}{\PYZsq{}}\PY{p}{)}
\PY{n}{p}\PY{o}{.}\PY{n}{create\PYZus{}multi\PYZus{}category\PYZus{}plot}\PY{p}{(}\PY{n}{data} \PY{o}{=} \PY{n}{carencia\PYZus{}nacional}\PY{p}{,} \PY{n}{x\PYZus{}col}\PY{o}{=}\PY{l+s+s2}{\PYZdq{}}\PY{l+s+s2}{periodo}\PY{l+s+s2}{\PYZdq{}}\PY{p}{,} \PY{n}{y\PYZus{}col}\PY{o}{=}\PY{l+s+s2}{\PYZdq{}}\PY{l+s+s2}{total}\PY{l+s+s2}{\PYZdq{}}\PY{p}{,} \PY{n}{label} \PY{o}{=} \PY{l+s+s2}{\PYZdq{}}\PY{l+s+s2}{Tipo de carencia}\PY{l+s+s2}{\PYZdq{}}\PY{p}{,}
                           \PY{n}{category\PYZus{}col}\PY{o}{=}\PY{l+s+s2}{\PYZdq{}}\PY{l+s+s2}{carencia\PYZus{}material}\PY{l+s+s2}{\PYZdq{}}\PY{p}{,} 
                           \PY{n}{xlabel}\PY{o}{=}\PY{l+s+s2}{\PYZdq{}}\PY{l+s+s2}{Año}\PY{l+s+s2}{\PYZdq{}}\PY{p}{,} \PY{n}{ylabel}\PY{o}{=}\PY{l+s+s2}{\PYZdq{}}\PY{l+s+s2}{Porcentaje (}\PY{l+s+s2}{\PYZpc{}}\PY{l+s+s2}{)}\PY{l+s+s2}{\PYZdq{}}\PY{p}{,}
                           \PY{n}{title}\PY{o}{=}\PY{l+s+s2}{\PYZdq{}}\PY{l+s+s2}{Carencia material nacional}\PY{l+s+s2}{\PYZdq{}}\PY{p}{,}
                           \PY{n}{xticks\PYZus{}rotation}\PY{o}{=}\PY{l+m+mi}{0}\PY{p}{,} \PY{n}{style}\PY{o}{=}\PY{l+s+s2}{\PYZdq{}}\PY{l+s+s2}{whitegrid}\PY{l+s+s2}{\PYZdq{}}\PY{p}{,} \PY{n}{figsize}\PY{o}{=}\PY{p}{(}\PY{l+m+mi}{12}\PY{p}{,} \PY{l+m+mi}{7}\PY{p}{)}\PY{p}{,} 
                           \PY{n}{save\PYZus{}path}\PY{o}{=}\PY{l+s+s2}{\PYZdq{}}\PY{l+s+s2}{../../images/carencia\PYZus{}nacional.png}\PY{l+s+s2}{\PYZdq{}}\PY{p}{)}
\end{Verbatim}
\end{tcolorbox}

    \begin{Verbatim}[commandchars=\\\{\}]
Plot saved to ../../images/carencia\_nacional.png
    \end{Verbatim}

    \begin{center}
    \adjustimage{max size={0.9\linewidth}{0.9\paperheight}}{tfm_project_files/tfm_project_101_1.png}
    \end{center}
    { \hspace*{\fill} \\}
    
    Observamos entonces que ciertas carencias como mantener la vivienda a
una temperatura adecuada o tener retrasos en pagos tiende a decrecer de
forma medianamente aparejada al incremento del salario mínimo a partir
de 2015, pero otras carencias como no poder permitirse comer carne,
pollo y pescado cada dos días tienden a crecer con el tiempo,
duplicándose entre 2015 y 2020.

    \subsection{Creación de ratios y selección de variables para el
modelo}\label{creaciuxf3n-de-ratios-y-selecciuxf3n-de-variables-para-el-modelo}

    Tras haber analizado las variables involucradas para crear un estado de
partida y evaluar los posibles resultados debemos escoger las variables
más adecuadas y representativas que nos permitan extraer la máxima
información del estado para hacer las predicciones.

La primera cuestión a tener en cuenta en tal menester es que, dado que
el objetivo es utilizar las economías de las comunidades autónomas para
hacer comparativas bajo diferentes circunstancias, es preciso tener
magnitudes comparables entre las mismas. Los valores absolutos no son
adecuados, pues no permiten una generalización posterior y necesaria
para los propósitos del análisis. En consecuencia hemos de definir las
variables a utilizar en concepto de ratios o incrementos porcentuales (o
su equivalente como factor multiplicativo). Las variables propuestas de
las cuales se seleccionarán las más relevantes en el análisis posterior
son:

\begin{itemize}
\item
  \textbf{SMI\_VIDA}: Definido como el ratio entre el coste medio de
  vivienda y alimentos básicos de una familia en un año dado y el SMI;
  es decir, cuantos SMIs son necesarios para cubrir estos gastos básicos
  en un territorio dado.
\item
  \textbf{SMI\_MEDIO}: Salario mínimo interprofesional dividido entre el
  salario medio de la región. nos indica qué tan cerca está el salario
  mínimo en una región dada del salario medio de dicha región.
\item
  \textbf{EMPRESAS\_10, EMPRESAS\_20, EMPRESAS\_50}: Porcentaje de
  empresas con un número de trabajadores entre 1 y 9, entre 10 y 19 y
  entre 20 y 49 respectivamente. Nos dan una idea de la estructura
  empresarial general de un territorio dado.
\item
  \textbf{PARO\_1\_AÑO}: Porcentaje de parados que han estado parados
  buscando empleo por más de un año. Nos permite entender la
  distribución de parados en el mercado.
\item
  \textbf{EMP\_1\_5}: Porcentaje de trabajadores sobre el total que
  cobran entre 0 y 1,5 salarios mínimos. Es una variable que nos puede
  indicar el posible número de afectados por el incremento del salario
  mínimo.
\item
  \textbf{PARO\_25}: El ratio entre el porcentaje de paro de los
  trabajadores menores de 25 años y la tasa total de paro. Nos puede dar
  una idea de la rigidez del mercado laboral en un momento determinado.
\item
  \textbf{OC\_CONSTRUCCION, OC\_SERVICIOS}: Porcentaje de ocupados en
  construcción y servicios. Al ser estos los sectores más bajos en
  términos de retribución, nos proporciona información sobre los
  posibles afectados por el salario mínimo.
\end{itemize}

El resto de variables a probar serán ratios previamente observados en el
análisis descriptivo, que serán:

\begin{itemize}
\item
  \textbf{PARO} : Tasa de paro.
\item
  \textbf{PIB\_CAPITA}: PIB per cápita.
\item
  \textbf{PROD\_HORA}: Productividad de cada trabajador por hora.
\item
  \textbf{CARENCIA}: Porcentaje de personas que no puede permitirse
  comer pollo o pescado al menos cada dos días. Escogemos esta carencia
  en concreto porque supone una carencia relacionada con las necesidades
  más básicas, mientras que las otras se pueden ir intercambiado entre
  sí según las necesidades de las personas (p.ej alguien que decida no
  irse de vacaciones porque prefiere destinar sus pocos recursos a
  renovar sus muebles rotos).
\item
  \textbf{RIESGO\_POBREZA}: Riesgo de pobreza. Es una variable útil para
  entender tanto cuánta gente puede haber afectada por el incremento del
  salario mínimo como el posible impacto del mismo.
\item
  \textbf{INC\_SMI\_REAL}: Incremento porcentual del SMI real año contra
  año en la región dada. Esta será la variable predictora fundamental
  que nos interesa evaluar.
\item
  \textbf{DESIGUALDAD}: Desigualdad de ingresos medida mediante el
  índice S80/S20.
\item
  \textbf{PARCIAL}: Porcentaje de trabajadores a jornada parcial dentro
  del total de trabajadores.
\item
  \textbf{HORAS\_TRABAJO}: Horas trabajadas por empleado. Nos puede
  ayudar a entender los cambios en la estructura del trabajo junto con
  el paro.
\item
  \textbf{IPC}: Valor del IPC, nos ayudará a estudiar cómo afecta la
  subida de salarios mínimos a la inflación.
\end{itemize}

Notemos que a excepción de las variables derivadas del SMI el resto de
variables son tanto predictoras como predecidas, pues como hemos
mencionado antes, queremos conocer todo el estado posterior al
incremento del salario mínimo para evaluar la totalidad de sus
consecuencias. De cara a poder realizar análisis y correlaciones,
utilizaremos los incrementos porcentuales de determinadas variables, que
señalaremos con un sufijo ``\_delta1'' para indicar que mirarmos el
incremento año contra año. En el caso de la variable CARENCIA, al poder
moverse entre valores muy bajos, medir el incremento porcentual puede
dar lugar a valores muy grandes que pueden afectar al rendimiento final,
por lo que para este caso particular mediremos el incremento porcentual
absoluto año contra año.

Es importante destacar que no todas las variables definirán el estado
que se tomará como base, pero sí pueden ser variables que sean predichas
por ser de interés. Por ejemplo el índice S80/S20 podría no estar
incluida entre las variables predictoras, pero es una variable que
queremos predecir por ser uno de los principales focos de atención en
otros estudios del salario mínimo.

Por último, todas las variables que estén evaluadas en euros serán
ajustadas por inflación a precios del 2021 para que todos los análisis
se hagan sobre bases similares. El ajuste se hará a nivel de comunidad
autónoma.

    \begin{tcolorbox}[breakable, size=fbox, boxrule=1pt, pad at break*=1mm,colback=cellbackground, colframe=cellborder]
\prompt{In}{incolor}{44}{\boxspacing}
\begin{Verbatim}[commandchars=\\\{\}]
\PY{c+c1}{\PYZsh{}Creamos la tabla final con todas las variables}
\PY{n}{total\PYZus{}merge} \PY{o}{=} \PY{n}{dformat}\PY{o}{.}\PY{n}{combinar\PYZus{}tablas}\PY{p}{(}\PY{n}{gasto\PYZus{}basico}\PY{p}{,} \PY{n}{smi}\PY{p}{,} \PY{n}{pobreza}\PY{p}{,} \PY{n}{desigualdad}\PY{p}{,}
                                      \PY{n}{salarios\PYZus{}ocupacion}\PY{p}{,} \PY{n}{salarios\PYZus{}smis}\PY{p}{,} \PY{n}{empresas}\PY{p}{,}
                                      \PY{n}{ipc}\PY{p}{,} \PY{n}{pib\PYZus{}per\PYZus{}capita}\PY{p}{,} \PY{n}{productividad\PYZus{}hora}\PY{p}{,}
                                      \PY{n}{carencia}\PY{p}{,} \PY{n}{empleo\PYZus{}hora}\PY{p}{,} \PY{n}{paro}\PY{p}{,}
                                      \PY{n}{paro\PYZus{}duracion}\PY{p}{,} \PY{n}{ocupados\PYZus{}jornada}\PY{p}{)}

\PY{c+c1}{\PYZsh{}Creamos las variables restantes y ajustamos las que correspondan al IPC}
\PY{c+c1}{\PYZsh{}Preservamos solo las columnas de interés}
\PY{n}{variables} \PY{o}{=} \PY{p}{[}\PY{l+s+s1}{\PYZsq{}}\PY{l+s+s1}{ccaa}\PY{l+s+s1}{\PYZsq{}}\PY{p}{,} \PY{l+s+s1}{\PYZsq{}}\PY{l+s+s1}{periodo}\PY{l+s+s1}{\PYZsq{}}\PY{p}{,} \PY{l+s+s1}{\PYZsq{}}\PY{l+s+s1}{SMI\PYZus{}VIDA}\PY{l+s+s1}{\PYZsq{}}\PY{p}{,} \PY{l+s+s1}{\PYZsq{}}\PY{l+s+s1}{SMI\PYZus{}MEDIO}\PY{l+s+s1}{\PYZsq{}}\PY{p}{,}
              \PY{l+s+s1}{\PYZsq{}}\PY{l+s+s1}{EMPRESAS\PYZus{}10}\PY{l+s+s1}{\PYZsq{}}\PY{p}{,} \PY{l+s+s1}{\PYZsq{}}\PY{l+s+s1}{EMPRESAS\PYZus{}20}\PY{l+s+s1}{\PYZsq{}}\PY{p}{,}
              \PY{l+s+s1}{\PYZsq{}}\PY{l+s+s1}{EMPRESAS\PYZus{}50}\PY{l+s+s1}{\PYZsq{}}\PY{p}{,} \PY{l+s+s1}{\PYZsq{}}\PY{l+s+s1}{PARO\PYZus{}1\PYZus{}AÑO}\PY{l+s+s1}{\PYZsq{}}\PY{p}{,} 
              \PY{l+s+s1}{\PYZsq{}}\PY{l+s+s1}{EMP\PYZus{}1\PYZus{}5}\PY{l+s+s1}{\PYZsq{}}\PY{p}{,} \PY{l+s+s1}{\PYZsq{}}\PY{l+s+s1}{PARO\PYZus{}25}\PY{l+s+s1}{\PYZsq{}}\PY{p}{,} \PY{l+s+s1}{\PYZsq{}}\PY{l+s+s1}{OC\PYZus{}CONSTRUCCION}\PY{l+s+s1}{\PYZsq{}}\PY{p}{,} 
              \PY{l+s+s1}{\PYZsq{}}\PY{l+s+s1}{OC\PYZus{}SERVICIOS}\PY{l+s+s1}{\PYZsq{}}\PY{p}{,} \PY{l+s+s1}{\PYZsq{}}\PY{l+s+s1}{PARO}\PY{l+s+s1}{\PYZsq{}}\PY{p}{,} \PY{l+s+s1}{\PYZsq{}}\PY{l+s+s1}{PIB\PYZus{}CAPITA}\PY{l+s+s1}{\PYZsq{}}\PY{p}{,} 
              \PY{l+s+s1}{\PYZsq{}}\PY{l+s+s1}{PROD\PYZus{}HORA}\PY{l+s+s1}{\PYZsq{}}\PY{p}{,} \PY{l+s+s1}{\PYZsq{}}\PY{l+s+s1}{CARENCIA}\PY{l+s+s1}{\PYZsq{}}\PY{p}{,} \PY{l+s+s1}{\PYZsq{}}\PY{l+s+s1}{RIESGO\PYZus{}POBREZA}\PY{l+s+s1}{\PYZsq{}}\PY{p}{,} 
              \PY{l+s+s1}{\PYZsq{}}\PY{l+s+s1}{INC\PYZus{}SMI\PYZus{}REAL}\PY{l+s+s1}{\PYZsq{}}\PY{p}{,} \PY{l+s+s1}{\PYZsq{}}\PY{l+s+s1}{DESIGUALDAD}\PY{l+s+s1}{\PYZsq{}}\PY{p}{,} \PY{l+s+s1}{\PYZsq{}}\PY{l+s+s1}{PARCIAL}\PY{l+s+s1}{\PYZsq{}}\PY{p}{,} 
              \PY{l+s+s1}{\PYZsq{}}\PY{l+s+s1}{IPC}\PY{l+s+s1}{\PYZsq{}}\PY{p}{,} \PY{l+s+s1}{\PYZsq{}}\PY{l+s+s1}{HORAS\PYZus{}TRABAJO}\PY{l+s+s1}{\PYZsq{}}\PY{p}{]}

\PY{n}{df} \PY{o}{=} \PY{n}{dformat}\PY{o}{.}\PY{n}{format\PYZus{}total\PYZus{}merge}\PY{p}{(}\PY{n}{total\PYZus{}merge}\PY{p}{,} \PY{n}{variables}\PY{p}{)}
\end{Verbatim}
\end{tcolorbox}

    \begin{tcolorbox}[breakable, size=fbox, boxrule=1pt, pad at break*=1mm,colback=cellbackground, colframe=cellborder]
\prompt{In}{incolor}{45}{\boxspacing}
\begin{Verbatim}[commandchars=\\\{\}]
\PY{n+nb}{print}\PY{p}{(}\PY{l+s+s2}{\PYZdq{}}\PY{l+s+s2}{Valores nulos por columnas: }\PY{l+s+s2}{\PYZdq{}}\PY{p}{,} \PY{n}{sep}\PY{o}{=}\PY{l+s+s2}{\PYZdq{}}\PY{l+s+se}{\PYZbs{}n}\PY{l+s+s2}{\PYZdq{}}\PY{p}{)}
\PY{n}{display}\PY{p}{(}\PY{n}{df}\PY{o}{.}\PY{n}{isna}\PY{p}{(}\PY{p}{)}\PY{o}{.}\PY{n}{sum}\PY{p}{(}\PY{p}{)}\PY{p}{)}
\end{Verbatim}
\end{tcolorbox}

    \begin{Verbatim}[commandchars=\\\{\}]
Valores nulos por columnas:
    \end{Verbatim}

    
    \begin{Verbatim}[commandchars=\\\{\}]
ccaa                0
periodo             0
SMI\_VIDA            0
SMI\_MEDIO           0
EMPRESAS\_10         0
EMPRESAS\_20         0
EMPRESAS\_50         0
PARO\_1\_AÑO          0
EMP\_1\_5            26
PARO\_25             0
OC\_CONSTRUCCION     0
OC\_SERVICIOS        0
PARO                0
PIB\_CAPITA          0
PROD\_HORA           0
CARENCIA            0
RIESGO\_POBREZA      0
INC\_SMI\_REAL        0
DESIGUALDAD         0
PARCIAL             0
IPC                 0
HORAS\_TRABAJO       0
dtype: int64
    \end{Verbatim}

    
    Vemos que tenemos valores faltantes en EMP\_1\_5, esto se debe a que no
tenemos disponibles los valores para País Vasco y Navarra. Dado que el
resto de columnas está disponible y la cantidad de datos que tenemos es
escasa no es factible descartar las filas debido a la falta de estos
valores. En la siguiente sección donde estudiamos las correlaciones
veremos cuál es la manera conveniente de imputar estos valores.

Una vez con las variables principales a estudiar, observemos los
posibles outliers que estaremos tratando.

    \begin{tcolorbox}[breakable, size=fbox, boxrule=1pt, pad at break*=1mm,colback=cellbackground, colframe=cellborder]
\prompt{In}{incolor}{46}{\boxspacing}
\begin{Verbatim}[commandchars=\\\{\}]
\PY{n}{df\PYZus{}num} \PY{o}{=} \PY{n}{df}\PY{p}{[}\PY{p}{[}\PY{l+s+s1}{\PYZsq{}}\PY{l+s+s1}{SMI\PYZus{}VIDA}\PY{l+s+s1}{\PYZsq{}}\PY{p}{,} \PY{l+s+s1}{\PYZsq{}}\PY{l+s+s1}{SMI\PYZus{}MEDIO}\PY{l+s+s1}{\PYZsq{}}\PY{p}{,} \PY{l+s+s1}{\PYZsq{}}\PY{l+s+s1}{EMPRESAS\PYZus{}10}\PY{l+s+s1}{\PYZsq{}}\PY{p}{,} \PY{l+s+s1}{\PYZsq{}}\PY{l+s+s1}{EMPRESAS\PYZus{}20}\PY{l+s+s1}{\PYZsq{}}\PY{p}{,}
              \PY{l+s+s1}{\PYZsq{}}\PY{l+s+s1}{EMPRESAS\PYZus{}50}\PY{l+s+s1}{\PYZsq{}}\PY{p}{,} \PY{l+s+s1}{\PYZsq{}}\PY{l+s+s1}{PARO\PYZus{}1\PYZus{}AÑO}\PY{l+s+s1}{\PYZsq{}}\PY{p}{,} \PY{l+s+s1}{\PYZsq{}}\PY{l+s+s1}{EMP\PYZus{}1\PYZus{}5}\PY{l+s+s1}{\PYZsq{}}\PY{p}{,} \PY{l+s+s1}{\PYZsq{}}\PY{l+s+s1}{PARO\PYZus{}25}\PY{l+s+s1}{\PYZsq{}}\PY{p}{,} 
              \PY{l+s+s1}{\PYZsq{}}\PY{l+s+s1}{OC\PYZus{}CONSTRUCCION}\PY{l+s+s1}{\PYZsq{}}\PY{p}{,} \PY{l+s+s1}{\PYZsq{}}\PY{l+s+s1}{OC\PYZus{}SERVICIOS}\PY{l+s+s1}{\PYZsq{}}\PY{p}{,} \PY{l+s+s1}{\PYZsq{}}\PY{l+s+s1}{PARO}\PY{l+s+s1}{\PYZsq{}}\PY{p}{,} 
              \PY{l+s+s1}{\PYZsq{}}\PY{l+s+s1}{PIB\PYZus{}CAPITA}\PY{l+s+s1}{\PYZsq{}}\PY{p}{,} \PY{l+s+s1}{\PYZsq{}}\PY{l+s+s1}{PROD\PYZus{}HORA}\PY{l+s+s1}{\PYZsq{}}\PY{p}{,} \PY{l+s+s1}{\PYZsq{}}\PY{l+s+s1}{CARENCIA}\PY{l+s+s1}{\PYZsq{}}\PY{p}{,} \PY{l+s+s1}{\PYZsq{}}\PY{l+s+s1}{RIESGO\PYZus{}POBREZA}\PY{l+s+s1}{\PYZsq{}}\PY{p}{,} 
              \PY{l+s+s1}{\PYZsq{}}\PY{l+s+s1}{INC\PYZus{}SMI\PYZus{}REAL}\PY{l+s+s1}{\PYZsq{}}\PY{p}{,} \PY{l+s+s1}{\PYZsq{}}\PY{l+s+s1}{DESIGUALDAD}\PY{l+s+s1}{\PYZsq{}}\PY{p}{,} \PY{l+s+s1}{\PYZsq{}}\PY{l+s+s1}{PARCIAL}\PY{l+s+s1}{\PYZsq{}}\PY{p}{,} 
              \PY{l+s+s1}{\PYZsq{}}\PY{l+s+s1}{IPC}\PY{l+s+s1}{\PYZsq{}}\PY{p}{,} \PY{l+s+s1}{\PYZsq{}}\PY{l+s+s1}{HORAS\PYZus{}TRABAJO}\PY{l+s+s1}{\PYZsq{}}\PY{p}{]}\PY{p}{]}

\PY{n}{p}\PY{o}{.}\PY{n}{box\PYZus{}plot\PYZus{}var}\PY{p}{(}\PY{n}{df\PYZus{}num}\PY{p}{,}\PY{n}{nrows}\PY{o}{=}\PY{l+m+mi}{5}\PY{p}{,} \PY{n}{ncols}\PY{o}{=}\PY{l+m+mi}{4}\PY{p}{)}
\end{Verbatim}
\end{tcolorbox}

    \begin{center}
    \adjustimage{max size={0.9\linewidth}{0.9\paperheight}}{tfm_project_files/tfm_project_108_0.png}
    \end{center}
    { \hspace*{\fill} \\}
    
    En el presente estudio, se ha optado por no eliminar ni tratar los
outliers debido a que estos representan situaciones económicas reales y
relevantes. La presencia de valores atípicos en variables como las horas
trabajadas por empleo o la productividad por hora refleja disparidades
regionales o sectoriales importantes que pueden influir
significativamente en los efectos del salario mínimo. Excluir estos
datos podría llevar a una pérdida de información crítica y limitar la
capacidad del análisis para captar plenamente las dinámicas del mercado
laboral. Por ello, se consideran fundamentales para ofrecer una visión
más completa y realista del fenómeno estudiado.

    \subsection{Correlaciones y Selección de
Variables}\label{correlaciones-y-selecciuxf3n-de-variables}

    \subsubsection{Correlaciones}\label{correlaciones}

    Es importante notar que la cantidad de información es limitada.
Suponiendo que usaremos 17 comunidades autónomas y el periodo de
intersección de las variables de datos disponibles es entre 2008 y 2019,
tendríamos algo menos de 190 filas para llevar a cabo el análisis,
tomando además en cuenta que tendremos que hacer una separación test y
control. Por ello es de vital importancia reducir lo máximo posible el
número de variables empleadas para evitar problemas con la
dimensionalidad.

Empecemos analizando la correlación de pearson para entender qué
variables se mueven de manera más similar.

    \begin{tcolorbox}[breakable, size=fbox, boxrule=1pt, pad at break*=1mm,colback=cellbackground, colframe=cellborder]
\prompt{In}{incolor}{47}{\boxspacing}
\begin{Verbatim}[commandchars=\\\{\}]
\PY{c+c1}{\PYZsh{} Calcular la matriz de correlación}
\PY{n}{num\PYZus{}var} \PY{o}{=} \PY{p}{[}\PY{l+s+s1}{\PYZsq{}}\PY{l+s+s1}{SMI\PYZus{}VIDA}\PY{l+s+s1}{\PYZsq{}}\PY{p}{,} \PY{l+s+s1}{\PYZsq{}}\PY{l+s+s1}{SMI\PYZus{}MEDIO}\PY{l+s+s1}{\PYZsq{}}\PY{p}{,} \PY{l+s+s1}{\PYZsq{}}\PY{l+s+s1}{EMPRESAS\PYZus{}10}\PY{l+s+s1}{\PYZsq{}}\PY{p}{,} \PY{l+s+s1}{\PYZsq{}}\PY{l+s+s1}{EMPRESAS\PYZus{}20}\PY{l+s+s1}{\PYZsq{}}\PY{p}{,}
              \PY{l+s+s1}{\PYZsq{}}\PY{l+s+s1}{EMPRESAS\PYZus{}50}\PY{l+s+s1}{\PYZsq{}}\PY{p}{,} \PY{l+s+s1}{\PYZsq{}}\PY{l+s+s1}{PARO\PYZus{}1\PYZus{}AÑO}\PY{l+s+s1}{\PYZsq{}}\PY{p}{,} \PY{l+s+s1}{\PYZsq{}}\PY{l+s+s1}{EMP\PYZus{}1\PYZus{}5}\PY{l+s+s1}{\PYZsq{}}\PY{p}{,} \PY{l+s+s1}{\PYZsq{}}\PY{l+s+s1}{PARO\PYZus{}25}\PY{l+s+s1}{\PYZsq{}}\PY{p}{,} 
              \PY{l+s+s1}{\PYZsq{}}\PY{l+s+s1}{OC\PYZus{}CONSTRUCCION}\PY{l+s+s1}{\PYZsq{}}\PY{p}{,} \PY{l+s+s1}{\PYZsq{}}\PY{l+s+s1}{OC\PYZus{}SERVICIOS}\PY{l+s+s1}{\PYZsq{}}\PY{p}{,} \PY{l+s+s1}{\PYZsq{}}\PY{l+s+s1}{PARO}\PY{l+s+s1}{\PYZsq{}}\PY{p}{,} 
              \PY{l+s+s1}{\PYZsq{}}\PY{l+s+s1}{PIB\PYZus{}CAPITA}\PY{l+s+s1}{\PYZsq{}}\PY{p}{,} \PY{l+s+s1}{\PYZsq{}}\PY{l+s+s1}{PROD\PYZus{}HORA}\PY{l+s+s1}{\PYZsq{}}\PY{p}{,} \PY{l+s+s1}{\PYZsq{}}\PY{l+s+s1}{CARENCIA}\PY{l+s+s1}{\PYZsq{}}\PY{p}{,} 
              \PY{l+s+s1}{\PYZsq{}}\PY{l+s+s1}{RIESGO\PYZus{}POBREZA}\PY{l+s+s1}{\PYZsq{}}\PY{p}{,} \PY{l+s+s1}{\PYZsq{}}\PY{l+s+s1}{INC\PYZus{}SMI\PYZus{}REAL}\PY{l+s+s1}{\PYZsq{}}\PY{p}{,}
              \PY{l+s+s1}{\PYZsq{}}\PY{l+s+s1}{DESIGUALDAD}\PY{l+s+s1}{\PYZsq{}}\PY{p}{,} \PY{l+s+s1}{\PYZsq{}}\PY{l+s+s1}{PARCIAL}\PY{l+s+s1}{\PYZsq{}}\PY{p}{,} \PY{l+s+s1}{\PYZsq{}}\PY{l+s+s1}{IPC}\PY{l+s+s1}{\PYZsq{}}\PY{p}{,} \PY{l+s+s1}{\PYZsq{}}\PY{l+s+s1}{HORAS\PYZus{}TRABAJO}\PY{l+s+s1}{\PYZsq{}}\PY{p}{]}
\PY{n}{p}\PY{o}{.}\PY{n}{creat\PYZus{}corr\PYZus{}matrix}\PY{p}{(}\PY{n}{df}\PY{p}{,} \PY{n}{num\PYZus{}var}\PY{p}{)}
\end{Verbatim}
\end{tcolorbox}

    \begin{center}
    \adjustimage{max size={0.9\linewidth}{0.9\paperheight}}{tfm_project_files/tfm_project_113_0.png}
    \end{center}
    { \hspace*{\fill} \\}
    
    A priori observamos varias correlaciones fuertes entre diversas
variables. Se observa una correlación inversa esperada entre SMI\_MEDIO
y SMI\_VIDA, ya que uno es proporcional al SMI y el otro es inversamente
proporcional. Se observa también algunas correlaciones no esperadas,
como lo es la correlación entre la productividad por hora y el IPC, así
como la correlación entre IPC y porcentaje de trabajadores a jornada
parcial (PARCIAL).

Sin embargo, lo interesante es observar cómo afecta una variación del
salario mínimo a la variación de las variables año contra año que es lo
que veremos a continuación. No obstante, recordemos que una de las
variaciones de interés a observar es cómo afecta la variación del
salario mínimo al resto de variables.

Antes de eso, y aprovechando las múltiples correlaciones que se dan en
las variables haremos una imputación de la variable EMP\_1\_5 mediante
vecinos cercanos.

    \begin{tcolorbox}[breakable, size=fbox, boxrule=1pt, pad at break*=1mm,colback=cellbackground, colframe=cellborder]
\prompt{In}{incolor}{48}{\boxspacing}
\begin{Verbatim}[commandchars=\\\{\}]
\PY{c+c1}{\PYZsh{} Creamos el imputador}
\PY{n}{knn\PYZus{}imputer} \PY{o}{=} \PY{n}{KNNImputer}\PY{p}{(}\PY{n}{n\PYZus{}neighbors}\PY{o}{=}\PY{l+m+mi}{2}\PY{p}{)}

\PY{c+c1}{\PYZsh{} Realizamos la imputación}
\PY{n}{data\PYZus{}numeric} \PY{o}{=} \PY{n}{df}\PY{o}{.}\PY{n}{drop}\PY{p}{(}\PY{n}{columns}\PY{o}{=}\PY{p}{[}\PY{l+s+s2}{\PYZdq{}}\PY{l+s+s2}{ccaa}\PY{l+s+s2}{\PYZdq{}}\PY{p}{,} \PY{l+s+s2}{\PYZdq{}}\PY{l+s+s2}{periodo}\PY{l+s+s2}{\PYZdq{}}\PY{p}{]}\PY{p}{)}  \PY{c+c1}{\PYZsh{} Excluir columnas no numéricas}
\PY{n}{data\PYZus{}imputed} \PY{o}{=} \PY{n}{knn\PYZus{}imputer}\PY{o}{.}\PY{n}{fit\PYZus{}transform}\PY{p}{(}\PY{n}{data\PYZus{}numeric}\PY{p}{)}

\PY{c+c1}{\PYZsh{} Convertir de nuevo a DataFrame}
\PY{n}{data\PYZus{}imputed\PYZus{}df} \PY{o}{=} \PY{n}{pd}\PY{o}{.}\PY{n}{DataFrame}\PY{p}{(}\PY{n}{data\PYZus{}imputed}\PY{p}{,} \PY{n}{columns}\PY{o}{=}\PY{n}{data\PYZus{}numeric}\PY{o}{.}\PY{n}{columns}\PY{p}{)}
\PY{n}{data\PYZus{}imputed\PYZus{}df}\PY{p}{[}\PY{l+s+s2}{\PYZdq{}}\PY{l+s+s2}{ccaa}\PY{l+s+s2}{\PYZdq{}}\PY{p}{]} \PY{o}{=} \PY{n}{df}\PY{p}{[}\PY{l+s+s2}{\PYZdq{}}\PY{l+s+s2}{ccaa}\PY{l+s+s2}{\PYZdq{}}\PY{p}{]}\PY{o}{.}\PY{n}{values}
\PY{n}{data\PYZus{}imputed\PYZus{}df}\PY{p}{[}\PY{l+s+s2}{\PYZdq{}}\PY{l+s+s2}{periodo}\PY{l+s+s2}{\PYZdq{}}\PY{p}{]} \PY{o}{=} \PY{n}{df}\PY{p}{[}\PY{l+s+s2}{\PYZdq{}}\PY{l+s+s2}{periodo}\PY{l+s+s2}{\PYZdq{}}\PY{p}{]}\PY{o}{.}\PY{n}{values}

\PY{n}{df} \PY{o}{=} \PY{n}{data\PYZus{}imputed\PYZus{}df}\PY{o}{.}\PY{n}{copy}\PY{p}{(}\PY{p}{)}
\end{Verbatim}
\end{tcolorbox}

    \begin{tcolorbox}[breakable, size=fbox, boxrule=1pt, pad at break*=1mm,colback=cellbackground, colframe=cellborder]
\prompt{In}{incolor}{49}{\boxspacing}
\begin{Verbatim}[commandchars=\\\{\}]
\PY{n}{num\PYZus{}var\PYZus{}delta} \PY{o}{=} \PY{p}{[}\PY{l+s+s1}{\PYZsq{}}\PY{l+s+s1}{SMI\PYZus{}VIDA}\PY{l+s+s1}{\PYZsq{}}\PY{p}{,} \PY{l+s+s1}{\PYZsq{}}\PY{l+s+s1}{SMI\PYZus{}MEDIO}\PY{l+s+s1}{\PYZsq{}}\PY{p}{,} \PY{l+s+s1}{\PYZsq{}}\PY{l+s+s1}{EMPRESAS\PYZus{}10}\PY{l+s+s1}{\PYZsq{}}\PY{p}{,} \PY{l+s+s1}{\PYZsq{}}\PY{l+s+s1}{EMPRESAS\PYZus{}20}\PY{l+s+s1}{\PYZsq{}}\PY{p}{,}
              \PY{l+s+s1}{\PYZsq{}}\PY{l+s+s1}{EMPRESAS\PYZus{}50}\PY{l+s+s1}{\PYZsq{}}\PY{p}{,} \PY{l+s+s1}{\PYZsq{}}\PY{l+s+s1}{PARO\PYZus{}1\PYZus{}AÑO}\PY{l+s+s1}{\PYZsq{}}\PY{p}{,} \PY{l+s+s1}{\PYZsq{}}\PY{l+s+s1}{EMP\PYZus{}1\PYZus{}5}\PY{l+s+s1}{\PYZsq{}}\PY{p}{,} \PY{l+s+s1}{\PYZsq{}}\PY{l+s+s1}{PARO\PYZus{}25}\PY{l+s+s1}{\PYZsq{}}\PY{p}{,} \PY{l+s+s1}{\PYZsq{}}\PY{l+s+s1}{OC\PYZus{}CONSTRUCCION}\PY{l+s+s1}{\PYZsq{}}\PY{p}{,} \PY{l+s+s1}{\PYZsq{}}\PY{l+s+s1}{OC\PYZus{}SERVICIOS}\PY{l+s+s1}{\PYZsq{}}\PY{p}{,}
              \PY{l+s+s1}{\PYZsq{}}\PY{l+s+s1}{PARO}\PY{l+s+s1}{\PYZsq{}}\PY{p}{,} \PY{l+s+s1}{\PYZsq{}}\PY{l+s+s1}{PIB\PYZus{}CAPITA}\PY{l+s+s1}{\PYZsq{}}\PY{p}{,} \PY{l+s+s1}{\PYZsq{}}\PY{l+s+s1}{PROD\PYZus{}HORA}\PY{l+s+s1}{\PYZsq{}}\PY{p}{,} \PY{l+s+s1}{\PYZsq{}}\PY{l+s+s1}{CARENCIA}\PY{l+s+s1}{\PYZsq{}}\PY{p}{,} \PY{l+s+s1}{\PYZsq{}}\PY{l+s+s1}{RIESGO\PYZus{}POBREZA}\PY{l+s+s1}{\PYZsq{}}\PY{p}{,}
              \PY{l+s+s1}{\PYZsq{}}\PY{l+s+s1}{DESIGUALDAD}\PY{l+s+s1}{\PYZsq{}}\PY{p}{,} \PY{l+s+s1}{\PYZsq{}}\PY{l+s+s1}{PARCIAL}\PY{l+s+s1}{\PYZsq{}}\PY{p}{,} \PY{l+s+s1}{\PYZsq{}}\PY{l+s+s1}{IPC}\PY{l+s+s1}{\PYZsq{}}\PY{p}{,} \PY{l+s+s1}{\PYZsq{}}\PY{l+s+s1}{HORAS\PYZus{}TRABAJO}\PY{l+s+s1}{\PYZsq{}}\PY{p}{]}
\PY{n}{df\PYZus{}delta} \PY{o}{=} \PY{n}{df}\PY{o}{.}\PY{n}{copy}\PY{p}{(}\PY{p}{)}
\PY{k}{for} \PY{n}{var} \PY{o+ow}{in} \PY{n}{num\PYZus{}var\PYZus{}delta}\PY{p}{:}
    \PY{n}{df\PYZus{}delta} \PY{o}{=} \PY{n}{dformat}\PY{o}{.}\PY{n}{atrasar\PYZus{}año}\PY{p}{(}\PY{n}{df\PYZus{}delta}\PY{p}{,} \PY{n}{var}\PY{p}{,} \PY{n}{year\PYZus{}col}\PY{o}{=}\PY{l+s+s1}{\PYZsq{}}\PY{l+s+s1}{periodo}\PY{l+s+s1}{\PYZsq{}}\PY{p}{,} \PY{n}{region\PYZus{}col}\PY{o}{=}\PY{l+s+s1}{\PYZsq{}}\PY{l+s+s1}{ccaa}\PY{l+s+s1}{\PYZsq{}}\PY{p}{,} \PY{n}{periodos}\PY{o}{=}\PY{l+m+mi}{1}\PY{p}{,} \PY{n}{calc\PYZus{}delta}\PY{o}{=}\PY{k+kc}{True}\PY{p}{,} \PY{n}{drop\PYZus{}period\PYZus{}var}\PY{o}{=}\PY{k+kc}{True}\PY{p}{)}
\end{Verbatim}
\end{tcolorbox}

    \begin{tcolorbox}[breakable, size=fbox, boxrule=1pt, pad at break*=1mm,colback=cellbackground, colframe=cellborder]
\prompt{In}{incolor}{50}{\boxspacing}
\begin{Verbatim}[commandchars=\\\{\}]
\PY{n}{var\PYZus{}delta} \PY{o}{=} \PY{p}{[}\PY{l+s+s1}{\PYZsq{}}\PY{l+s+s1}{INC\PYZus{}SMI\PYZus{}REAL}\PY{l+s+s1}{\PYZsq{}}\PY{p}{]}\PY{o}{+}\PY{p}{[}\PY{n}{var} \PY{k}{for} \PY{n}{var} \PY{o+ow}{in} \PY{n}{df\PYZus{}delta}\PY{o}{.}\PY{n}{columns} \PY{k}{if} \PY{l+s+s1}{\PYZsq{}}\PY{l+s+s1}{delta}\PY{l+s+s1}{\PYZsq{}} \PY{o+ow}{in} \PY{n}{var}\PY{p}{]}
\PY{n}{p}\PY{o}{.}\PY{n}{creat\PYZus{}corr\PYZus{}matrix}\PY{p}{(}\PY{n}{df\PYZus{}delta}\PY{p}{,} \PY{n}{var\PYZus{}delta}\PY{p}{)}
\end{Verbatim}
\end{tcolorbox}

    \begin{center}
    \adjustimage{max size={0.9\linewidth}{0.9\paperheight}}{tfm_project_files/tfm_project_117_0.png}
    \end{center}
    { \hspace*{\fill} \\}
    
    Aunque las correlaciones entre incrementos puede ser interesante, la
única que nos interesa estudiar es la del salario mínimo contra el
resto, pues es la única que podremos tocar para influir en los
resultados del año que viene. Se aprecia de manera clara que la
correlación de esta con las variables que no están relacionadas o
construidas a partir del salario mínimo es relativamente baja.

    \subsubsection{Selección de variables}\label{selecciuxf3n-de-variables}

    Una vez observadas las correlaciones con las variables, procedamos a la
selección de las más importantes. Como hemos mencionado, la cantidad de
filas disponibles es escasa, por tanto es preciso seleccionar el
conjunto más reducido posible de variables.

Para ello y dadas las correlaciones existentes utilizaremos el PCA para
hacer nuestra selección de variables. Compararemos este método con la
importancia de variables proporcionadas por el algoritmo random forest,
que nos aportará información también en relación a las variables
dependientes (que de base será la variación año contra año de todas las
variables estudiadas).

    \begin{tcolorbox}[breakable, size=fbox, boxrule=1pt, pad at break*=1mm,colback=cellbackground, colframe=cellborder]
\prompt{In}{incolor}{51}{\boxspacing}
\begin{Verbatim}[commandchars=\\\{\}]
\PY{n}{df\PYZus{}num}\PY{o}{=} \PY{n}{df}\PY{p}{[}\PY{n}{num\PYZus{}var}\PY{p}{]}\PY{o}{.}\PY{n}{copy}\PY{p}{(}\PY{p}{)}
\PY{n}{smod}\PY{o}{.}\PY{n}{obtener\PYZus{}importancia\PYZus{}variables}\PY{p}{(}\PY{n}{df\PYZus{}delta}\PY{p}{[}\PY{n}{var\PYZus{}delta}\PY{p}{]}\PY{o}{.}\PY{n}{dropna}\PY{p}{(}\PY{p}{)}\PY{p}{,} \PY{n}{umbral\PYZus{}varianza}\PY{o}{=}\PY{l+m+mf}{0.8}\PY{p}{)}
\end{Verbatim}
\end{tcolorbox}

    \begin{Verbatim}[commandchars=\\\{\}]
Número de componentes seleccionados: 8
Varianza explicada por estos componentes:
Componente 1: 27.70\%
Componente 2: 15.12\%
Componente 3: 10.30\%
Componente 4: 7.34\%
Componente 5: 6.87\%
Componente 6: 5.26\%
Componente 7: 4.26\%
Componente 8: 3.92\%
Varianza acumulada: 80.76\%
    \end{Verbatim}

    \begin{center}
    \adjustimage{max size={0.9\linewidth}{0.9\paperheight}}{tfm_project_files/tfm_project_121_1.png}
    \end{center}
    { \hspace*{\fill} \\}
    
            \begin{tcolorbox}[breakable, size=fbox, boxrule=.5pt, pad at break*=1mm, opacityfill=0]
\prompt{Out}{outcolor}{51}{\boxspacing}
\begin{Verbatim}[commandchars=\\\{\}]
                  Variable  Importancia
14         CARENCIA\_delta1     0.093148
15   RIESGO\_POBREZA\_delta1     0.072723
18              IPC\_delta1     0.072462
3       EMPRESAS\_10\_delta1     0.069384
16      DESIGUALDAD\_delta1     0.069039
19    HORAS\_TRABAJO\_delta1     0.055056
4       EMPRESAS\_20\_delta1     0.053449
10     OC\_SERVICIOS\_delta1     0.049254
12       PIB\_CAPITA\_delta1     0.047703
13        PROD\_HORA\_delta1     0.046652
7           EMP\_1\_5\_delta1     0.046352
9   OC\_CONSTRUCCION\_delta1     0.045926
17          PARCIAL\_delta1     0.042757
5       EMPRESAS\_50\_delta1     0.037889
2         SMI\_MEDIO\_delta1     0.037884
0             INC\_SMI\_REAL     0.035816
1          SMI\_VIDA\_delta1     0.035365
8           PARO\_25\_delta1     0.030980
6        PARO\_1\_AÑO\_delta1     0.029129
11             PARO\_delta1     0.029032
\end{Verbatim}
\end{tcolorbox}
        
    Observamos que CARENCIA es la variable más importante, ya que no tiene
correlación prácticamente con ninguna otra variable, similar a lo que
ocurre con los siguientes puestos en importancia como es la variación
del IPC y del riesgo de pobreza.

Comparemos ahora el resultado con la importancia de las variables dentro
del árbol de decisión.

    \begin{tcolorbox}[breakable, size=fbox, boxrule=1pt, pad at break*=1mm,colback=cellbackground, colframe=cellborder]
\prompt{In}{incolor}{52}{\boxspacing}
\begin{Verbatim}[commandchars=\\\{\}]
\PY{c+c1}{\PYZsh{}Las variables X serán las seleccionadas originalmente y las y serán los incrementos de cara al año siguiente, }
\PY{c+c1}{\PYZsh{} solo podemos coger hasta 2019 porque si no tendremos nulos}
\PY{n}{cond} \PY{o}{=} \PY{n}{df\PYZus{}delta}\PY{p}{[}\PY{l+s+s1}{\PYZsq{}}\PY{l+s+s1}{periodo}\PY{l+s+s1}{\PYZsq{}}\PY{p}{]}\PY{o}{\PYZlt{}}\PY{o}{=}\PY{l+m+mi}{2019}
\PY{n}{X} \PY{o}{=} \PY{n}{df\PYZus{}delta}\PY{p}{[}\PY{n}{cond}\PY{p}{]}\PY{p}{[}\PY{n}{num\PYZus{}var}\PY{p}{]}
\PY{n}{y} \PY{o}{=} \PY{n}{df\PYZus{}delta}\PY{p}{[}\PY{n}{cond}\PY{p}{]}\PY{p}{[}\PY{p}{[}\PY{n}{var} \PY{k}{for} \PY{n}{var} \PY{o+ow}{in} \PY{n}{df\PYZus{}delta}\PY{o}{.}\PY{n}{columns} \PY{k}{if} \PY{l+s+s1}{\PYZsq{}}\PY{l+s+s1}{delta}\PY{l+s+s1}{\PYZsq{}} \PY{o+ow}{in} \PY{n}{var}\PY{p}{]}\PY{p}{]}
\PY{n}{df\PYZus{}importancia\PYZus{}rf}\PY{p}{,} \PY{n}{importancia\PYZus{}variables}\PY{p}{,} \PY{n}{variables\PYZus{}importantes} \PY{o}{=} \PY{n}{smod}\PY{o}{.}\PY{n}{obtener\PYZus{}importancia\PYZus{}variables\PYZus{}rf}\PY{p}{(}\PY{n}{X}\PY{p}{,}\PY{n}{y}\PY{p}{,} \PY{n}{mostrar\PYZus{}subplots}\PY{o}{=}\PY{k+kc}{True}\PY{p}{,} 
                                                                                                   \PY{n}{umbral\PYZus{}importancia}\PY{o}{=}\PY{l+m+mf}{0.95}\PY{p}{,} \PY{n}{num\PYZus{}variables}\PY{o}{=}\PY{l+m+mi}{8}\PY{p}{,} \PY{n}{columnas\PYZus{}subplots} \PY{o}{=} \PY{l+m+mi}{2}\PY{p}{,}
                                                                                                   \PY{n}{variables\PYZus{}forzadas}\PY{o}{=} \PY{p}{[}\PY{l+s+s1}{\PYZsq{}}\PY{l+s+s1}{INC\PYZus{}SMI\PYZus{}REAL}\PY{l+s+s1}{\PYZsq{}}\PY{p}{]}\PY{p}{)}
\end{Verbatim}
\end{tcolorbox}

    \begin{Verbatim}[commandchars=\\\{\}]
Variable añadida para la columna:  RIESGO\_POBREZA\_delta1
Variable añadida para la columna:  PARCIAL\_delta1
    \end{Verbatim}

    \begin{center}
    \adjustimage{max size={0.9\linewidth}{0.9\paperheight}}{tfm_project_files/tfm_project_123_1.png}
    \end{center}
    { \hspace*{\fill} \\}
    
    \begin{center}
    \adjustimage{max size={0.9\linewidth}{0.9\paperheight}}{tfm_project_files/tfm_project_123_2.png}
    \end{center}
    { \hspace*{\fill} \\}
    
    Vemos que ahora la importancia es mucho más variable y diferenciada,
tanto el promedio final como el la distribución para cada una de las
variables objetivo. Se observa que en varios casos la variable más
importante para la predicción del incremento de una variable es la
propia variable.

Dado el Random Forest nos ofrece una selección más adecuada a la
variable objetivo a predecir, nos quedaremos con este modelo para
escoger las variables.

La diferencia de presencia para cada variable en las diferentes
predicciones nos forzará a crear un modelo con diferentes variables para
cada predicción que vayamos a hacer. Para cada variable nos quedaremos
con el conjunto de variables que supongan el 95\% de la importancia de
cada variable o las 8 más importantes, lo que llegue primero. Se añadirá
la variable INC\_SMI\_REAL en caso de que no aparezca incluida en este
conjunto de variables.

    \subsection{Modelos}\label{modelos}

    A continuación se hará la selección del modelo a emplear para la
predicción de las características utilizadas. Estudiaremos 6 posibles
modelos, que evaluaremos mediante validación cruzada para cada una de
las variables que queremos predecir.

    \subsubsection{Regresión Lineal}\label{regresiuxf3n-lineal}

    La regresión lineal es uno de los modelos más simples y ampliamente
utilizados en el análisis de relaciones entre variables. En este
trabajo, se utiliza como un modelo base para explorar la relación entre
las variaciones en el salario mínimo y su impacto en variables
económicas como el PIB o la carencia material. A pesar de su
simplicidad, la regresión lineal permite establecer una primera
aproximación a los datos y sirve como referencia para comparar el
rendimiento de modelos más complejos.

La regresión lineal asume que existe una relación lineal entre la
variable dependiente y una o más variables independientes. El modelo
busca ajustar una recta que minimice la suma de los errores cuadráticos,
proporcionando un coeficiente que indica la fuerza y dirección de esta
relación. Es un modelo fácil de interpretar y útil para obtener una
visión inicial de cómo los cambios en el salario mínimo pueden influir
en otras variables económicas.

    \begin{tcolorbox}[breakable, size=fbox, boxrule=1pt, pad at break*=1mm,colback=cellbackground, colframe=cellborder]
\prompt{In}{incolor}{53}{\boxspacing}
\begin{Verbatim}[commandchars=\\\{\}]
\PY{n}{results\PYZus{}df\PYZus{}r} \PY{o}{=} \PY{n}{em}\PY{o}{.}\PY{n}{evaluacion\PYZus{}modelo\PYZus{}simple}\PY{p}{(}\PY{n}{X}\PY{p}{,} \PY{n}{y}\PY{p}{,} 
                                           \PY{n}{variables\PYZus{}importantes}\PY{p}{,} 
                                           \PY{n}{LinearRegression}\PY{p}{(}\PY{p}{)}\PY{p}{)}
\PY{n}{display}\PY{p}{(}\PY{n}{results\PYZus{}df\PYZus{}r}\PY{p}{)}
\end{Verbatim}
\end{tcolorbox}

    
    \begin{Verbatim}[commandchars=\\\{\}]
         Variable Objetivo   Mean R²
0          SMI\_VIDA\_delta1  0.717986
1         SMI\_MEDIO\_delta1  0.805572
2       EMPRESAS\_10\_delta1  0.068285
3       EMPRESAS\_20\_delta1  0.412015
4       EMPRESAS\_50\_delta1  0.472386
5        PARO\_1\_AÑO\_delta1  0.731343
6           EMP\_1\_5\_delta1  0.278892
7           PARO\_25\_delta1  0.564756
8   OC\_CONSTRUCCION\_delta1  0.272410
9      OC\_SERVICIOS\_delta1  0.239297
10             PARO\_delta1  0.659321
11       PIB\_CAPITA\_delta1  0.383170
12        PROD\_HORA\_delta1  0.401261
13         CARENCIA\_delta1 -0.005807
14   RIESGO\_POBREZA\_delta1  0.035480
15      DESIGUALDAD\_delta1  0.103064
16          PARCIAL\_delta1  0.165365
17              IPC\_delta1  0.203873
18    HORAS\_TRABAJO\_delta1  0.402046
    \end{Verbatim}

    
    Vemos que para este modelo los resultados en términos de coeficientes R2
son relativamente bajos para la mayoría de las variables, excepto para
SMI\_MEDIO, pues tiene una correlación muy directa con el SMI.

    \subsubsection{Regresión Lasso}\label{regresiuxf3n-lasso}

    Dado que el conjunto de datos utilizado en este trabajo es relativamente
pequeño (\textasciitilde200 observaciones), existe el riesgo de
sobreajuste, lo que podría afectar la capacidad de generalización del
modelo. La regresión Ridge y Lasso son variantes de la regresión lineal
que incorporan técnicas de regularización, lo que ayuda a mitigar el
sobreajuste y a mejorar la estabilidad y precisión del modelo,
especialmente cuando se dispone de múltiples características.

La regresión Ridge y Lasso son formas de regresión lineal que aplican
penalizaciones a los coeficientes de las variables para reducir la
complejidad del modelo. La regresión Ridge utiliza una penalización L2,
que limita el tamaño de los coeficientes, mientras que la regresión
Lasso aplica una penalización L1, que no solo reduce el tamaño de los
coeficientes, sino que también puede hacer que algunos se vuelvan cero,
eliminando así variables menos relevantes. Estas técnicas ayudan a
mejorar la generalización del modelo y a reducir la multicolinealidad,
lo que las hace especialmente útiles cuando el número de variables es
elevado.

    \begin{tcolorbox}[breakable, size=fbox, boxrule=1pt, pad at break*=1mm,colback=cellbackground, colframe=cellborder]
\prompt{In}{incolor}{54}{\boxspacing}
\begin{Verbatim}[commandchars=\\\{\}]
\PY{n}{results\PYZus{}df\PYZus{}rl} \PY{o}{=} \PY{n}{em}\PY{o}{.}\PY{n}{evaluacion\PYZus{}modelo\PYZus{}simple}\PY{p}{(}\PY{n}{X}\PY{p}{,} \PY{n}{y}\PY{p}{,} 
                                            \PY{n}{variables\PYZus{}importantes}\PY{p}{,} 
                                            \PY{n}{Lasso}\PY{p}{(}\PY{p}{)}\PY{p}{)}
\PY{n}{display}\PY{p}{(}\PY{n}{results\PYZus{}df\PYZus{}rl}\PY{p}{[}\PY{p}{[}\PY{l+s+s1}{\PYZsq{}}\PY{l+s+s1}{Variable Objetivo}\PY{l+s+s1}{\PYZsq{}}\PY{p}{,} \PY{l+s+s1}{\PYZsq{}}\PY{l+s+s1}{Mean R²}\PY{l+s+s1}{\PYZsq{}}\PY{p}{]}\PY{p}{]}\PY{p}{)}
\end{Verbatim}
\end{tcolorbox}

    
    \begin{Verbatim}[commandchars=\\\{\}]
         Variable Objetivo   Mean R²
0          SMI\_VIDA\_delta1 -0.041823
1         SMI\_MEDIO\_delta1 -0.052622
2       EMPRESAS\_10\_delta1 -0.020857
3       EMPRESAS\_20\_delta1 -0.036210
4       EMPRESAS\_50\_delta1 -0.030413
5        PARO\_1\_AÑO\_delta1  0.349116
6           EMP\_1\_5\_delta1 -0.022176
7           PARO\_25\_delta1  0.340407
8   OC\_CONSTRUCCION\_delta1 -0.011867
9      OC\_SERVICIOS\_delta1 -0.009050
10             PARO\_delta1  0.461257
11       PIB\_CAPITA\_delta1 -0.046437
12        PROD\_HORA\_delta1 -0.001072
13         CARENCIA\_delta1 -0.070551
14   RIESGO\_POBREZA\_delta1 -0.003718
15      DESIGUALDAD\_delta1 -0.053224
16          PARCIAL\_delta1 -0.018889
17              IPC\_delta1 -0.046092
18    HORAS\_TRABAJO\_delta1 -0.023125
    \end{Verbatim}

    
    Vemos que la regesión Lasso apenas aporta mejora al resultado final, por
lo que a priori el problema no es el sobreajuste.

    \subsubsection{Árbol de decisión}\label{uxe1rbol-de-decisiuxf3n}

Los árboles de decisión son una herramienta poderosa para modelar
relaciones no lineales y complejas entre variables. En el contexto de
este trabajo, se emplean para capturar interacciones no lineales entre
el salario mínimo y otros factores económicos. Aunque los árboles de
decisión pueden ser propensos al sobreajuste, su simplicidad y capacidad
para modelar relaciones complejas los convierte en una opción válida en
este análisis.

Un árbol de decisión es un modelo que divide el espacio de
características en diferentes regiones basándose en preguntas binarias
sobre las variables. Cada nodo del árbol representa una pregunta sobre
una variable, y las ramas representan las respuestas posibles. El
objetivo es segmentar los datos de manera que cada región tenga
características similares en cuanto a la variable dependiente. Este
modelo es fácil de interpretar y puede capturar relaciones no lineales
entre las variables, lo que lo hace adecuado cuando se espera que las
variables interactúen de manera compleja.

    \begin{tcolorbox}[breakable, size=fbox, boxrule=1pt, pad at break*=1mm,colback=cellbackground, colframe=cellborder]
\prompt{In}{incolor}{55}{\boxspacing}
\begin{Verbatim}[commandchars=\\\{\}]
\PY{c+c1}{\PYZsh{}Definimos los parámetros a probar en el gridsearch}
\PY{n}{param\PYZus{}grid\PYZus{}ad} \PY{o}{=} \PY{p}{\PYZob{}}
    \PY{l+s+s1}{\PYZsq{}}\PY{l+s+s1}{max\PYZus{}depth}\PY{l+s+s1}{\PYZsq{}}\PY{p}{:} \PY{p}{[}\PY{l+m+mi}{5}\PY{p}{,} \PY{l+m+mi}{10}\PY{p}{,} \PY{l+m+mi}{15}\PY{p}{,} \PY{k+kc}{None}\PY{p}{]}\PY{p}{,}  \PY{c+c1}{\PYZsh{} Profundidad máxima del árbol}
    \PY{l+s+s1}{\PYZsq{}}\PY{l+s+s1}{min\PYZus{}samples\PYZus{}split}\PY{l+s+s1}{\PYZsq{}}\PY{p}{:} \PY{p}{[}\PY{l+m+mi}{2}\PY{p}{,} \PY{l+m+mi}{5}\PY{p}{,} \PY{l+m+mi}{10}\PY{p}{]}\PY{p}{,}  \PY{c+c1}{\PYZsh{} Número mínimo de muestras para dividir un nodo}
    \PY{l+s+s1}{\PYZsq{}}\PY{l+s+s1}{min\PYZus{}samples\PYZus{}leaf}\PY{l+s+s1}{\PYZsq{}}\PY{p}{:} \PY{p}{[}\PY{l+m+mi}{1}\PY{p}{,} \PY{l+m+mi}{2}\PY{p}{,} \PY{l+m+mi}{4}\PY{p}{]}\PY{p}{,}  \PY{c+c1}{\PYZsh{} Número mínimo de muestras por hoja}
    \PY{l+s+s1}{\PYZsq{}}\PY{l+s+s1}{criterion}\PY{l+s+s1}{\PYZsq{}}\PY{p}{:} \PY{p}{[}\PY{l+s+s1}{\PYZsq{}}\PY{l+s+s1}{squared\PYZus{}error}\PY{l+s+s1}{\PYZsq{}}\PY{p}{,} \PY{l+s+s1}{\PYZsq{}}\PY{l+s+s1}{friedman\PYZus{}mse}\PY{l+s+s1}{\PYZsq{}}\PY{p}{,} \PY{l+s+s1}{\PYZsq{}}\PY{l+s+s1}{absolute\PYZus{}error}\PY{l+s+s1}{\PYZsq{}}\PY{p}{]}\PY{p}{,}  \PY{c+c1}{\PYZsh{} Criterios de división}
\PY{p}{\PYZcb{}}
\PY{n}{results\PYZus{}df\PYZus{}ad}\PY{p}{,} \PY{n}{best\PYZus{}params\PYZus{}dict\PYZus{}ad} \PY{o}{=} \PY{n}{em}\PY{o}{.}\PY{n}{evaluacion\PYZus{}modelo}\PY{p}{(}\PY{n}{X}\PY{p}{,}\PY{n}{y}\PY{p}{,}\PY{n}{variables\PYZus{}importantes}\PY{p}{,} \PY{n}{DecisionTreeRegressor}\PY{p}{(}\PY{p}{)}\PY{p}{,} \PY{n}{param\PYZus{}grid\PYZus{}ad}\PY{p}{)}
\PY{c+c1}{\PYZsh{} Mostrar la tabla de resultados}
\PY{n}{display}\PY{p}{(}\PY{n}{results\PYZus{}df\PYZus{}ad}\PY{p}{)}
\end{Verbatim}
\end{tcolorbox}

    
    \begin{Verbatim}[commandchars=\\\{\}]
         Variable Objetivo   Best R²
0          SMI\_VIDA\_delta1  0.704378
1         SMI\_MEDIO\_delta1  0.909806
2       EMPRESAS\_10\_delta1  0.447868
3       EMPRESAS\_20\_delta1  0.482855
4       EMPRESAS\_50\_delta1  0.568933
5        PARO\_1\_AÑO\_delta1  0.728087
6           EMP\_1\_5\_delta1  0.266905
7           PARO\_25\_delta1  0.643060
8   OC\_CONSTRUCCION\_delta1  0.042178
9      OC\_SERVICIOS\_delta1  0.089477
10             PARO\_delta1  0.796417
11       PIB\_CAPITA\_delta1  0.829071
12        PROD\_HORA\_delta1  0.245204
13         CARENCIA\_delta1 -0.299569
14   RIESGO\_POBREZA\_delta1 -0.056993
15      DESIGUALDAD\_delta1 -0.155428
16          PARCIAL\_delta1 -0.211729
17              IPC\_delta1  0.932263
18    HORAS\_TRABAJO\_delta1  0.741912
    \end{Verbatim}

    
    Vemos que ahora obtenemos mejoras sustanciales en variables como
PIB\_CAPITA e IPC, pero en otros como la carencia material y el riesgo
de pobreza los resultados empeoran.

    \subsubsection{Random Forest}\label{random-forest}

    El modelo Random Forest se utiliza para mejorar la precisión de los
árboles de decisión individuales al combinarlos en un conjunto de
árboles. Este enfoque reduce el riesgo de sobreajuste y mejora la
capacidad de generalización del modelo. Dado que el conjunto de datos es
relativamente pequeño, Random Forest ofrece una mayor robustez y
precisión en la predicción del impacto del salario mínimo sobre las
variables económicas, al combinar las predicciones de múltiples árboles
de decisión entrenados con diferentes subconjuntos de los datos.

Random Forest es un modelo de aprendizaje basado en el ensamblaje de
múltiples árboles de decisión. Cada árbol se entrena utilizando una
muestra aleatoria de los datos y un subconjunto aleatorio de
características, lo que introduce variabilidad y reduce el sobreajuste.
La predicción final del modelo se obtiene mediante el promedio de las
predicciones de todos los árboles. Este enfoque mejora la precisión y
estabilidad del modelo al aprovechar la diversidad de los árboles
individuales, siendo especialmente útil cuando se tienen muchas
variables y se requiere capturar relaciones complejas y no lineales.

    \begin{tcolorbox}[breakable, size=fbox, boxrule=1pt, pad at break*=1mm,colback=cellbackground, colframe=cellborder]
\prompt{In}{incolor}{56}{\boxspacing}
\begin{Verbatim}[commandchars=\\\{\}]
\PY{c+c1}{\PYZsh{} Definir el espacio de parámetros a probar en el GridSearch para Random Forest}
\PY{n}{param\PYZus{}grid\PYZus{}rf} \PY{o}{=} \PY{p}{\PYZob{}}
    \PY{l+s+s1}{\PYZsq{}}\PY{l+s+s1}{n\PYZus{}estimators}\PY{l+s+s1}{\PYZsq{}}\PY{p}{:} \PY{p}{[}\PY{l+m+mi}{50}\PY{p}{,} \PY{l+m+mi}{100}\PY{p}{,} \PY{l+m+mi}{200}\PY{p}{]}\PY{p}{,}  \PY{c+c1}{\PYZsh{} Número de árboles en el bosque}
    \PY{l+s+s1}{\PYZsq{}}\PY{l+s+s1}{max\PYZus{}depth}\PY{l+s+s1}{\PYZsq{}}\PY{p}{:} \PY{p}{[}\PY{l+m+mi}{5}\PY{p}{,} \PY{l+m+mi}{10}\PY{p}{,} \PY{l+m+mi}{15}\PY{p}{,} \PY{k+kc}{None}\PY{p}{]}\PY{p}{,}   \PY{c+c1}{\PYZsh{} Profundidad máxima de los árboles}
    \PY{l+s+s1}{\PYZsq{}}\PY{l+s+s1}{min\PYZus{}samples\PYZus{}split}\PY{l+s+s1}{\PYZsq{}}\PY{p}{:} \PY{p}{[}\PY{l+m+mi}{2}\PY{p}{,} \PY{l+m+mi}{5}\PY{p}{,} \PY{l+m+mi}{10}\PY{p}{]}\PY{p}{,}  \PY{c+c1}{\PYZsh{} Número mínimo de muestras para dividir un nodo}
    \PY{l+s+s1}{\PYZsq{}}\PY{l+s+s1}{min\PYZus{}samples\PYZus{}leaf}\PY{l+s+s1}{\PYZsq{}}\PY{p}{:} \PY{p}{[}\PY{l+m+mi}{1}\PY{p}{,} \PY{l+m+mi}{2}\PY{p}{,} \PY{l+m+mi}{4}\PY{p}{]}\PY{p}{,}    \PY{c+c1}{\PYZsh{} Número mínimo de muestras por hoja}
    \PY{l+s+s1}{\PYZsq{}}\PY{l+s+s1}{criterion}\PY{l+s+s1}{\PYZsq{}}\PY{p}{:} \PY{p}{[}\PY{l+s+s1}{\PYZsq{}}\PY{l+s+s1}{squared\PYZus{}error}\PY{l+s+s1}{\PYZsq{}}\PY{p}{,} \PY{l+s+s1}{\PYZsq{}}\PY{l+s+s1}{absolute\PYZus{}error}\PY{l+s+s1}{\PYZsq{}}\PY{p}{]}\PY{p}{,}  \PY{c+c1}{\PYZsh{} Criterios de división}
    \PY{l+s+s1}{\PYZsq{}}\PY{l+s+s1}{bootstrap}\PY{l+s+s1}{\PYZsq{}}\PY{p}{:} \PY{p}{[}\PY{k+kc}{True}\PY{p}{,} \PY{k+kc}{False}\PY{p}{]}        \PY{c+c1}{\PYZsh{} Si se usa o no el muestreo bootstrap}
\PY{p}{\PYZcb{}}

\PY{n}{results\PYZus{}df\PYZus{}rf}\PY{p}{,} \PY{n}{best\PYZus{}params\PYZus{}dict\PYZus{}rf} \PY{o}{=} \PY{n}{em}\PY{o}{.}\PY{n}{evaluacion\PYZus{}modelo}\PY{p}{(}\PY{n}{X}\PY{p}{,}\PY{n}{y}\PY{p}{,} \PY{n}{variables\PYZus{}importantes}\PY{p}{,} \PY{n}{RandomForestRegressor}\PY{p}{(}\PY{p}{)}\PY{p}{,} \PY{n}{param\PYZus{}grid\PYZus{}rf}\PY{p}{)}

\PY{c+c1}{\PYZsh{} Mostrar la tabla de resultados}
\PY{n}{display}\PY{p}{(}\PY{n}{results\PYZus{}df\PYZus{}rf}\PY{p}{)}
\end{Verbatim}
\end{tcolorbox}

    
    \begin{Verbatim}[commandchars=\\\{\}]
         Variable Objetivo   Best R²
0          SMI\_VIDA\_delta1  0.779235
1         SMI\_MEDIO\_delta1  0.921084
2       EMPRESAS\_10\_delta1  0.626106
3       EMPRESAS\_20\_delta1  0.677624
4       EMPRESAS\_50\_delta1  0.699365
5        PARO\_1\_AÑO\_delta1  0.814805
6           EMP\_1\_5\_delta1  0.386298
7           PARO\_25\_delta1  0.747459
8   OC\_CONSTRUCCION\_delta1  0.302759
9      OC\_SERVICIOS\_delta1  0.253227
10             PARO\_delta1  0.848125
11       PIB\_CAPITA\_delta1  0.864205
12        PROD\_HORA\_delta1  0.468929
13         CARENCIA\_delta1  0.056056
14   RIESGO\_POBREZA\_delta1  0.088656
15      DESIGUALDAD\_delta1  0.026319
16          PARCIAL\_delta1  0.169404
17              IPC\_delta1  0.938318
18    HORAS\_TRABAJO\_delta1  0.808209
    \end{Verbatim}

    
    En los resultados se observa una clara mejoría en el resultado final, si
bien es cierto que para el caso de CARENCIA, RIESGO\_POBREZA y GINI las
predicciones siguen sin ser especialmente mejores que usar la media.

    \subsubsection{Gradient Boosting}\label{gradient-boosting}

    El modelo de Gradient Boosting es conocido por su capacidad para
capturar relaciones complejas entre las variables y por su alto
rendimiento predictivo. Dado que este trabajo se enfoca en un problema
de predicción con un número limitado de datos, el uso de estos modelos
permite obtener resultados altamente precisos, incluso cuando las
relaciones entre las variables son no lineales y complejas.

Gradient Boosting es una técnica de aprendizaje que combina múltiples
modelos débiles (normalmente árboles de decisión) de manera secuencial
para corregir los errores cometidos en las predicciones anteriores. Los
algoritmos XGBoost y LightGBM son implementaciones avanzadas de boosting
que son especialmente eficientes en términos de tiempo de computación y
capacidad de manejo de grandes volúmenes de datos. Estos modelos se
optimizan iterativamente, lo que mejora la predicción en cada paso, y
son particularmente eficaces cuando las relaciones entre las variables
son complejas y no lineales.

    \begin{tcolorbox}[breakable, size=fbox, boxrule=1pt, pad at break*=1mm,colback=cellbackground, colframe=cellborder]
\prompt{In}{incolor}{57}{\boxspacing}
\begin{Verbatim}[commandchars=\\\{\}]
\PY{c+c1}{\PYZsh{} Definir el espacio de parámetros a probar en el GridSearch para Gradient Boosting}
\PY{n}{param\PYZus{}grid\PYZus{}gb} \PY{o}{=} \PY{p}{\PYZob{}}
    \PY{l+s+s1}{\PYZsq{}}\PY{l+s+s1}{n\PYZus{}estimators}\PY{l+s+s1}{\PYZsq{}}\PY{p}{:} \PY{p}{[}\PY{l+m+mi}{50}\PY{p}{,} \PY{l+m+mi}{100}\PY{p}{,} \PY{l+m+mi}{200}\PY{p}{]}\PY{p}{,}        \PY{c+c1}{\PYZsh{} Número de árboles}
    \PY{l+s+s1}{\PYZsq{}}\PY{l+s+s1}{learning\PYZus{}rate}\PY{l+s+s1}{\PYZsq{}}\PY{p}{:} \PY{p}{[}\PY{l+m+mf}{0.01}\PY{p}{,} \PY{l+m+mf}{0.1}\PY{p}{,} \PY{l+m+mf}{0.2}\PY{p}{]}\PY{p}{,}     \PY{c+c1}{\PYZsh{} Tasa de aprendizaje}
    \PY{l+s+s1}{\PYZsq{}}\PY{l+s+s1}{max\PYZus{}depth}\PY{l+s+s1}{\PYZsq{}}\PY{p}{:} \PY{p}{[}\PY{l+m+mi}{3}\PY{p}{,} \PY{l+m+mi}{5}\PY{p}{,} \PY{l+m+mi}{10}\PY{p}{]}\PY{p}{,}               \PY{c+c1}{\PYZsh{} Profundidad máxima de los árboles}
    \PY{l+s+s1}{\PYZsq{}}\PY{l+s+s1}{min\PYZus{}samples\PYZus{}split}\PY{l+s+s1}{\PYZsq{}}\PY{p}{:} \PY{p}{[}\PY{l+m+mi}{2}\PY{p}{,} \PY{l+m+mi}{5}\PY{p}{,} \PY{l+m+mi}{10}\PY{p}{]}\PY{p}{,}       \PY{c+c1}{\PYZsh{} Número mínimo de muestras para dividir un nodo}
    \PY{l+s+s1}{\PYZsq{}}\PY{l+s+s1}{min\PYZus{}samples\PYZus{}leaf}\PY{l+s+s1}{\PYZsq{}}\PY{p}{:} \PY{p}{[}\PY{l+m+mi}{1}\PY{p}{,} \PY{l+m+mi}{2}\PY{p}{,} \PY{l+m+mi}{4}\PY{p}{]}\PY{p}{,}         \PY{c+c1}{\PYZsh{} Número mínimo de muestras en una hoja}
    \PY{l+s+s1}{\PYZsq{}}\PY{l+s+s1}{subsample}\PY{l+s+s1}{\PYZsq{}}\PY{p}{:} \PY{p}{[}\PY{l+m+mf}{0.8}\PY{p}{,} \PY{l+m+mf}{1.0}\PY{p}{]}                \PY{c+c1}{\PYZsh{} Proporción de muestras usadas para entrenar cada árbol}
\PY{p}{\PYZcb{}}

\PY{n}{results\PYZus{}df\PYZus{}gb}\PY{p}{,} \PY{n}{best\PYZus{}params\PYZus{}dict\PYZus{}gb} \PY{o}{=} \PY{n}{em}\PY{o}{.}\PY{n}{evaluacion\PYZus{}modelo}\PY{p}{(}\PY{n}{X}\PY{p}{,}\PY{n}{y}\PY{p}{,} \PY{n}{variables\PYZus{}importantes}\PY{p}{,} \PY{n}{GradientBoostingRegressor}\PY{p}{(}\PY{p}{)}\PY{p}{,} \PY{n}{param\PYZus{}grid\PYZus{}gb}\PY{p}{)}
\PY{c+c1}{\PYZsh{} Mostrar la tabla de resultados}
\PY{n}{display}\PY{p}{(}\PY{n}{results\PYZus{}df\PYZus{}gb}\PY{p}{)}
\end{Verbatim}
\end{tcolorbox}

    
    \begin{Verbatim}[commandchars=\\\{\}]
         Variable Objetivo   Best R²
0          SMI\_VIDA\_delta1  0.770852
1         SMI\_MEDIO\_delta1  0.921569
2       EMPRESAS\_10\_delta1  0.655831
3       EMPRESAS\_20\_delta1  0.680635
4       EMPRESAS\_50\_delta1  0.703799
5        PARO\_1\_AÑO\_delta1  0.812996
6           EMP\_1\_5\_delta1  0.338875
7           PARO\_25\_delta1  0.754338
8   OC\_CONSTRUCCION\_delta1  0.249949
9      OC\_SERVICIOS\_delta1  0.274759
10             PARO\_delta1  0.838500
11       PIB\_CAPITA\_delta1  0.865254
12        PROD\_HORA\_delta1  0.442303
13         CARENCIA\_delta1  0.040095
14   RIESGO\_POBREZA\_delta1  0.096588
15      DESIGUALDAD\_delta1 -0.007075
16          PARCIAL\_delta1  0.152460
17              IPC\_delta1  0.952963
18    HORAS\_TRABAJO\_delta1  0.804420
    \end{Verbatim}

    
    Similar al caso anterior, se observa mejoría, pero las variables que
estaban a bajos valores de \(R^2\) lo siguen estando

    \subsubsection{SVM}\label{svm}

    Las Máquinas de Soporte Vectorial para regresión (SVR) son útiles cuando
se tienen datos con relaciones no lineales complejas. En este trabajo,
SVR se emplea debido a su capacidad para encontrar una línea de ajuste
que minimice los errores de predicción mientras maximiza el margen de
tolerancia para los puntos de datos. Dado que el conjunto de datos es
relativamente pequeño, SVR es particularmente útil cuando se busca una
solución precisa y robusta para modelar las relaciones no lineales entre
el salario mínimo y los efectos económicos.

Las máquinas de soporte vectorial (SVR) son un enfoque de aprendizaje
supervisado que busca encontrar el mejor hiperplano que minimice los
errores de predicción dentro de un margen de tolerancia determinado. A
través del uso de kernels, SVR puede manejar relaciones no lineales
entre las variables. Este modelo es eficaz en escenarios con pocos datos
y cuando las relaciones entre las variables no pueden ser capturadas
adecuadamente por modelos lineales. SVR es especialmente adecuado cuando
se busca una alta precisión en modelos que incluyen características
complejas.

    \begin{tcolorbox}[breakable, size=fbox, boxrule=1pt, pad at break*=1mm,colback=cellbackground, colframe=cellborder]
\prompt{In}{incolor}{58}{\boxspacing}
\begin{Verbatim}[commandchars=\\\{\}]
\PY{c+c1}{\PYZsh{} Definir el espacio de parámetros a probar en el GridSearch para SVR}
\PY{n}{param\PYZus{}grid\PYZus{}svm} \PY{o}{=} \PY{p}{\PYZob{}}
    \PY{l+s+s1}{\PYZsq{}}\PY{l+s+s1}{C}\PY{l+s+s1}{\PYZsq{}}\PY{p}{:} \PY{p}{[}\PY{l+m+mf}{0.1}\PY{p}{,} \PY{l+m+mi}{1}\PY{p}{,} \PY{l+m+mi}{2}\PY{p}{,} \PY{l+m+mi}{5}\PY{p}{,} \PY{l+m+mi}{10}\PY{p}{,} \PY{l+m+mi}{100}\PY{p}{,} \PY{l+m+mi}{200}\PY{p}{,} \PY{l+m+mi}{500}\PY{p}{]}\PY{p}{,}                 \PY{c+c1}{\PYZsh{} Parámetro de regularización}
    \PY{l+s+s1}{\PYZsq{}}\PY{l+s+s1}{kernel}\PY{l+s+s1}{\PYZsq{}}\PY{p}{:} \PY{p}{[}\PY{l+s+s1}{\PYZsq{}}\PY{l+s+s1}{rbf}\PY{l+s+s1}{\PYZsq{}}\PY{p}{,} \PY{l+s+s1}{\PYZsq{}}\PY{l+s+s1}{linear}\PY{l+s+s1}{\PYZsq{}}\PY{p}{,} \PY{l+s+s1}{\PYZsq{}}\PY{l+s+s1}{sigmoid}\PY{l+s+s1}{\PYZsq{}}\PY{p}{]}\PY{p}{,}     \PY{c+c1}{\PYZsh{} Tipos de kernel}
    \PY{l+s+s1}{\PYZsq{}}\PY{l+s+s1}{gamma}\PY{l+s+s1}{\PYZsq{}}\PY{p}{:} \PY{p}{[}\PY{l+s+s1}{\PYZsq{}}\PY{l+s+s1}{scale}\PY{l+s+s1}{\PYZsq{}}\PY{p}{,} \PY{l+s+s1}{\PYZsq{}}\PY{l+s+s1}{auto}\PY{l+s+s1}{\PYZsq{}}\PY{p}{]}\PY{p}{,}              \PY{c+c1}{\PYZsh{} Parámetro de kernel (\PYZsq{}rbf\PYZsq{}, \PYZsq{}poly\PYZsq{})}
\PY{p}{\PYZcb{}}

\PY{n}{results\PYZus{}df\PYZus{}svm}\PY{p}{,} \PY{n}{best\PYZus{}params\PYZus{}dict\PYZus{}svm} \PY{o}{=} \PY{n}{em}\PY{o}{.}\PY{n}{evaluacion\PYZus{}modelo}\PY{p}{(}\PY{n}{X}\PY{p}{,} \PY{n}{y}\PY{p}{,} \PY{n}{variables\PYZus{}importantes}\PY{p}{,} \PY{n}{SVR}\PY{p}{(}\PY{p}{)}\PY{p}{,} \PY{n}{param\PYZus{}grid\PYZus{}svm}\PY{p}{)}

\PY{c+c1}{\PYZsh{} Mostrar la tabla de resultados}
\PY{n+nb}{print}\PY{p}{(}\PY{l+s+s2}{\PYZdq{}}\PY{l+s+se}{\PYZbs{}n}\PY{l+s+s2}{Resultados de GridSearchCV para SVR:}\PY{l+s+s2}{\PYZdq{}}\PY{p}{)}
\PY{n}{display}\PY{p}{(}\PY{n}{results\PYZus{}df\PYZus{}svm}\PY{p}{)}
\end{Verbatim}
\end{tcolorbox}

    \begin{Verbatim}[commandchars=\\\{\}]

Resultados de GridSearchCV para SVR:
    \end{Verbatim}

    
    \begin{Verbatim}[commandchars=\\\{\}]
         Variable Objetivo   Best R²
0          SMI\_VIDA\_delta1 -0.443526
1         SMI\_MEDIO\_delta1 -0.059896
2       EMPRESAS\_10\_delta1 -0.572743
3       EMPRESAS\_20\_delta1  0.000310
4       EMPRESAS\_50\_delta1  0.309427
5        PARO\_1\_AÑO\_delta1  0.831229
6           EMP\_1\_5\_delta1  0.030032
7           PARO\_25\_delta1  0.611790
8   OC\_CONSTRUCCION\_delta1  0.248385
9      OC\_SERVICIOS\_delta1 -0.035127
10             PARO\_delta1  0.731158
11       PIB\_CAPITA\_delta1 -0.113922
12        PROD\_HORA\_delta1 -0.307339
13         CARENCIA\_delta1 -0.046094
14   RIESGO\_POBREZA\_delta1  0.056512
15      DESIGUALDAD\_delta1  0.105005
16          PARCIAL\_delta1  0.154788
17              IPC\_delta1 -0.075466
18    HORAS\_TRABAJO\_delta1 -3.012158
    \end{Verbatim}

    
    Para el caso de Support Vector Machines está claro que el rendimiento en
general es inferior al obtenido en los modelos previos.

    \subsubsection{Selección de Modelo}\label{selecciuxf3n-de-modelo}

    Habiendo realizado las pruebas pertienentes para todos los modelos,
vamos que no es posible elegir un modelo único con mismos parámetros
para todas las variables objetivos, por lo que escogeremos el mejor
modelo y los mejores parámetros para cada uno.

    \begin{tcolorbox}[breakable, size=fbox, boxrule=1pt, pad at break*=1mm,colback=cellbackground, colframe=cellborder]
\prompt{In}{incolor}{59}{\boxspacing}
\begin{Verbatim}[commandchars=\\\{\}]
\PY{c+c1}{\PYZsh{}Renombramos las columnas y combinamos todos los resultados}
\PY{n}{final\PYZus{}df\PYZus{}results} \PY{o}{=} \PY{n}{results\PYZus{}df\PYZus{}r}\PY{p}{[}\PY{p}{[}\PY{l+s+s1}{\PYZsq{}}\PY{l+s+s1}{Variable Objetivo}\PY{l+s+s1}{\PYZsq{}}\PY{p}{]}\PY{p}{]}
\PY{k}{for} \PY{n}{model}\PY{p}{,}\PY{n}{df\PYZus{}res} \PY{o+ow}{in} \PY{p}{\PYZob{}}\PY{l+s+s1}{\PYZsq{}}\PY{l+s+s1}{Regresión Lineal}\PY{l+s+s1}{\PYZsq{}}\PY{p}{:} \PY{n}{results\PYZus{}df\PYZus{}r}\PY{p}{,} \PY{l+s+s1}{\PYZsq{}}\PY{l+s+s1}{Regresión Lasso}\PY{l+s+s1}{\PYZsq{}}\PY{p}{:} \PY{n}{results\PYZus{}df\PYZus{}rl}\PY{p}{,} \PY{l+s+s1}{\PYZsq{}}\PY{l+s+s1}{Árbol de decisión}\PY{l+s+s1}{\PYZsq{}}\PY{p}{:} \PY{n}{results\PYZus{}df\PYZus{}ad}\PY{p}{,} 
                     \PY{l+s+s1}{\PYZsq{}}\PY{l+s+s1}{Random Forest}\PY{l+s+s1}{\PYZsq{}}\PY{p}{:} \PY{n}{results\PYZus{}df\PYZus{}rf}\PY{p}{,} \PY{l+s+s1}{\PYZsq{}}\PY{l+s+s1}{Gradient Boosting}\PY{l+s+s1}{\PYZsq{}}\PY{p}{:} \PY{n}{results\PYZus{}df\PYZus{}gb}\PY{p}{,} \PY{l+s+s1}{\PYZsq{}}\PY{l+s+s1}{SVM}\PY{l+s+s1}{\PYZsq{}}\PY{p}{:} \PY{n}{results\PYZus{}df\PYZus{}svm}\PY{p}{\PYZcb{}}\PY{o}{.}\PY{n}{items}\PY{p}{(}\PY{p}{)}\PY{p}{:}
    \PY{k}{if} \PY{l+s+s1}{\PYZsq{}}\PY{l+s+s1}{Best R²}\PY{l+s+s1}{\PYZsq{}} \PY{o+ow}{in} \PY{n}{df\PYZus{}res}\PY{o}{.}\PY{n}{columns}\PY{p}{:}
        \PY{n}{res} \PY{o}{=} \PY{n}{df\PYZus{}res}\PY{p}{[}\PY{p}{[}\PY{l+s+s2}{\PYZdq{}}\PY{l+s+s2}{Variable Objetivo}\PY{l+s+s2}{\PYZdq{}}\PY{p}{,} \PY{l+s+s2}{\PYZdq{}}\PY{l+s+s2}{Best R²}\PY{l+s+s2}{\PYZdq{}}\PY{p}{]}\PY{p}{]}
        \PY{n}{res}\PY{o}{.}\PY{n}{rename}\PY{p}{(}\PY{n}{columns} \PY{o}{=} \PY{p}{\PYZob{}}\PY{l+s+s2}{\PYZdq{}}\PY{l+s+s2}{Best R²}\PY{l+s+s2}{\PYZdq{}}\PY{p}{:}\PY{l+s+sa}{f}\PY{l+s+s2}{\PYZdq{}}\PY{l+s+s2}{Best R2 }\PY{l+s+si}{\PYZob{}}\PY{n}{model}\PY{l+s+si}{\PYZcb{}}\PY{l+s+s2}{\PYZdq{}}\PY{p}{\PYZcb{}}\PY{p}{,} \PY{n}{inplace} \PY{o}{=} \PY{k+kc}{True}\PY{p}{)}
    \PY{k}{else}\PY{p}{:}
        \PY{n}{res} \PY{o}{=} \PY{n}{df\PYZus{}res}\PY{p}{[}\PY{p}{[}\PY{l+s+s2}{\PYZdq{}}\PY{l+s+s2}{Variable Objetivo}\PY{l+s+s2}{\PYZdq{}}\PY{p}{,} \PY{l+s+s2}{\PYZdq{}}\PY{l+s+s2}{Mean R²}\PY{l+s+s2}{\PYZdq{}}\PY{p}{]}\PY{p}{]}
        \PY{n}{res}\PY{o}{.}\PY{n}{rename}\PY{p}{(}\PY{n}{columns} \PY{o}{=} \PY{p}{\PYZob{}}\PY{l+s+s2}{\PYZdq{}}\PY{l+s+s2}{Mean R²}\PY{l+s+s2}{\PYZdq{}}\PY{p}{:}\PY{l+s+sa}{f}\PY{l+s+s2}{\PYZdq{}}\PY{l+s+s2}{Best R2 }\PY{l+s+si}{\PYZob{}}\PY{n}{model}\PY{l+s+si}{\PYZcb{}}\PY{l+s+s2}{\PYZdq{}}\PY{p}{\PYZcb{}}\PY{p}{,} \PY{n}{inplace} \PY{o}{=} \PY{k+kc}{True}\PY{p}{)}
    \PY{n}{final\PYZus{}df\PYZus{}results} \PY{o}{=} \PY{n}{final\PYZus{}df\PYZus{}results}\PY{o}{.}\PY{n}{merge}\PY{p}{(}\PY{n}{res}\PY{p}{,} \PY{n}{on} \PY{o}{=} \PY{p}{[}\PY{l+s+s1}{\PYZsq{}}\PY{l+s+s1}{Variable Objetivo}\PY{l+s+s1}{\PYZsq{}}\PY{p}{]}\PY{p}{)}
\PY{c+c1}{\PYZsh{} Encontrar el mejor R² y el modelo correspondiente para cada fila}
\PY{n}{final\PYZus{}df\PYZus{}results}\PY{p}{[}\PY{l+s+s1}{\PYZsq{}}\PY{l+s+s1}{Mejor R2}\PY{l+s+s1}{\PYZsq{}}\PY{p}{]} \PY{o}{=} \PY{n}{final\PYZus{}df\PYZus{}results}\PY{o}{.}\PY{n}{iloc}\PY{p}{[}\PY{p}{:}\PY{p}{,} \PY{l+m+mi}{1}\PY{p}{:}\PY{p}{]}\PY{o}{.}\PY{n}{max}\PY{p}{(}\PY{n}{axis}\PY{o}{=}\PY{l+m+mi}{1}\PY{p}{)}
\PY{n}{final\PYZus{}df\PYZus{}results}\PY{p}{[}\PY{l+s+s1}{\PYZsq{}}\PY{l+s+s1}{Modelo}\PY{l+s+s1}{\PYZsq{}}\PY{p}{]} \PY{o}{=} \PY{n}{final\PYZus{}df\PYZus{}results}\PY{o}{.}\PY{n}{iloc}\PY{p}{[}\PY{p}{:}\PY{p}{,} \PY{l+m+mi}{1}\PY{p}{:}\PY{p}{]}\PY{o}{.}\PY{n}{idxmax}\PY{p}{(}\PY{n}{axis}\PY{o}{=}\PY{l+m+mi}{1}\PY{p}{)}

\PY{c+c1}{\PYZsh{} Crear la tabla final con las tres columnas deseadas}
\PY{n}{final\PYZus{}summary} \PY{o}{=} \PY{n}{final\PYZus{}df\PYZus{}results}\PY{p}{[}\PY{p}{[}\PY{l+s+s1}{\PYZsq{}}\PY{l+s+s1}{Variable Objetivo}\PY{l+s+s1}{\PYZsq{}}\PY{p}{,} \PY{l+s+s1}{\PYZsq{}}\PY{l+s+s1}{Mejor R2}\PY{l+s+s1}{\PYZsq{}}\PY{p}{,} \PY{l+s+s1}{\PYZsq{}}\PY{l+s+s1}{Modelo}\PY{l+s+s1}{\PYZsq{}}\PY{p}{]}\PY{p}{]}
\PY{n}{final\PYZus{}summary}\PY{p}{[}\PY{l+s+s1}{\PYZsq{}}\PY{l+s+s1}{Modelo}\PY{l+s+s1}{\PYZsq{}}\PY{p}{]} \PY{o}{=} \PY{n}{final\PYZus{}summary}\PY{p}{[}\PY{l+s+s1}{\PYZsq{}}\PY{l+s+s1}{Modelo}\PY{l+s+s1}{\PYZsq{}}\PY{p}{]}\PY{o}{.}\PY{n}{str}\PY{o}{.}\PY{n}{replace}\PY{p}{(}\PY{l+s+s1}{\PYZsq{}}\PY{l+s+s1}{Best R2 }\PY{l+s+s1}{\PYZsq{}}\PY{p}{,} \PY{l+s+s1}{\PYZsq{}}\PY{l+s+s1}{\PYZsq{}}\PY{p}{)}
\PY{c+c1}{\PYZsh{} Mostrar el resultado}
\PY{n}{display}\PY{p}{(}\PY{n}{final\PYZus{}summary}\PY{p}{)}
\end{Verbatim}
\end{tcolorbox}

    
    \begin{Verbatim}[commandchars=\\\{\}]
         Variable Objetivo  Mejor R2             Modelo
0          SMI\_VIDA\_delta1  0.779235      Random Forest
1         SMI\_MEDIO\_delta1  0.921569  Gradient Boosting
2       EMPRESAS\_10\_delta1  0.655831  Gradient Boosting
3       EMPRESAS\_20\_delta1  0.680635  Gradient Boosting
4       EMPRESAS\_50\_delta1  0.703799  Gradient Boosting
5        PARO\_1\_AÑO\_delta1  0.831229                SVM
6           EMP\_1\_5\_delta1  0.386298      Random Forest
7           PARO\_25\_delta1  0.754338  Gradient Boosting
8   OC\_CONSTRUCCION\_delta1  0.302759      Random Forest
9      OC\_SERVICIOS\_delta1  0.274759  Gradient Boosting
10             PARO\_delta1  0.848125      Random Forest
11       PIB\_CAPITA\_delta1  0.865254  Gradient Boosting
12        PROD\_HORA\_delta1  0.468929      Random Forest
13         CARENCIA\_delta1  0.056056      Random Forest
14   RIESGO\_POBREZA\_delta1  0.096588  Gradient Boosting
15      DESIGUALDAD\_delta1  0.105005                SVM
16          PARCIAL\_delta1  0.169404      Random Forest
17              IPC\_delta1  0.952963  Gradient Boosting
18    HORAS\_TRABAJO\_delta1  0.808209      Random Forest
    \end{Verbatim}

    
    Observamos que en la mayoría de los casos el mejor modelo es o bien
Random Forest o bien Gradient Boosting, exceptuando el caso del índice
de Gini, donde la regresión lineal es el modelo más adecuado; y los
parados por más de un año, donde SVM obtiene mejores resultados. Dado
que para un estudio del salario mínimo de cara a hacer predicciones para
el año que viene no es preciso obtener resultados rápidamente o en
tiempo real el tiempo de entrenamiento pasa a segundo plano y podemos
seleccionar el mejor modelo basándonos en su capacidad predictiva. En la
tabla a continuación podemos ver cuál es el mejor modelo para cada
variable objetivo.

    \subsection{Interpretación de
variables}\label{interpretaciuxf3n-de-variables}

    Dado que el objetivo de este estudio es analizar el impacto de una
subida del salario mínimo sobre cada variable objetivo en esta sección
nos centraremos en estudiar la importancia de este, así como estudiar
cómo varía la predicción de la variable objetivo si cambiamos la
variable objetivo.

    \begin{tcolorbox}[breakable, size=fbox, boxrule=1pt, pad at break*=1mm,colback=cellbackground, colframe=cellborder]
\prompt{In}{incolor}{60}{\boxspacing}
\begin{Verbatim}[commandchars=\\\{\}]
\PY{c+c1}{\PYZsh{}Creamos el diccionario con cada variable objetivo y sus mejores modelos}
\PY{n}{best\PYZus{}models} \PY{o}{=} \PY{p}{\PYZob{}}\PY{p}{\PYZcb{}}

\PY{k}{for} \PY{n}{target\PYZus{}variable}\PY{p}{,} \PY{n}{best\PYZus{}model} \PY{o+ow}{in} \PY{n+nb}{zip}\PY{p}{(}\PY{n}{final\PYZus{}summary}\PY{p}{[}\PY{l+s+s1}{\PYZsq{}}\PY{l+s+s1}{Variable Objetivo}\PY{l+s+s1}{\PYZsq{}}\PY{p}{]}\PY{p}{,} \PY{n}{final\PYZus{}summary}\PY{p}{[}\PY{l+s+s1}{\PYZsq{}}\PY{l+s+s1}{Modelo}\PY{l+s+s1}{\PYZsq{}}\PY{p}{]}\PY{p}{)}\PY{p}{:}
    \PY{k}{if} \PY{n}{best\PYZus{}model} \PY{o}{==} \PY{l+s+s2}{\PYZdq{}}\PY{l+s+s2}{Regresión Lineal}\PY{l+s+s2}{\PYZdq{}}\PY{p}{:}
        \PY{n}{model} \PY{o}{=} \PY{n}{LinearRegression}\PY{p}{(}\PY{p}{)}
        \PY{n}{model}\PY{o}{.}\PY{n}{fit}\PY{p}{(}\PY{n}{X}\PY{p}{[}\PY{n}{variables\PYZus{}importantes}\PY{p}{[}\PY{n}{target\PYZus{}variable}\PY{p}{]}\PY{p}{]}\PY{p}{,} \PY{n}{y}\PY{p}{[}\PY{n}{target\PYZus{}variable}\PY{p}{]}\PY{p}{)}
    \PY{k}{elif} \PY{n}{best\PYZus{}model} \PY{o}{==} \PY{l+s+s2}{\PYZdq{}}\PY{l+s+s2}{Regresión Lasso}\PY{l+s+s2}{\PYZdq{}}\PY{p}{:}
        \PY{n}{model} \PY{o}{=} \PY{n}{Lasso}\PY{p}{(}\PY{p}{)}
        \PY{n}{model}\PY{o}{.}\PY{n}{fit}\PY{p}{(}\PY{n}{X}\PY{p}{[}\PY{n}{variables\PYZus{}importantes}\PY{p}{[}\PY{n}{target\PYZus{}variable}\PY{p}{]}\PY{p}{]}\PY{p}{,} \PY{n}{y}\PY{p}{[}\PY{n}{target\PYZus{}variable}\PY{p}{]}\PY{p}{)}
    \PY{k}{elif} \PY{n}{best\PYZus{}model} \PY{o}{==} \PY{l+s+s2}{\PYZdq{}}\PY{l+s+s2}{Árbol de decisión}\PY{l+s+s2}{\PYZdq{}}\PY{p}{:}
        \PY{n}{model} \PY{o}{=} \PY{n}{DecisionTreeRegressor}\PY{p}{(}\PY{o}{*}\PY{o}{*}\PY{n}{best\PYZus{}params\PYZus{}dict\PYZus{}ad}\PY{p}{[}\PY{n}{target\PYZus{}variable}\PY{p}{]}\PY{p}{)}
        \PY{n}{model}\PY{o}{.}\PY{n}{fit}\PY{p}{(}\PY{n}{X}\PY{p}{[}\PY{n}{variables\PYZus{}importantes}\PY{p}{[}\PY{n}{target\PYZus{}variable}\PY{p}{]}\PY{p}{]}\PY{p}{,} \PY{n}{y}\PY{p}{[}\PY{n}{target\PYZus{}variable}\PY{p}{]}\PY{p}{)}
    \PY{k}{elif} \PY{n}{best\PYZus{}model} \PY{o}{==} \PY{l+s+s2}{\PYZdq{}}\PY{l+s+s2}{Random Forest}\PY{l+s+s2}{\PYZdq{}}\PY{p}{:}
        \PY{n}{model} \PY{o}{=} \PY{n}{RandomForestRegressor}\PY{p}{(}\PY{o}{*}\PY{o}{*}\PY{n}{best\PYZus{}params\PYZus{}dict\PYZus{}rf}\PY{p}{[}\PY{n}{target\PYZus{}variable}\PY{p}{]}\PY{p}{)}
        \PY{n}{model}\PY{o}{.}\PY{n}{fit}\PY{p}{(}\PY{n}{X}\PY{p}{[}\PY{n}{variables\PYZus{}importantes}\PY{p}{[}\PY{n}{target\PYZus{}variable}\PY{p}{]}\PY{p}{]}\PY{p}{,} \PY{n}{y}\PY{p}{[}\PY{n}{target\PYZus{}variable}\PY{p}{]}\PY{p}{)}
    \PY{k}{elif} \PY{n}{best\PYZus{}model} \PY{o}{==} \PY{l+s+s2}{\PYZdq{}}\PY{l+s+s2}{Gradient Boosting}\PY{l+s+s2}{\PYZdq{}}\PY{p}{:}
        \PY{n}{model} \PY{o}{=} \PY{n}{GradientBoostingRegressor}\PY{p}{(}\PY{o}{*}\PY{o}{*}\PY{n}{best\PYZus{}params\PYZus{}dict\PYZus{}gb}\PY{p}{[}\PY{n}{target\PYZus{}variable}\PY{p}{]}\PY{p}{)}
        \PY{n}{model}\PY{o}{.}\PY{n}{fit}\PY{p}{(}\PY{n}{X}\PY{p}{[}\PY{n}{variables\PYZus{}importantes}\PY{p}{[}\PY{n}{target\PYZus{}variable}\PY{p}{]}\PY{p}{]}\PY{p}{,} \PY{n}{y}\PY{p}{[}\PY{n}{target\PYZus{}variable}\PY{p}{]}\PY{p}{)}
    \PY{k}{elif} \PY{n}{best\PYZus{}model} \PY{o}{==} \PY{l+s+s2}{\PYZdq{}}\PY{l+s+s2}{SVM}\PY{l+s+s2}{\PYZdq{}}\PY{p}{:}
        \PY{n}{model} \PY{o}{=} \PY{n}{SVR}\PY{p}{(}\PY{o}{*}\PY{o}{*}\PY{n}{best\PYZus{}params\PYZus{}dict\PYZus{}svm}\PY{p}{[}\PY{n}{target\PYZus{}variable}\PY{p}{]}\PY{p}{)}
        \PY{n}{model}\PY{o}{.}\PY{n}{fit}\PY{p}{(}\PY{n}{X}\PY{p}{[}\PY{n}{variables\PYZus{}importantes}\PY{p}{[}\PY{n}{target\PYZus{}variable}\PY{p}{]}\PY{p}{]}\PY{p}{,} \PY{n}{y}\PY{p}{[}\PY{n}{target\PYZus{}variable}\PY{p}{]}\PY{p}{)}
    \PY{n}{best\PYZus{}models}\PY{p}{[}\PY{n}{target\PYZus{}variable}\PY{p}{]} \PY{o}{=} \PY{n}{model}
\end{Verbatim}
\end{tcolorbox}

    \begin{tcolorbox}[breakable, size=fbox, boxrule=1pt, pad at break*=1mm,colback=cellbackground, colframe=cellborder]
\prompt{In}{incolor}{61}{\boxspacing}
\begin{Verbatim}[commandchars=\\\{\}]
\PY{n}{p}\PY{o}{.}\PY{n}{plot\PYZus{}importancia\PYZus{}univariable}\PY{p}{(}\PY{n}{best\PYZus{}models}\PY{p}{,} \PY{n}{X}\PY{p}{,} \PY{n}{y}\PY{p}{,} \PY{n}{variables\PYZus{}importantes}\PY{p}{)}
\end{Verbatim}
\end{tcolorbox}

    \begin{center}
    \adjustimage{max size={0.9\linewidth}{0.9\paperheight}}{tfm_project_files/tfm_project_157_0.png}
    \end{center}
    { \hspace*{\fill} \\}
    
    CAMBIAR GINI

Contrario a lo que se esperaría y a lo observado en otros estudios, el
índice GINI no se ve influenciado por las variaciones del salario
mínimo, lo que resulta claramente contraintuitivo teniendo en cuenta que
esperaríamos una clara contribución a reducir la desigualdad de
ingresos.

Para estudiar cómo afecta el salario mínimo a cada una de los variables,
realizaremos dos simulaciones básicas. Estas simulaciones consistirán en
escoger unas variables de partida que no alteraremos y variaremos el
salario mínimo para estudiar cómo cambia la variable objetivo según el
valor del salario mínimo. Para poder apreciar el contraste utilizaremos
datos de 2015 para Madrid (una de la comunidades autónomas más ricas) y
Extremadura (una de las comunidades autónomas más pobres)

    \begin{tcolorbox}[breakable, size=fbox, boxrule=1pt, pad at break*=1mm,colback=cellbackground, colframe=cellborder]
\prompt{In}{incolor}{62}{\boxspacing}
\begin{Verbatim}[commandchars=\\\{\}]
\PY{c+c1}{\PYZsh{}Seleccionamos los dos sets de valores a utilizar}
\PY{n}{df\PYZus{}sets} \PY{o}{=} \PY{n}{df}\PY{p}{[}\PY{n}{variables}\PY{p}{]}
\PY{n}{ccaa\PYZus{}1} \PY{o}{=} \PY{l+s+s2}{\PYZdq{}}\PY{l+s+s2}{Madrid, Comunidad de}\PY{l+s+s2}{\PYZdq{}}
\PY{n}{ccaa\PYZus{}2} \PY{o}{=} \PY{l+s+s2}{\PYZdq{}}\PY{l+s+s2}{Extremadura}\PY{l+s+s2}{\PYZdq{}}
\PY{n}{periodo} \PY{o}{=} \PY{l+m+mi}{2015}
\PY{n}{df\PYZus{}set\PYZus{}1} \PY{o}{=} \PY{n}{df\PYZus{}sets}\PY{p}{[}\PY{p}{(}\PY{n}{df\PYZus{}sets}\PY{p}{[}\PY{l+s+s1}{\PYZsq{}}\PY{l+s+s1}{ccaa}\PY{l+s+s1}{\PYZsq{}}\PY{p}{]} \PY{o}{==} \PY{n}{ccaa\PYZus{}1}\PY{p}{)} \PY{o}{\PYZam{}} \PY{p}{(}\PY{n}{df\PYZus{}sets}\PY{p}{[}\PY{l+s+s1}{\PYZsq{}}\PY{l+s+s1}{periodo}\PY{l+s+s1}{\PYZsq{}}\PY{p}{]} \PY{o}{==} \PY{n}{periodo}\PY{p}{)}\PY{p}{]}\PY{p}{[}\PY{n}{num\PYZus{}var}\PY{p}{]}
\PY{n}{df\PYZus{}set\PYZus{}2} \PY{o}{=} \PY{n}{df\PYZus{}sets}\PY{p}{[}\PY{p}{(}\PY{n}{df\PYZus{}sets}\PY{p}{[}\PY{l+s+s1}{\PYZsq{}}\PY{l+s+s1}{ccaa}\PY{l+s+s1}{\PYZsq{}}\PY{p}{]} \PY{o}{==} \PY{n}{ccaa\PYZus{}2}\PY{p}{)} \PY{o}{\PYZam{}} \PY{p}{(}\PY{n}{df\PYZus{}sets}\PY{p}{[}\PY{l+s+s1}{\PYZsq{}}\PY{l+s+s1}{periodo}\PY{l+s+s1}{\PYZsq{}}\PY{p}{]} \PY{o}{==} \PY{n}{periodo}\PY{p}{)}\PY{p}{]}\PY{p}{[}\PY{n}{num\PYZus{}var}\PY{p}{]}
\end{Verbatim}
\end{tcolorbox}

    \begin{tcolorbox}[breakable, size=fbox, boxrule=1pt, pad at break*=1mm,colback=cellbackground, colframe=cellborder]
\prompt{In}{incolor}{63}{\boxspacing}
\begin{Verbatim}[commandchars=\\\{\}]
\PY{c+c1}{\PYZsh{}Calculamos las predicciones para cada variable objetivo en cada set}
\PY{n}{rango\PYZus{}incrementos} \PY{o}{=} \PY{n}{np}\PY{o}{.}\PY{n}{linspace}\PY{p}{(}\PY{o}{\PYZhy{}}\PY{l+m+mf}{0.15}\PY{p}{,}\PY{l+m+mf}{0.3}\PY{p}{,}\PY{l+m+mi}{200}\PY{p}{)}
\PY{n}{dfs\PYZus{}variables\PYZus{}1} \PY{o}{=} \PY{p}{\PYZob{}}\PY{p}{\PYZcb{}}
\PY{n}{dfs\PYZus{}variables\PYZus{}2} \PY{o}{=} \PY{p}{\PYZob{}}\PY{p}{\PYZcb{}}
\PY{n}{df\PYZus{}set\PYZus{}1c} \PY{o}{=} \PY{n}{df\PYZus{}set\PYZus{}1}\PY{o}{.}\PY{n}{copy}\PY{p}{(}\PY{p}{)}
\PY{n}{df\PYZus{}set\PYZus{}2c} \PY{o}{=} \PY{n}{df\PYZus{}set\PYZus{}2}\PY{o}{.}\PY{n}{copy}\PY{p}{(}\PY{p}{)}
\PY{k}{for} \PY{n}{target\PYZus{}variable}\PY{p}{,} \PY{n}{model} \PY{o+ow}{in} \PY{n}{best\PYZus{}models}\PY{o}{.}\PY{n}{items}\PY{p}{(}\PY{p}{)}\PY{p}{:}
    \PY{c+c1}{\PYZsh{}Calculamos la predicción de la variable para cada caso}
    \PY{n}{var\PYZus{}data\PYZus{}1} \PY{o}{=} \PY{p}{[}\PY{p}{]}
    \PY{n}{var\PYZus{}data\PYZus{}2} \PY{o}{=} \PY{p}{[}\PY{p}{]}
    \PY{k}{for} \PY{n}{inc} \PY{o+ow}{in} \PY{n}{rango\PYZus{}incrementos}\PY{p}{:}
        \PY{n}{df\PYZus{}set\PYZus{}1c}\PY{p}{[}\PY{l+s+s1}{\PYZsq{}}\PY{l+s+s1}{INC\PYZus{}SMI\PYZus{}REAL}\PY{l+s+s1}{\PYZsq{}}\PY{p}{]} \PY{o}{=} \PY{n}{inc}
        \PY{n}{df\PYZus{}set\PYZus{}2c}\PY{p}{[}\PY{l+s+s1}{\PYZsq{}}\PY{l+s+s1}{INC\PYZus{}SMI\PYZus{}REAL}\PY{l+s+s1}{\PYZsq{}}\PY{p}{]} \PY{o}{=} \PY{n}{inc}
        \PY{n}{var\PYZus{}data\PYZus{}1}\PY{o}{.}\PY{n}{append}\PY{p}{(}\PY{n}{model}\PY{o}{.}\PY{n}{predict}\PY{p}{(}\PY{n}{df\PYZus{}set\PYZus{}1c}\PY{p}{[}\PY{n}{variables\PYZus{}importantes}\PY{p}{[}\PY{n}{target\PYZus{}variable}\PY{p}{]}\PY{p}{]}\PY{p}{)}\PY{p}{[}\PY{l+m+mi}{0}\PY{p}{]}\PY{p}{)}
        \PY{n}{var\PYZus{}data\PYZus{}2}\PY{o}{.}\PY{n}{append}\PY{p}{(}\PY{n}{model}\PY{o}{.}\PY{n}{predict}\PY{p}{(}\PY{n}{df\PYZus{}set\PYZus{}2c}\PY{p}{[}\PY{n}{variables\PYZus{}importantes}\PY{p}{[}\PY{n}{target\PYZus{}variable}\PY{p}{]}\PY{p}{]}\PY{p}{)}\PY{p}{[}\PY{l+m+mi}{0}\PY{p}{]}\PY{p}{)}
    \PY{n}{dfs\PYZus{}variables\PYZus{}1}\PY{p}{[}\PY{l+s+s1}{\PYZsq{}}\PY{l+s+s1}{INC\PYZus{}SMI\PYZus{}REAL}\PY{l+s+s1}{\PYZsq{}}\PY{p}{]} \PY{o}{=} \PY{n+nb}{list}\PY{p}{(}\PY{n}{rango\PYZus{}incrementos}\PY{p}{)}
    \PY{n}{dfs\PYZus{}variables\PYZus{}2}\PY{p}{[}\PY{l+s+s1}{\PYZsq{}}\PY{l+s+s1}{INC\PYZus{}SMI\PYZus{}REAL}\PY{l+s+s1}{\PYZsq{}}\PY{p}{]} \PY{o}{=} \PY{n+nb}{list}\PY{p}{(}\PY{n}{rango\PYZus{}incrementos}\PY{p}{)}
    \PY{n}{dfs\PYZus{}variables\PYZus{}1}\PY{p}{[}\PY{n}{target\PYZus{}variable}\PY{p}{]} \PY{o}{=} \PY{n}{var\PYZus{}data\PYZus{}1}
    \PY{n}{dfs\PYZus{}variables\PYZus{}2}\PY{p}{[}\PY{n}{target\PYZus{}variable}\PY{p}{]} \PY{o}{=} \PY{n}{var\PYZus{}data\PYZus{}2}

\PY{n}{df\PYZus{}res\PYZus{}1} \PY{o}{=} \PY{n}{pd}\PY{o}{.}\PY{n}{DataFrame}\PY{p}{(}\PY{n}{dfs\PYZus{}variables\PYZus{}1}\PY{p}{)}
\PY{n}{df\PYZus{}res\PYZus{}2} \PY{o}{=} \PY{n}{pd}\PY{o}{.}\PY{n}{DataFrame}\PY{p}{(}\PY{n}{dfs\PYZus{}variables\PYZus{}2}\PY{p}{)}
\end{Verbatim}
\end{tcolorbox}

    \begin{tcolorbox}[breakable, size=fbox, boxrule=1pt, pad at break*=1mm,colback=cellbackground, colframe=cellborder]
\prompt{In}{incolor}{64}{\boxspacing}
\begin{Verbatim}[commandchars=\\\{\}]
\PY{n}{p}\PY{o}{.}\PY{n}{plot\PYZus{}simulacion}\PY{p}{(}\PY{n}{df\PYZus{}res\PYZus{}1}\PY{p}{,} \PY{n}{df\PYZus{}res\PYZus{}2}\PY{p}{,} \PY{n}{ccaa\PYZus{}1}\PY{p}{,} \PY{n}{ccaa\PYZus{}2}\PY{p}{,} \PY{n}{variables}\PY{p}{)}
\end{Verbatim}
\end{tcolorbox}

    \begin{center}
    \adjustimage{max size={0.9\linewidth}{0.9\paperheight}}{tfm_project_files/tfm_project_161_0.png}
    \end{center}
    { \hspace*{\fill} \\}
    
    Observamos que el posible cambio porcentual en cada variable según la
variación del salario mínimo es, si bien igual en forma (algo esperable
con los modelos utilizados) distintos en sus magnitudes debido a las
diferencias en el resto de variables.

La interpretación de estos resultados es algo complicada pues para
muchos casos hay una alta volatilidad que se puede observar en los
grandes escalones en determinadas variables. Tomemos por ejemplo el caso
del IPC. Vemos que si incrementamos un 4\% el salario mínimo tendríamos
una deflación de algo más del 1.5\%, mientras que subir el salario
mínimo un 7\% resultaría en una inflación de alrededor del 3\% y subirlo
un 30\% resultaría solo en una inflación del 2.8\%, algo que podría
justificarse mediante la suposición de ajustes de mercado para bajos
incrementos y traslado del IPC en regiones altas en las que podemos
introducir el factor de despidos, si bien es cierto que probar esta
causalidad no es trivial.

Esta y otras relaciones que se observan no son necesariamente
inconsistentes, pues no es extraño que bajo ciertas circunstancias una
ligerta diferencia produzca un cambio brusco en los resultados finales,
y más en un sistema económico que depende de múltiples factores.

La escasez de datos en el rango 8-20\% se hace bastante patente en el
modelo, ya que se aprecian zonas planas en las variables que no utilizan
modelos de regresión lineal, muestra de una mala extrapolación a esta
zona. Dado esto, es más interesante observar las variaciones en el rango
-2\% a 5\% de incremento, pues es del que más datos disponemos.

    \begin{tcolorbox}[breakable, size=fbox, boxrule=1pt, pad at break*=1mm,colback=cellbackground, colframe=cellborder]
\prompt{In}{incolor}{65}{\boxspacing}
\begin{Verbatim}[commandchars=\\\{\}]
\PY{n}{cond\PYZus{}1} \PY{o}{=} \PY{p}{(}\PY{o}{\PYZhy{}}\PY{l+m+mf}{0.02} \PY{o}{\PYZlt{}}\PY{o}{=} \PY{n}{df\PYZus{}res\PYZus{}1}\PY{p}{[}\PY{l+s+s1}{\PYZsq{}}\PY{l+s+s1}{INC\PYZus{}SMI\PYZus{}REAL}\PY{l+s+s1}{\PYZsq{}}\PY{p}{]}\PY{p}{)} \PY{o}{\PYZam{}} \PY{p}{(}\PY{n}{df\PYZus{}res\PYZus{}1}\PY{p}{[}\PY{l+s+s1}{\PYZsq{}}\PY{l+s+s1}{INC\PYZus{}SMI\PYZus{}REAL}\PY{l+s+s1}{\PYZsq{}}\PY{p}{]} \PY{o}{\PYZlt{}}\PY{o}{=} \PY{l+m+mf}{0.05}\PY{p}{)}
\PY{n}{cond\PYZus{}2} \PY{o}{=} \PY{p}{(}\PY{o}{\PYZhy{}}\PY{l+m+mf}{0.02} \PY{o}{\PYZlt{}}\PY{o}{=} \PY{n}{df\PYZus{}res\PYZus{}2}\PY{p}{[}\PY{l+s+s1}{\PYZsq{}}\PY{l+s+s1}{INC\PYZus{}SMI\PYZus{}REAL}\PY{l+s+s1}{\PYZsq{}}\PY{p}{]}\PY{p}{)} \PY{o}{\PYZam{}} \PY{p}{(}\PY{n}{df\PYZus{}res\PYZus{}2}\PY{p}{[}\PY{l+s+s1}{\PYZsq{}}\PY{l+s+s1}{INC\PYZus{}SMI\PYZus{}REAL}\PY{l+s+s1}{\PYZsq{}}\PY{p}{]} \PY{o}{\PYZlt{}}\PY{o}{=} \PY{l+m+mf}{0.05}\PY{p}{)}
\PY{n}{p}\PY{o}{.}\PY{n}{plot\PYZus{}simulacion}\PY{p}{(}\PY{n}{df\PYZus{}res\PYZus{}1}\PY{p}{[}\PY{n}{cond\PYZus{}1}\PY{p}{]}\PY{p}{,} \PY{n}{df\PYZus{}res\PYZus{}2}\PY{p}{[}\PY{n}{cond\PYZus{}2}\PY{p}{]}\PY{p}{,} \PY{n}{ccaa\PYZus{}1}\PY{p}{,} \PY{n}{ccaa\PYZus{}2}\PY{p}{,} \PY{n}{variables}\PY{p}{,} \PY{n}{n\PYZus{}columns}\PY{o}{=}\PY{l+m+mi}{5}\PY{p}{)}
\end{Verbatim}
\end{tcolorbox}

    \begin{center}
    \adjustimage{max size={0.9\linewidth}{0.9\paperheight}}{tfm_project_files/tfm_project_163_0.png}
    \end{center}
    { \hspace*{\fill} \\}
    
    En esta ventana se pueden observar relaciones, que, si bien siguen
siendo volátiles, siguen una tendencia más clara y más sencilla de
explicar. Por ejemplo, para el caso de todas las variables de paro y
para el porcentaje de trabajadores de jornada parcial se produce un
descenso mayor a mayor incremento del SMI, lo que puede explicarse como
que en bajos rangos un incremento puede suponer un incentivo mayor para
empezar a trabajar o incrementar el número de horas que se trabaja.

Por otro lado y atendiendo a uno de los principales objetivos de este
estudio, nos interesa saber cómo llegar a un determinado nivel de
salario mínimo de la manera más óptima, es decir, causando el menor
perjuicio posible en determinadas variables. Como se puede observar a
simple vista esta tarea no es trivial, ya lo que para algunas variables
supone el mínimo (máximo) valor, en otras puede alcanzarse el máximo o
quedarse en una zon intermedia.

Aunque obtener un escenario ideal donde todos los resultados mejoran no
es factible, sí que es posible, en base a determinados pesos, usar una
función, que, maximizándola, nos permita obtener una guía de cómo
mejorar los resultados en determinadas variables. A continuación haremos
una simulación a 4 años para el caso de Madrid desde 2015, y lo
compararemos con la evolución real. Para ello supondremos que querremos
centrarnos en incrementar lo más posible la productividad por hora y el
pib a la vez que reducimos el IPC y el paro en los jóvenes menores de 25
años, utilizando la siguiente función:

\(f(SMI) = 0.5\cdot \Delta PIB\_ CAPITA + 0.5\cdot \Delta PROD\_HORA - 0.5 \cdot \Delta PARO\_ 25  - 0.5\cdot \Delta IPC\)

    \begin{tcolorbox}[breakable, size=fbox, boxrule=1pt, pad at break*=1mm,colback=cellbackground, colframe=cellborder]
\prompt{In}{incolor}{ }{\boxspacing}
\begin{Verbatim}[commandchars=\\\{\}]
\PY{k}{def} \PY{n+nf}{fun\PYZus{}opt}\PY{p}{(}\PY{n}{inc\PYZus{}smi}\PY{p}{,} \PY{n}{df\PYZus{}set\PYZus{}1}\PY{p}{,} \PY{n}{best\PYZus{}models}\PY{p}{,} \PY{n}{variables\PYZus{}importantes}\PY{p}{)}\PY{p}{:}
    \PY{n}{x} \PY{o}{=} \PY{n}{sim}\PY{o}{.}\PY{n}{model\PYZus{}prediction}\PY{p}{(}\PY{n}{inc\PYZus{}smi}\PY{p}{,} \PY{n}{df\PYZus{}set\PYZus{}1}\PY{p}{,} \PY{n}{best\PYZus{}models}\PY{p}{,} \PY{n}{variables\PYZus{}importantes}\PY{p}{)}
    \PY{k}{return} \PY{p}{(}\PY{l+m+mf}{0.5}\PY{o}{*}\PY{n}{x}\PY{p}{[}\PY{l+s+s1}{\PYZsq{}}\PY{l+s+s1}{PIB\PYZus{}CAPITA\PYZus{}delta1}\PY{l+s+s1}{\PYZsq{}}\PY{p}{]}\PY{o}{+}\PY{l+m+mf}{0.5}\PY{o}{*}\PY{n}{x}\PY{p}{[}\PY{l+s+s1}{\PYZsq{}}\PY{l+s+s1}{PROD\PYZus{}HORA\PYZus{}delta1}\PY{l+s+s1}{\PYZsq{}}\PY{p}{]}\PY{o}{\PYZhy{}}\PY{l+m+mf}{0.5}\PY{o}{*}\PY{n}{x}\PY{p}{[}\PY{l+s+s1}{\PYZsq{}}\PY{l+s+s1}{PARO\PYZus{}25\PYZus{}delta1}\PY{l+s+s1}{\PYZsq{}}\PY{p}{]}\PY{o}{\PYZhy{}}\PY{l+m+mf}{0.5}\PY{o}{*}\PY{n}{x}\PY{p}{[}\PY{l+s+s1}{\PYZsq{}}\PY{l+s+s1}{IPC\PYZus{}delta1}\PY{l+s+s1}{\PYZsq{}}\PY{p}{]}\PY{p}{)}
\PY{n}{evol\PYZus{}df} \PY{o}{=} \PY{n}{sim}\PY{o}{.}\PY{n}{simulacion\PYZus{}smi}\PY{p}{(}\PY{l+m+mi}{0}\PY{p}{,} \PY{l+m+mf}{0.1}\PY{p}{,} \PY{n}{df\PYZus{}set\PYZus{}1}\PY{p}{,} \PY{n}{fun\PYZus{}opt}\PY{p}{,} \PY{n}{best\PYZus{}models}\PY{p}{,} \PY{n}{variables\PYZus{}importantes}\PY{p}{,} \PY{l+m+mi}{5}\PY{p}{)}
\end{Verbatim}
\end{tcolorbox}

    \begin{tcolorbox}[breakable, size=fbox, boxrule=1pt, pad at break*=1mm,colback=cellbackground, colframe=cellborder]
\prompt{In}{incolor}{ }{\boxspacing}
\begin{Verbatim}[commandchars=\\\{\}]
\PY{c+c1}{\PYZsh{}Comparamos los resultados con la evolución real de Madrid}
\PY{n}{df\PYZus{}madrid} \PY{o}{=} \PY{n}{df\PYZus{}sets}\PY{p}{[}\PY{p}{(}\PY{n}{df\PYZus{}sets}\PY{p}{[}\PY{l+s+s1}{\PYZsq{}}\PY{l+s+s1}{ccaa}\PY{l+s+s1}{\PYZsq{}}\PY{p}{]} \PY{o}{==} \PY{l+s+s1}{\PYZsq{}}\PY{l+s+s1}{Madrid, Comunidad de}\PY{l+s+s1}{\PYZsq{}}\PY{p}{)} \PY{o}{\PYZam{}} \PY{p}{(}\PY{n}{df\PYZus{}sets}\PY{p}{[}\PY{l+s+s1}{\PYZsq{}}\PY{l+s+s1}{periodo}\PY{l+s+s1}{\PYZsq{}}\PY{p}{]} \PY{o}{\PYZgt{}}\PY{o}{=} \PY{l+m+mi}{2015}\PY{p}{)} \PY{o}{\PYZam{}} \PY{p}{(}\PY{n}{df\PYZus{}sets}\PY{p}{[}\PY{l+s+s1}{\PYZsq{}}\PY{l+s+s1}{periodo}\PY{l+s+s1}{\PYZsq{}}\PY{p}{]} \PY{o}{\PYZlt{}}\PY{o}{=} \PY{l+m+mi}{2019}\PY{p}{)}\PY{p}{]}
\PY{n}{var\PYZus{}plot} \PY{o}{=} \PY{p}{[}\PY{l+s+s1}{\PYZsq{}}\PY{l+s+s1}{PIB\PYZus{}CAPITA}\PY{l+s+s1}{\PYZsq{}}\PY{p}{,} \PY{l+s+s1}{\PYZsq{}}\PY{l+s+s1}{HORAS\PYZus{}TRABAJO}\PY{l+s+s1}{\PYZsq{}}\PY{p}{,} \PY{l+s+s1}{\PYZsq{}}\PY{l+s+s1}{PARO\PYZus{}25}\PY{l+s+s1}{\PYZsq{}}\PY{p}{,} \PY{l+s+s1}{\PYZsq{}}\PY{l+s+s1}{IPC}\PY{l+s+s1}{\PYZsq{}}\PY{p}{]}
\PY{n}{p}\PY{o}{.}\PY{n}{plot\PYZus{}real\PYZus{}vs\PYZus{}simulacion}\PY{p}{(}\PY{n}{df\PYZus{}madrid}\PY{p}{,} \PY{n}{evol\PYZus{}df}\PY{p}{,} \PY{n}{variables}\PY{p}{,} \PY{n}{var\PYZus{}plot}\PY{p}{)}
\end{Verbatim}
\end{tcolorbox}

    \begin{center}
    \adjustimage{max size={0.9\linewidth}{0.9\paperheight}}{tfm_project_files/tfm_project_166_0.png}
    \end{center}
    { \hspace*{\fill} \\}
    
    \begin{tcolorbox}[breakable, size=fbox, boxrule=1pt, pad at break*=1mm,colback=cellbackground, colframe=cellborder]
\prompt{In}{incolor}{68}{\boxspacing}
\begin{Verbatim}[commandchars=\\\{\}]
\PY{c+c1}{\PYZsh{}Mostramos los incrementos reales vs simulados}
\PY{n+nb}{print}\PY{p}{(}\PY{l+s+s2}{\PYZdq{}}\PY{l+s+s2}{Incrementos del SMI aplicados:}\PY{l+s+s2}{\PYZdq{}}\PY{p}{)}
\PY{n}{pd}\PY{o}{.}\PY{n}{concat}\PY{p}{(}\PY{p}{[}\PY{n}{df\PYZus{}madrid}\PY{o}{.}\PY{n}{reset\PYZus{}index}\PY{p}{(}\PY{p}{)}\PY{p}{[}\PY{l+s+s1}{\PYZsq{}}\PY{l+s+s1}{INC\PYZus{}SMI\PYZus{}REAL}\PY{l+s+s1}{\PYZsq{}}\PY{p}{]}\PY{p}{,} 
           \PY{n}{evol\PYZus{}df}\PY{o}{.}\PY{n}{rename}\PY{p}{(}\PY{n}{columns}\PY{o}{=}\PY{p}{\PYZob{}}\PY{l+s+s1}{\PYZsq{}}\PY{l+s+s1}{INC\PYZus{}SMI\PYZus{}REAL}\PY{l+s+s1}{\PYZsq{}}\PY{p}{:} \PY{l+s+s1}{\PYZsq{}}\PY{l+s+s1}{INC\PYZus{}SMI\PYZus{}REAL\PYZus{}sim}\PY{l+s+s1}{\PYZsq{}}\PY{p}{\PYZcb{}}\PY{p}{)}\PY{o}{.}\PY{n}{reset\PYZus{}index}\PY{p}{(}\PY{p}{)}\PY{p}{[}\PY{p}{[}\PY{l+s+s1}{\PYZsq{}}\PY{l+s+s1}{INC\PYZus{}SMI\PYZus{}REAL\PYZus{}sim}\PY{l+s+s1}{\PYZsq{}}\PY{p}{]}\PY{p}{]}\PY{p}{]}\PY{p}{,} \PY{n}{axis}\PY{o}{=}\PY{l+m+mi}{1}\PY{p}{)}
\end{Verbatim}
\end{tcolorbox}

    \begin{Verbatim}[commandchars=\\\{\}]
Incrementos del SMI aplicados:
    \end{Verbatim}

            \begin{tcolorbox}[breakable, size=fbox, boxrule=.5pt, pad at break*=1mm, opacityfill=0]
\prompt{Out}{outcolor}{68}{\boxspacing}
\begin{Verbatim}[commandchars=\\\{\}]
   INC\_SMI\_REAL  INC\_SMI\_REAL\_sim
0      0.012649          0.023490
1      0.052239          0.016779
2      0.032959          0.017450
3      0.207562          0.016107
4      0.045728          0.016107
\end{Verbatim}
\end{tcolorbox}
        
    Observamos para el IPC se obtienen resultados exitosos, mientras que el
paro tiene unos resultados algo peores que los reales y los de
productividad real y pib per cápita son sustancialmente peores. Esto es
en parte esperado dada la importancia que tiene el incremento del SMI
para el IPC y la poca influencia que tiene esta misma variable en el PIB
per cápita y en la productividad por hora.

Si quisiésemos obtener mejores resultados en predicciones de variables
cuyo incremento está poco condicionado con el salario mínimo
requeriríamos de predicciones a largo plazo, en la que tendríamos que
hallar qué variables podemos alterar mediante el salario mínimo para
poder crear las condiciones ideales que permitan obtener los mejores
resultados para la variable que queremos tratar, lo que requeriría de un
cálculo mucho más complejo que dependería además de la ventana temporal
a usar (predicción a 3, 5, 10 años etc.) y en la que es posible que no
exista una solución óptima.

Mirando el caso del PIB per cápita se observa que una de las variables
más importantes es el valor del IPC, lo que implica que IPCs más bajos
resultan en valores más bajos del PIB per cápita. Esto resulta en
dificultades a largo plazo para la mejora simultánea de ambas variables,
pues una mayor inflación o deflación resultaría eventualmente en unas
malas condiciones para el incremento del PIB. Para casos como este, es
posible hacer redeficiones de las funciones que permitan capturar estas
particularidades. Por ejemplo, podemos sustituir el componente del IPC
por un componente que sume un valor fijo si la inflación está entre el
0\% y el 2\%, de manera que demos más margen al resto de variables:

    \begin{tcolorbox}[breakable, size=fbox, boxrule=1pt, pad at break*=1mm,colback=cellbackground, colframe=cellborder]
\prompt{In}{incolor}{96}{\boxspacing}
\begin{Verbatim}[commandchars=\\\{\}]
\PY{k}{def} \PY{n+nf}{fun\PYZus{}opt}\PY{p}{(}\PY{n}{inc\PYZus{}smi}\PY{p}{,} \PY{n}{df\PYZus{}set\PYZus{}1}\PY{p}{,} \PY{n}{best\PYZus{}models}\PY{p}{,} \PY{n}{variables\PYZus{}importantes}\PY{p}{)}\PY{p}{:}
    \PY{n}{x} \PY{o}{=} \PY{n}{sim}\PY{o}{.}\PY{n}{model\PYZus{}prediction}\PY{p}{(}\PY{n}{inc\PYZus{}smi}\PY{p}{,} \PY{n}{df\PYZus{}set\PYZus{}1}\PY{p}{,} \PY{n}{best\PYZus{}models}\PY{p}{,} \PY{n}{variables\PYZus{}importantes}\PY{p}{)}
    \PY{k}{if} \PY{l+m+mi}{0} \PY{o}{\PYZlt{}} \PY{n}{x}\PY{p}{[}\PY{l+s+s1}{\PYZsq{}}\PY{l+s+s1}{IPC\PYZus{}delta1}\PY{l+s+s1}{\PYZsq{}}\PY{p}{]} \PY{o}{\PYZlt{}} \PY{l+m+mf}{0.02}\PY{p}{:}
        \PY{n}{plus} \PY{o}{=} \PY{l+m+mf}{0.05}
    \PY{k}{else}\PY{p}{:}
        \PY{n}{plus} \PY{o}{=} \PY{l+m+mi}{0}
    \PY{k}{return} \PY{p}{(}\PY{l+m+mf}{0.5}\PY{o}{*}\PY{n}{x}\PY{p}{[}\PY{l+s+s1}{\PYZsq{}}\PY{l+s+s1}{PIB\PYZus{}CAPITA\PYZus{}delta1}\PY{l+s+s1}{\PYZsq{}}\PY{p}{]}\PY{o}{+}\PY{l+m+mf}{0.5}\PY{o}{*}\PY{n}{x}\PY{p}{[}\PY{l+s+s1}{\PYZsq{}}\PY{l+s+s1}{PROD\PYZus{}HORA\PYZus{}delta1}\PY{l+s+s1}{\PYZsq{}}\PY{p}{]}\PY{o}{\PYZhy{}}\PY{l+m+mf}{0.5}\PY{o}{*}\PY{n}{x}\PY{p}{[}\PY{l+s+s1}{\PYZsq{}}\PY{l+s+s1}{PARO\PYZus{}25\PYZus{}delta1}\PY{l+s+s1}{\PYZsq{}}\PY{p}{]}\PY{o}{+}\PY{n}{plus}\PY{p}{)}
\PY{n}{evol\PYZus{}df} \PY{o}{=} \PY{n}{sim}\PY{o}{.}\PY{n}{simulacion\PYZus{}smi}\PY{p}{(}\PY{l+m+mi}{0}\PY{p}{,} \PY{l+m+mf}{0.1}\PY{p}{,} \PY{n}{df\PYZus{}set\PYZus{}1}\PY{p}{,} \PY{n}{fun\PYZus{}opt}\PY{p}{,} \PY{n}{best\PYZus{}models}\PY{p}{,} \PY{n}{variables\PYZus{}importantes}\PY{p}{,} \PY{l+m+mi}{5}\PY{p}{)}
\PY{n}{df\PYZus{}madrid} \PY{o}{=} \PY{n}{df\PYZus{}sets}\PY{p}{[}\PY{p}{(}\PY{n}{df\PYZus{}sets}\PY{p}{[}\PY{l+s+s1}{\PYZsq{}}\PY{l+s+s1}{ccaa}\PY{l+s+s1}{\PYZsq{}}\PY{p}{]} \PY{o}{==} \PY{l+s+s1}{\PYZsq{}}\PY{l+s+s1}{Madrid, Comunidad de}\PY{l+s+s1}{\PYZsq{}}\PY{p}{)} \PY{o}{\PYZam{}} \PY{p}{(}\PY{n}{df\PYZus{}sets}\PY{p}{[}\PY{l+s+s1}{\PYZsq{}}\PY{l+s+s1}{periodo}\PY{l+s+s1}{\PYZsq{}}\PY{p}{]} \PY{o}{\PYZgt{}}\PY{o}{=} \PY{l+m+mi}{2015}\PY{p}{)} \PY{o}{\PYZam{}} \PY{p}{(}\PY{n}{df\PYZus{}sets}\PY{p}{[}\PY{l+s+s1}{\PYZsq{}}\PY{l+s+s1}{periodo}\PY{l+s+s1}{\PYZsq{}}\PY{p}{]} \PY{o}{\PYZlt{}}\PY{o}{=} \PY{l+m+mi}{2019}\PY{p}{)}\PY{p}{]}
\PY{n}{var\PYZus{}plot} \PY{o}{=} \PY{p}{[}\PY{l+s+s1}{\PYZsq{}}\PY{l+s+s1}{PIB\PYZus{}CAPITA}\PY{l+s+s1}{\PYZsq{}}\PY{p}{,} \PY{l+s+s1}{\PYZsq{}}\PY{l+s+s1}{PROD\PYZus{}HORA}\PY{l+s+s1}{\PYZsq{}}\PY{p}{,} \PY{l+s+s1}{\PYZsq{}}\PY{l+s+s1}{PARO\PYZus{}25}\PY{l+s+s1}{\PYZsq{}}\PY{p}{,} \PY{l+s+s1}{\PYZsq{}}\PY{l+s+s1}{IPC}\PY{l+s+s1}{\PYZsq{}}\PY{p}{]}
\PY{n}{p}\PY{o}{.}\PY{n}{plot\PYZus{}real\PYZus{}vs\PYZus{}simulacion}\PY{p}{(}\PY{n}{df\PYZus{}madrid}\PY{p}{,} \PY{n}{evol\PYZus{}df}\PY{p}{,} \PY{n}{variables}\PY{p}{,} \PY{n}{var\PYZus{}plot}\PY{p}{)}
\end{Verbatim}
\end{tcolorbox}

    \begin{center}
    \adjustimage{max size={0.9\linewidth}{0.9\paperheight}}{tfm_project_files/tfm_project_169_0.png}
    \end{center}
    { \hspace*{\fill} \\}
    
    Ahora vemos que la evolución es mucho mejor para el PIB per cápita, si
bien es cierto que para poder obtener mejor resultados en todas las
variables (aún a corto plazo) puede ser preciso incluir reglas muy
complejas en la función para capturar todos los tejemanejes entre la
realización de los objetivos y la relación que tiene el modelo con las
variables.

    \section{Conclusiones}\label{conclusiones}

    En el presente trabajo se han desarrollado modelos específicos para cada
variable de interés con el fin de analizar y explicar su evolución
frente a incrementos del salario mínimo. Estos modelos no pretenden
justificar de manera categórica las decisiones de aumentar o reducir el
salario mínimo, ni afirmar con rotundidad los efectos que estos cambios
pueden tener en un sistema económico y social complejo, donde convergen
múltiples factores. Sin embargo, permiten identificar patrones
relevantes que deben ser considerados en el diseño de políticas
públicas.

Los resultados obtenidos sugieren que los modelos son efectivos a la
hora de capturar las diferencias en la evolución de las variables
económicas entre comunidades autónomas con distintos niveles de
desarrollo. Aunque existen similitudes en los patrones observados entre
las regiones más prósperas y aquellas con economías más frágiles,
también se evidencian diferencias notables en cuanto a magnitud y
dinámica. Estas divergencias se deben a las condiciones iniciales de
cada región y a su capacidad para absorber los efectos de un aumento del
salario mínimo.

Estos hallazgos indican que una estrategia adecuada para unas
comunidades podría no serlo para otras, lo que sugiere que una
aplicación más granular del salario mínimo podría ser beneficiosa a
nivel nacional. Así, es posible establecer unas condiciones de partida y
una función que refleje la importancia y el peso de las variables a
optimizar, obteniendo así una estimación orientativa del incremento del
salario mínimo ideal para maximizar dicha función y lograr los mejores
resultados en un horizonte de un año.

No obstante, es importante señalar que, aunque se han seleccionado
variables representativas del contexto económico de la época, el
análisis no logra capturar completamente todas las dinámicas
subyacentes. En particular, la relación entre las variaciones del IPC y
el salario mínimo parece estar influenciada por el periodo analizado
(2008-2014), caracterizado por una profunda crisis económica. Durante
esos años, los aumentos reales del salario mínimo fueron limitados o
inexistentes, lo que podría explicar el comportamiento deflacionario
inicial del IPC ante incrementos salariales moderados. Por el contrario,
los aumentos más sustanciales del salario mínimo ocurrieron en años
posteriores, reflejándose en un comportamiento más inflacionario del IPC
para incrementos de mayor magnitud.

En conclusión, las simulaciones realizadas permiten identificar
estrategias óptimas para implementar subidas del salario mínimo,
maximizando los beneficios y minimizando los efectos adversos. Estas
estrategias ofrecen una guía preliminar para la formulación de políticas
laborales, proporcionando un marco que pondera la relevancia de las
distintas variables económicas en juego. De este modo, se puede avanzar
hacia un diseño de políticas más equilibrado y adaptado a las realidades
socioeconómicas de cada región.

    


    % Add a bibliography block to the postdoc
    
    
    
\end{document}
